\begin{frame}
    \frametitle{Components of GLIF: GF}
    \enablepart{hlgf}
    \includestandalone[width=\textwidth]{fig/glif-architecture}
\end{frame}

\begin{frame}[fragile]
    \frametitle{Components of GLIF: Grammatical Framework \cite{GF:on}}
    \begin{itemize}
        \item Specialized for developing natural language grammars
        \item Separates abstract and concrete syntax\par
            \quad\lstinline[language=GF]|sentence : NP -> VP -> S;               --abstract|\par
            \quad\lstinline[language=GF]|sentence np vp = np.s ++ vp.s!np.n;   --concrete|
        \item Abstract syntax based on type theory
        \item Comes with large library \com{$\ge 36$ languages}
    \end{itemize}

    FIGURE: german/concrete syntax to AST
\end{frame}

\begin{frame}
    \frametitle{Components of GLIF: MMT}
    \enablepart{hlmmt}
    \includestandalone[width=\textwidth]{fig/glif-architecture}
\end{frame}

\begin{frame}[fragile]
    \frametitle{Components of GLIF: MMT}
    \lstset{frame=single}
    \begin{itemize}   
        \item Modular logic development and knowledge representation
        \item Foundation-independent\com{we use LF}
        \item We will use it to:
        \begin{itemize}
            \item { \only<2>{\bf\color{hlfont}} represent abstract syntax }
            \item { \only<3>{\bf\color{hlfont}} specify target logic and domain theory}
            \item { \only<4>{\bf\color{hlfont}} specify semantics construction}
        \end{itemize}
    \end{itemize}
    \lstset{basicstyle=\footnotesize\ttfamily}
    
    \vspace{1em}
    \begin{minipage}[t][4cm]{\textwidth}
        \centering
        \only<2>{% Used in slides/glif-components.tex
% Has to be in separate file...
\begin{minipage}[t]{0.4\textwidth}
    \parbox[t][1em][t]{\textwidth}{\centering\bf GF}\par
    \begin{lstlisting}[language=GF,linewidth=\textwidth]
cat
  NP; VP; S;
fun
  make_S :
    NP -> VP -> S;
    \end{lstlisting}
\end{minipage}
\begin{minipage}[t]{0.1\textwidth}\vskip3em \centering\LARGE$\mapsto$\end{minipage}
\begin{minipage}[t]{0.35\textwidth}
    \parbox[t][1em][t]{\textwidth}{\centering\bf MMT}\par
    \begin{lstlisting}[language=MMT,linewidth=\textwidth]
NP : type
VP : type
S  : type
make_S :
  NP \raa VP \raa S \end{lstlisting}
\end{minipage}
}
        \only<3>{% Used in slides/glif-components.tex
% Has to be in separate file...
\begin{minipage}[t]{0.45\textwidth}
    \parbox[t][1em][t]{\textwidth}{\centering\bf Logic}\par
    \begin{lstlisting}[language=MMT,linewidth=\textwidth]
o : type //propositions
\neg : o \raa o
\wedge : o \raa o \raa o
\vee : o \raa o \raa o

\iota : type //individuals
\forall : (\iota \raa o) \raa o
\exists : (\iota \raa o) \raa o \end{lstlisting}
\end{minipage}\hskip2em
\begin{minipage}[t]{0.3\textwidth}
    \parbox[t][1em][t]{\textwidth}{\centering\bf Domain Theory}\par
    \begin{lstlisting}[language=MMT,linewidth=\textwidth]
paint : \iota \raa o
quiet : \iota \raa o
ahmed : \iota
berta : \iota \end{lstlisting}

    \footnotesize
    \vspace{1.5em}
    \hskip-1em
    idea:
    $\forall f$
    or $\forall \lambda x.f(x)$\par
    \hskip-1em instead of $\forall x.f(x)$
\end{minipage}
}
        \only<4>{\textbf{Semantics Construction}

\textit{map symbols in abstract syntax to terms in logic/domain theory}

\vspace{0.5em}
\begin{minipage}[t]{0.4\textwidth}
\parbox[t][1em][t]{\textwidth}{\centering Simple setting}\par
\ifpart{switchtomathexample}{
    \lstinputlisting[language=MMT]{slides/misc/snippets/s000.txt}
}{
    \lstinputlisting[language=MMT]{slides/misc/snippets/s001.txt}
}
\end{minipage}
\hspace{1em}
\begin{minipage}[t]{0.5\textwidth}
\parbox[t][1em][t]{\textwidth}{\centering More advanced}\par
\ifpart{switchtomathexample}{
    \lstinputlisting[language=MMT]{slides/misc/snippets/s002.txt}
}{
    \lstinputlisting[language=MMT]{slides/misc/snippets/s003.txt}
}
\end{minipage}
}
    \end{minipage}
\end{frame}
