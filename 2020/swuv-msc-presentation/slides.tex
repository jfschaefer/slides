\documentclass[aspectratio=169]{beamer}
\usepackage{standalone}

\usepackage{stmaryrd}
\usepackage{listings}

\usepackage[hyperref=auto,style=alphabetic,backend=bibtex]{biblatex}
\addbibresource{kwarc.bib}
\usepackage{appendixnumberbeamer}
\usepackage{tikz}
\usepackage{tikz-qtree}
\usetikzlibrary{arrows.meta}

\usetheme{Pittsburgh}
\setbeamertemplate{footline}[frame number]
\setbeamertemplate{navigation symbols}{}
\usecolortheme{beaver}
\setbeamertemplate{frametitle}[default][left]
\setbeamersize{text margin left=3em}

\usepackage[forbeamer]{utils/basic}
\usepackage{utils/colors}
\usepackage{utils/operators}
\usepackage{utils/lstmisc}
\usepackage{utils/lstmmt}

\lstset{basicstyle=\ttfamily}
\lstset{commentstyle=\itshape\color{commentfont}}

\title{{\footnotesize\itshape --- Master's Thesis Presentation ---}\\[1pt] Prototyping NLU Pipelines\\[3pt] \normalsize A Type-Theoretical Framework}

\author{Jan Frederik Schaefer}
\institute{FAU Erlangen-N\"urnberg}
\date{\textbf{Seminar Wissensrepr\"asentation und -verarbeitung} \\ \textit{presented virtually due to COVID-19} \\ December 16, 2020 }

\begin{document}
\frame\titlepage

% \begin{frame}
%     \frametitle{Motivation}
%     TODO: Tease with example applications (without any details)\\
%     then: let's take a step back - how do 
% \end{frame}

\begin{frame}
    \frametitle{Natural Language Understanding (NLU)}
    \textbf{My definition:}\\
    NLU means translating natural language into a formal semantic representation.

    \vspace{3em}
    \makebox[9cm]{\str{How many people live in Slovakia?}} $\leadsto\;\;$ \str{5.458 million}

    \vspace{1.5em}
    \makebox[9cm]{\str{Do more people live in Slovakia than in Thailand?}} $\leadsto\;\;$ ???


%     \vspace{2em}
%     \centering
%     \begin{tikzpicture}
%         \node (nbnd) at (1.2,0.7) {useless};
%         \node (bnd) at (3.5,0.7) {\begin{tabular}{c}also in-\\teresting\end{tabular}};
%         \node (bd) at (3.5,2.0) {AI-complete?};
%         \node (nbd) at (1.2,2.0) {\begin{tabular}{c}this\\work\end{tabular}};
%         \draw[->,very thick] (-0.3,0) -- (5,0);
%         \draw[->,very thick] (0,-0.3) -- (0,2.7);
%         \node at (4.0, -0.3) {\itshape breadth};
%         \node[rotate=90] at (-0.3,1.8) {\itshape depth};
%     \end{tikzpicture}
\end{frame}


\begin{frame}
    \frametitle{Method of Fragments}
    \only<1-2>{
        \centering
        \only<1>{\def\fraglevel{0}}
        \only<2>{\def\fraglevel{1}}
        \includestandalone{fig/montague-fragments}

        \begin{minipage}[t][2cm]{\textwidth}\vspace{1em}
            How do we get from messy language to formal logic?\\[0.5em]
            \emph{Montague}~\cite{Montague:efl70}: Look at a ``nice'' subset
            and map into logic.
        \end{minipage}
    }

    \only<3>{
        \centering
        \def\fraglevel{1}
        \includestandalone{fig/montague-fragments}
        
        \begin{minipage}[t][2cm]{0.6\textwidth}\vspace{1em}
            \str{Ahmed paints and Berta is quiet.}\\[0.5em]
            \str{Ahmed doesn't paint.}
        \end{minipage}\hfill
        \begin{minipage}[t][2cm]{0.39\textwidth}\vspace{1em}
            $p(a) \wedge q(b)$\\[0.5em]
            $\neg p(a)$
        \end{minipage}
    }

    \only<4>{
        \centering
        \def\fraglevel{2}
        \includestandalone{fig/montague-fragments}
        
        \begin{minipage}[t][2cm]{0.6\textwidth}\vspace{1em}
            \str{Every student paints and is quiet.}\\[0.5em]
            \str{Nobody paints.}
        \end{minipage}\hfill
        \begin{minipage}[t][2cm]{0.39\textwidth}\vspace{1em}
            $\forall x.s(x) \Rightarrow (p(x) \wedge q(x))$\\[0.5em]
            $\neg \exists x.p(x)$
        \end{minipage}
    }

    \only<5>{
        \centering
        \def\fraglevel{3}
        \includestandalone{fig/montague-fragments}

        \begin{minipage}[t][2cm]{0.6\textwidth}\vspace{1em}
            \str{Ahmed isn't allowed to paint.}\\[0.5em]
            \str{Ahmed and Berta must paint.}
        \end{minipage}\hfill
        \begin{minipage}[t][2cm]{0.39\textwidth}\vspace{1em}
            $\neg\lozenge p(a)$\\[0.5em]
            $(\square p(a)) \wedge \square p(b)$
        \end{minipage}
    }
\end{frame}


\begin{frame}
    \frametitle{Method of Fragments}
    {\color{hlfont}Hand-waving} is problematic:

    \hspace{2em}\str{Ahmed paints. He is quiet.}
    {$\quad\stackrel{?}{\leadsto}\quad$ \color{logicfont} $p(a)\wedge q(a)$}

    \vspace{1.2em}
    {\color{hlfont}Montague}: Specify
    \begin{itemize}
        \item grammar,\com{fixes NL subset}
        \item target logic,
        \item semantics construction.\com{maps parse trees to logic}
    \end{itemize}

    {
        \centering
        \vspace{0.3em}
        {\itshape\footnotesize Example from~\cite{Montague:tptoqi73}}

        \vspace{0.2em}\fbox{\includegraphics[trim=0 0 0 80,clip,width=0.7\textwidth]{fig/montague-tptoqioe.png}}

        \vspace{1.2em}
        Claim: That doesn't scale well $\leadsto$ \textbf{We need {\color{hlfont}prototyping}!}
    }

    % \newcommand\VP{\text{\upshape\tiny VP}}
    % \hspace{2em}$\llbracket\text{\strplain{$P_{\VP}$ and $Q_{\VP}$}}\rrbracket_{\VP} = \lambda x. \llbracket\text{\strplain{$P_{\VP}$}}\rrbracket(x) \wedge \llbracket\text{\strplain{$Q_{\VP}$}}\rrbracket(x)$
\end{frame}


\begin{frame}
    \frametitle{NLU Prototyping}
    \begin{itemize}
        \item Traditionally done in Prolog/Haskell
        \begin{itemize}
            \item[$\raa$] requires a lot of work
        \end{itemize}
            \item A dedicated framework might be better
        \begin{itemize}
            \item[$\raa$] only partial solutions exist
        \end{itemize}
        \item Can we combine existing partial solutions?\com{Research Question}
        \begin{itemize}
            \item[$\leadsto$] GLIF
        \end{itemize}
    \end{itemize}
\end{frame}



\begin{frame}
    \frametitle{Components of GLIF: GF}
    \enablepart{hlgf}
    \includestandalone[width=\textwidth]{fig/glif-architecture}
\end{frame}

\begin{frame}[fragile]
    \frametitle{Components of GLIF: Grammatical Framework \cite{GF:on}}
    \begin{itemize}
        \item Specialized for developing natural language grammars
        \item Separates abstract and concrete syntax\par
            \quad\lstinline[language=GF]|sentence : NP -> VP -> S;               --abstract|\par
            \quad\lstinline[language=GF]|sentence np vp = np.s ++ vp.s!np.n;   --concrete|
        \item Abstract syntax based on type theory
        \item Comes with large library \com{$\ge 36$ languages}
    \end{itemize}

    \vspace{2em}
    \hspace{2.5em}\begin{tikzpicture}
        \node(eng) at (-2,0.7) {\str{Ahmed paints}};
        \node(ger) at (-2,-0.7) {\str{Ahmed zeichnet}};
        \node(ast) at (4,0) {
                    \color{logicfont!50!nlfont}
                    \resizebox{2.cm}{!}{\tikzset{edge from parent/.append style={very thick}}
                        \Tree [ .\texttt{sentence} \texttt{ahmed} \texttt{paint} ]
                    }
                };
        \draw[<->, thick] (eng) to node[above,rotate=-9]{Eng. concr. syn.} (ast);
        \draw[<->, thick] (ger) to node[below,rotate=9]{Ger. concr. syn.} (ast);
    \end{tikzpicture}
\end{frame}

\begin{frame}
    \frametitle{Components of GLIF: MMT}
    \enablepart{hlmmt}
    \includestandalone[width=\textwidth]{fig/glif-architecture}
\end{frame}

\begin{frame}[fragile]
    \frametitle{Components of GLIF: MMT}
    \lstset{frame=single}
    \begin{itemize}   
        \item Modular logic development and knowledge representation
        \item Not specialized in one logical framework \com{we use LF}
        \item We will use MMT to:
        \begin{itemize}
            \item { \only<2>{\bf\color{hlfont}} represent abstract syntax }
            \item { \only<3>{\bf\color{hlfont}} specify target logic and domain theory}
            \item { \only<4>{\bf\color{hlfont}} specify semantics construction}
        \end{itemize}
    \end{itemize}
    \lstset{basicstyle=\footnotesize\ttfamily}
    
    \vspace{1em}
    \begin{minipage}[t][4cm]{\textwidth}
        \centering
        \only<2>{% Used in slides/glif-components.tex
% Has to be in separate file...
\begin{minipage}[t]{0.4\textwidth}
    \parbox[t][1em][t]{\textwidth}{\centering\bf GF}\par
    \begin{lstlisting}[language=GF,linewidth=\textwidth]
cat
  NP; VP; S;
fun
  make_S :
    NP -> VP -> S;
    \end{lstlisting}
\end{minipage}
\begin{minipage}[t]{0.1\textwidth}\vskip3em \centering\LARGE$\mapsto$\end{minipage}
\begin{minipage}[t]{0.35\textwidth}
    \parbox[t][1em][t]{\textwidth}{\centering\bf MMT}\par
    \begin{lstlisting}[language=MMT,linewidth=\textwidth]
NP : type
VP : type
S  : type
make_S :
  NP \raa VP \raa S \end{lstlisting}
\end{minipage}
}
        \only<3>{% Used in slides/glif-components.tex
% Has to be in separate file...
\begin{minipage}[t]{0.45\textwidth}
    \parbox[t][1em][t]{\textwidth}{\centering\bf Logic}\par
    \begin{lstlisting}[language=MMT,linewidth=\textwidth]
o : type //propositions
\neg : o \raa o
\wedge : o \raa o \raa o
\vee : o \raa o \raa o

\iota : type //individuals
\forall : (\iota \raa o) \raa o
\exists : (\iota \raa o) \raa o \end{lstlisting}
\end{minipage}\hskip2em
\begin{minipage}[t]{0.3\textwidth}
    \parbox[t][1em][t]{\textwidth}{\centering\bf Domain Theory}\par
    \begin{lstlisting}[language=MMT,linewidth=\textwidth]
paint : \iota \raa o
quiet : \iota \raa o
ahmed : \iota
berta : \iota \end{lstlisting}

    \footnotesize
    \vspace{1.5em}
    \hskip-1em
    idea:
    $\forall f$
    or $\forall \lambda x.f(x)$\par
    \hskip-1em instead of $\forall x.f(x)$
\end{minipage}
}
        \only<4>{\textbf{Semantics Construction}

\textit{map symbols in abstract syntax to terms in logic/domain theory}

\vspace{0.5em}
\begin{minipage}[t]{0.4\textwidth}
\parbox[t][1em][t]{\textwidth}{\centering Simple setting}\par
\ifpart{switchtomathexample}{
    \lstinputlisting[language=MMT]{slides/misc/snippets/s000.txt}
}{
    \lstinputlisting[language=MMT]{slides/misc/snippets/s001.txt}
}
\end{minipage}
\hspace{1em}
\begin{minipage}[t]{0.5\textwidth}
\parbox[t][1em][t]{\textwidth}{\centering More advanced}\par
\ifpart{switchtomathexample}{
    \lstinputlisting[language=MMT]{slides/misc/snippets/s002.txt}
}{
    \lstinputlisting[language=MMT]{slides/misc/snippets/s003.txt}
}
\end{minipage}
}
    \end{minipage}
\end{frame}

\begin{frame}[fragile]
    \frametitle{Example: Parsing + Semantics Construction}
    {\centering\str{Ahmed and Berta paint}\par}\vspace{1em}
    \hspace{0.49\textwidth}$\downarrow_{\text{parsing}}$\par\vspace{1em}
    {\centering\color{logicfont!50!nlfont} sentence (andNP ahmed berta) paint\par}\vspace{1em}
    \hspace{0.49\textwidth}$\downarrow_{\text{semantics construction}}$\par\vspace{1em}
    {\color{logicfont}\footnotesize\lstinline[language=MMT]|(\lambdan.\lambdav.n v) ((\lambdaa.\lambdab.\lambdap.a p \wedge b p) (\lambdap.p ahmed) (\lambdap.p berta)) paint|}\par\vspace{1em}
    \hspace{0.49\textwidth}$\downarrow_{\text{$\beta$-reduction}}$\par\vspace{1em}
    \hspace{7.5em}{\color{logicfont}\small\lstinline[language=MMT]|paint ahmed \wedge paint berta|}
\end{frame}

\begin{frame}
    \frametitle{Components of GLIF: ELPI}
    \enablepart{hlelpi}
    \includestandalone[width=\textwidth]{fig/glif-architecture}
\end{frame}

\begin{frame}[fragile]
    \frametitle{Components of GLIF: ELPI}
    \begin{itemize}
        \item Implementation and extension of $\lambda$Prolog\com{$\approx$ Prolog + HOAS}
        \item MMT can generate logic signatures
        \item First experiments with prover generation
        \item Generic inference/reasoning step after semantics construction
    \end{itemize}
    \lstset{basicstyle=\footnotesize\ttfamily}

    \vspace{1em}
    \begin{minipage}[t]{\textwidth}
        \centering
        \begin{minipage}[t]{0.5\textwidth}
            \begin{lstlisting}[language=ELPI,frame=single]
kind o type.
type not o -> o.
type and o -> o -> o.

kind i type.
type forall (i -> o) -> o.
            \end{lstlisting}
        \end{minipage}
    \end{minipage}
\end{frame}

\begin{frame}
    \frametitle{Components of GLIF: Jupyter}
    \enablepart{hljupyter}
    \includestandalone[width=\textwidth]{fig/glif-architecture}
\end{frame}

\begin{frame}
    \frametitle{Components of GLIF: Jupyter}
    \begin{itemize}
        \item Unified, notebook-based interface
        \item Supports implementation and testing
        \item Useful for prototype, demos, teaching, \dots
    \end{itemize}

    \centering
    \vspace{1.5em}
    \fbox{\includegraphics[trim={0 0 20cm 5.7cm},clip,width=0.7\textwidth]{img/screenshot-glif-1.png}}
\end{frame}


\providedisablepart{showscreenshot}
\begin{frame}[fragile]
    \def\mybox#1{\square_{#1}}
    \def\mydia#1{\lozenge_{#1}}
    \def\sfiven{{S5_n}}
    \frametitle{Example: Epistemic Q\&A}
    \centering
    \strplain{\makebox[9.5cm][l]{John knows that Mary or Eve knows that Ping has a dog.}\makebox[1.5em]{\upshape($S_1$)}\\
              \makebox[9.5cm][l]{Mary doesn't know if Ping has a dog.}\makebox[1.5em]{\upshape($S_2$)}\\
              \makebox[9.5cm][l]{Does Eve know if Ping has a dog?}\makebox[1.5em]{\upshape($Q$)}}

    {\color{logicfont}
        \begin{align*}
            S_1 &= \mybox{john}(\mybox{mary} hd(ping)\vee \mybox{eve}hd(ping))\\
            S_2 &= \neg(\mybox{mary}hd(ping) \vee \mybox{mary}\neg hd(ping))\\
            Q &= \mybox{eve}hd(ping) \vee \mybox{eve}\neg hd(ping)
        \end{align*}
    }

    \begin{table}
        \begin{tabular}{l l}
            $S_1, S_2 \vdash_\sfiven Q$\quad      &$\leadsto$\quad yes\\
            $S_1, S_2 \vdash_\sfiven \neg Q$\quad &$\leadsto$\quad no\\
            else &$\leadsto$\quad maybe
        \end{tabular}
    \end{table}
    \ifpart{showscreenshot}{\only<2>{
        \begin{tikzpicture}[overlay,remember picture]
            \fill[gray!80,opacity=0.8] (current page.north west) rectangle (current page.south east);
            \node at (current page.center) { \includegraphics[width=0.9\textwidth]{img/screenshot-glif-4.png} };
        \end{tikzpicture}
    }}{}
\end{frame}


\begin{frame}
    \frametitle{Example: Controlled Natural Languages}
    \begin{itemize}
        \item Formal languages
        \item that are a subset of natural language
        \item and have fixed semantics\com{formal verification, \dots}
    \end{itemize}

    \vspace{1em}
    \str{$S$ is a subset of every set iff $S$ is empty}

    $\leadsto$ {\color{logicfont}$(\forall V_{new}. set(V_{new}) \Rightarrow subset(V_S, V_{new})) \Leftrightarrow empty(V_S)$}

    \pause
    \vspace{1.5em}
    Use inference for disambiguation:

    \vspace{1em}
    \only<2>{\disablepart{crossout}}
    \only<3>{\enablepart{crossout}}
    \includestandalone[width=\textwidth]{fig/cnl-simple-discard} 
\end{frame}


\begin{frame}[fragile]
    \frametitle{Example: Translation}
    \begin{itemize}
        \item Two German words for \str{cousin}, depending on the gender
        \item Two entries in abstract syntax: \verb|cousin_female| and \verb|cousin_male|
        \item Use inference to discard ASTs
    \end{itemize}
    
    \vspace{2em}
    \small
    \begin{minipage}[t][5cm]{\textwidth}
        \begin{tikzpicture}
            \node(eng) at (-4,1) {\parbox{4.2cm}{\str{Kim is Ahmed's cousin and the father of Grace}}};
                \node(ger1) at (-4,-0.5) {\parbox{4.2cm}{\str{Kim ist Ahmeds {\upshape\bf Cousine} und Graces Vater}}};
                \node(ger2) at (-4,-2.0) {\parbox{4.2cm}{\str{Kim ist Ahmeds {\upshape\bf Cousin} und Graces Vater}}};
            \node(ast1) at (0,1) {AST$_1$};
            \node(ast2) at (0,-1) {AST$_2$};
            \only<2->{
                \node(log1) at (3,1) {\color{logicfont} \parbox{2.2cm}{$female(kim) \wedge$ $male(kim)$}};
                \node(log2) at (3,-1) {\color{logicfont} \parbox{2.2cm}{$male(kim) \wedge$ $male(kim)$}};
            }
            \draw[->,thick] (eng) -- (ast1);
            \draw[->,thick] (eng) -- (ast2);
                \draw[->,thick] (ast1) -- (ger1);
                \draw[->,thick] (ast2) -- (ger2);
            \only<2->{
                \draw[->,thick] (ast1) -- (log1);
                \draw[->,thick] (ast2) -- (log2);
            }
            \only<3>{
                \draw[ultra thick,red] (2,1.5) -- (4,0.5);
                \draw[ultra thick,red] (2,0.5) -- (4,1.5);
                
                \draw[ultra thick,red] (-0.5,1.5) -- (0.5,0.5);
                \draw[ultra thick,red] (-0.5,0.5) -- (0.5,1.5);

                \draw[ultra thick,red] (-6,0.0) -- (-2,-1.0);
                \draw[ultra thick,red] (-6,-1.0) -- (-2,0.0);
            }
        \end{tikzpicture}
    \end{minipage}
\end{frame}


\begin{frame}[fragile]
    \frametitle{Example: Tableaux Machine~\cite{KohKol:ramgpm03}}
    \begin{itemize}
        \item Can use tableaux for model generation
        \item Tableau machine: pick ``best'' branch as model and continue there with next sentence
            \com{like a human?}
    \end{itemize}

    \vspace{1.5em}
    \begin{minipage}[t][4.5cm]{\textwidth}
        \only<1>{\setshowlevel{1}}
        \only<2>{\setshowlevel{2}}
        \only<3>{\setshowlevel{3}}
        % \only<4>{\setshowlevel{4}}
        \includestandalone{fig/tab-machine-simple}
    \end{minipage}
\end{frame}

\begin{frame}
    \frametitle{Example: Tableaux Machine}
    \only<1>{\setshowlevel{1}}
    \only<2>{\setshowlevel{2}}
    \only<3>{\setshowlevel{3}}
    \only<4>{\setshowlevel{4}}
    \makebox[\linewidth]{\includestandalone[scale=0.9]{fig/tab-machine-complex}}
\end{frame}


\begin{frame}
    \frametitle{Conclusion}
    \begin{itemize}
        \item GLIF is a tool for prototyping natural language understanding pipelines
        \item Combines existing, declarative frameworks
    \end{itemize}

    \vspace{2em}
    \includestandalone[width=\textwidth]{fig/glif-architecture}
\end{frame}

\appendix

\begin{frame}[allowframebreaks,t]
    \frametitle{References}
    \printbibliography
\end{frame}

\end{document}
