\def\gfsemconstrpicAST{$\text{AST}$}
\def\gfsemconstrpicLOG{$\text{String}_\text{Logic}$}
\def\gfsemconstrpicArrow{\ttfamily linearize}
\def\gfsemconstrpic{
    {\centering
        \begin{tikzpicture}[xscale=1.2]
            \tikzset{ll/.style={line width=0.7pt}}
            \draw[ll,fill=nlbg] (-5.0,-0.5) rectangle (-3.0,0.5);
            \draw[ll,fill=nlbg!50!logicbg] (-1.0,-0.5) rectangle (1.0,0.5);
            \draw[ll,fill=logicbg] (3.0,-0.5) rectangle (5.0,0.5);
            \node[] at (-4.0, 0.0) {$\text{String}_\text{Eng}$};
            \node[] at (0.0, 0.0) {\gfsemconstrpicAST};
            \node[] at (4.0, 0.0) {\gfsemconstrpicLOG};
            \draw[ll, ->] (-3,0) to node[above]{\footnotesize \ttfamily parse} (-1,0);
            \draw[ll, ->] (1,0) to node[above]{\footnotesize \gfsemconstrpicArrow} (3,0);
        \end{tikzpicture}\par}
}

\begin{frame}
    \frametitle{Method of Fragments}
    \textbf{How can we implement such fragments (with the help of GF)?}

    \vspace{0.5em}
    \begin{itemize}
        \item Use GF for parsing
        \item 4 ideas how to continue
    \end{itemize}

    \vspace{1.5em}

    \def\gfsemconstrpicAST{$\text{AST}$}
    \def\gfsemconstrpicArrow{???}
    \def\gfsemconstrpicLOG{Logical Expr.}
    \gfsemconstrpic{}
\end{frame}


\begin{frame}[fragile]
    \frametitle{Idea 1: Concrete Syntax for Logic}
    \gfsemconstrpic{}

    \vspace{0.5em}
    \only<1-2>{
        \lstinline[language=GLIFcmd,stringstyle=\color{nlfg},keepspaces=true]!parse -lang=Eng "Ahmed paints and Berta is quiet" | linearize -lang=Logic!\\
        {\color{logicfg}\ttfamily p ( a ) \lstmmtboldmath$\wedge$ q ( b )}
    }
    \only<3-4>{
        \lstinline[language=GLIFcmd,stringstyle=\color{nlfg},keepspaces=true]!parse -lang=Eng "Ahmed is quiet and paints" | linearize -lang=Logic!\\
        \only<3>{\color{hlfg}\ttfamily q ( a ) \lstmmtboldmath$\wedge$ p ( a ) ???}
        \only<4>{\color{logicfg}\ttfamily (λx. q (x) \lstmmtboldmath$\wedge$ p (x)) ( a )\com{$\leadsto_\beta q(a)\wedge p(a)$}}
    }

    \onslide<2-4>{
    \begin{columns}[t]
        \begin{column}{0.38\textwidth}
            \begin{mdframed}[backgroundcolor=logicbg!50!nlbg, innerleftmargin=2.5, innerrightmargin=0.2, innertopmargin=0.2,innerbottommargin=0.2, outerlinewidth=0.7]
                \only<1-3>{\onslide<2-3>{
                    \footnotesize\lstinputlisting[linewidth=6.0cm,language=GF]{slides/misc/snippets/s006.txt}
                }}
                \only<4>{
                    \footnotesize\lstinputlisting[linewidth=6.0cm,language=GF]{slides/misc/snippets/s008.txt}
                }
            \end{mdframed}
        \end{column}
        \begin{column}{0.54\textwidth}
            \begin{mdframed}[backgroundcolor=logicbg, innerleftmargin=2.5, innerrightmargin=0.2, innertopmargin=0.2,innerbottommargin=0.2, outerlinewidth=0.7]
                \only<1-3>{\onslide<2-3>{
                    \footnotesize\lstinputlisting[linewidth=6.0cm,language=GF,alsoletter={\\},literate={\\wedge}{\lstmmtboldmath$\wedge$}1,stringstyle=\color{logicfg}]{slides/misc/snippets/s007.txt}
                }}
                \only<4>{
                    \footnotesize\lstinputlisting[linewidth=6.0cm,language=GF,alsoletter={\\},literate={\\wedge}{\lstmmtboldmath$\wedge$}1{\\lambda}{$\uplambda$}1,stringstyle=\color{logicfg}]{slides/misc/snippets/s009.txt}
                }
            \end{mdframed}
        \end{column}
    \end{columns}
    }
\end{frame}

\begin{frame}[fragile]
    \frametitle{Idea 2: Abstract Syntax for Logic}
    \def\gfsemconstrpicAST{$\text{AST}_\text{NL}$}
    \def\gfsemconstrpicArrow{\ttfamily pt -compute}
    \def\gfsemconstrpicLOG{$\text{AST}_\text{Logic}$}
    \gfsemconstrpic{}

    \begin{columns}[t]
        \begin{column}{0.42\textwidth}
            \begin{mdframed}[backgroundcolor=logicbg, innerleftmargin=2.5, innerrightmargin=0.2, innertopmargin=0.2,innerbottommargin=0.2, outerlinewidth=0.7]
                \footnotesize\lstinputlisting[linewidth=3.5cm,language=GF]{slides/misc/snippets/s010.txt}
            \end{mdframed}
        \end{column}
        \begin{column}{0.58\textwidth}
            \onslide<2-3>{
                \begin{mdframed}[backgroundcolor=white, innerleftmargin=2.5, innerrightmargin=0.2, innertopmargin=0.2,innerbottommargin=0.2, outerlinewidth=0.7]
                    \footnotesize\lstinputlisting[linewidth=6.0cm,language=GF]{slides/misc/snippets/s011.txt}
                \end{mdframed}
            }
        \end{column}
    \end{columns}

    \vspace{1.0em}
    \onslide<3>{
        \lstinline[language=GLIFcmd,stringstyle=\color{nlfg},keepspaces=true]!parse -lang=Eng -cat=Prop "Ahmed is quiet and paints" | put_tree -compute!\\
        {\color{logicfg}\ttfamily and (q a) (p a)}
    }
\end{frame}

\begin{frame}
    \frametitle{All 4 Ideas}
    \begin{enumerate}
        \item \textbf{Use another concrete syntax:}\par
            \gfsemconstrpic{}
        \item \textbf{Use another abstract syntax:}\par
            \def\gfsemconstrpicAST{$\text{AST}_\text{NL}$}
            \def\gfsemconstrpicArrow{\ttfamily pt -compute}
            \def\gfsemconstrpicLOG{$\text{AST}_\text{Logic}$}
            \gfsemconstrpic{}
        \pause
        \item \textbf{Use a general-purpose programming language:}\par
            \def\gfsemconstrpicAST{$\text{AST}$}
            \def\gfsemconstrpicArrow{Haskell code}
            \def\gfsemconstrpicLOG{Logical Expr.}
            \gfsemconstrpic{}
        \pause
        \item \textbf{Use a logic development framework:}\par
            \def\gfsemconstrpicAST{$\text{AST}$}
            \def\gfsemconstrpicArrow{MMT stuff}
            \def\gfsemconstrpicLOG{MMT term}
            \gfsemconstrpic{}
    \end{enumerate}
\end{frame}


\begin{frame}
    \frametitle{MMT}
    \begin{itemize}
        \item Tool for logic development and knowledge representation
        \item Developed by KWARC group\com{we are biased towards MMT}
        \item ``Bring your own logic''
        \item Implement syntax, semantics, calculi\com{``rapid prototyping''}
        \item Lots of logics already implemented\com{LATIN2 project}
        \item Supports type theory underlying GF
        \item Three steps:
            \begin{enumerate}
                \item Represent abstract syntax in MMT\com{is automated}
                \item Declare target logic
                \item Declare mapping from 1.\ to 2.
            \end{enumerate}
    \end{itemize}
\end{frame}

% beamer doesn't like it otherwise
\def\parseconstructlstinline{\lstinline[language=GLIFcmd,stringstyle=\color{nlfg},keepspaces=true]!parse -lang=Eng "Ahmed is quiet and paints" | construct view=Semantics!}

\begin{frame}[fragile]
    \frametitle{MMT}
    \begin{enumerate}
        \item {\only<1>{\color{hlfg}\bfseries} Represent abstract syntax in MMT}\com{is automated}
        \item {\only<2-3>{\color{hlfg}\bfseries} Declare target logic}
        \item {\only<4>{\color{hlfg}\bfseries} Declare mapping from 1.\ to 2.}
    \end{enumerate}

    \begin{columns}[t]
        \only<1>{
            \begin{column}{0.48\textwidth}
                \footnotesize\lstinputlisting[linewidth=6.0cm,language=GF]{slides/misc/snippets/s012.txt}
            \end{column}
            \begin{column}{0.48\textwidth}
                \footnotesize\lstinputlisting[linewidth=6.0cm,language=MMT]{slides/misc/snippets/s013.txt}
            \end{column}
        }
        \only<2-3>{
            \begin{column}{0.43\textwidth}
                \footnotesize\lstinputlisting[linewidth=6.0cm,language=MMT]{slides/misc/snippets/s014.txt}
            \end{column}
            \begin{column}{0.53\textwidth}
                \only<3>{
                    \footnotesize\lstinputlisting[linewidth=6.0cm,language=MMT]{slides/misc/snippets/s015.txt}
                }
            \end{column}
        }
        \only<4-5>{
            \begin{column}{0.5\textwidth}
                \footnotesize\lstinputlisting[linewidth=6.0cm,language=MMT]{slides/misc/snippets/s016.txt}
            \end{column}
        }
    \end{columns}
    \only<5>{
        \vspace{0.3cm}
        \parseconstructlstinline\\
        {\color{logicfg}\ttfamily q(a)\lstmmtboldmath$\wedge$p(a)}
    }
    \vspace{10cm}\par
\end{frame}

