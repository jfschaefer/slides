\documentclass[aspectratio=169]{beamer}
\usepackage{standalone}

\usepackage{stmaryrd}
\usepackage{listings}
\usepackage{bussproofs}

\usepackage[hyperref=auto,style=alphabetic,backend=bibtex]{biblatex}
\addbibresource{kwarcpubs.bib}
\addbibresource{extpubs.bib}
\addbibresource{extcrossrefs.bib}
\addbibresource{bib.bib}
\usepackage{appendixnumberbeamer}
\usepackage{tikz}
\usepackage{tikz-qtree}
\usetikzlibrary{arrows.meta}
\usetikzlibrary{mmt}
\usetikzlibrary{docicon}

\usetheme{Pittsburgh}
% \setbeamertemplate{footline}[frame number]
\setbeamertemplate{footline}{\hfill\insertframenumber\,/\,\inserttotalframenumber\quad\strut}
\setbeamertemplate{navigation symbols}{}
\usecolortheme{beaver}
\setbeamertemplate{frametitle}[default][left]
% \setbeamersize{text margin left=3em}

\usepackage{utils/colors}
\usepackage[forbeamer]{utils/basic}
\usepackage{utils/operators}
\usepackage{utils/mylstmisc}
\usepackage{utils/lstmmt}

\lstset{basicstyle=\ttfamily}
\lstset{commentstyle=\itshape\color{commentfont}}

\title{GLIF: A Framework for Prototyping Symbolic Natural Language Understanding}

\author{Jan Frederik Schaefer}
\institute{FAU Erlangen-N\"urnberg}
\date{\textbf{Prospects of Formal Mathematics -- Bridging between informal and formal}\\Hausdorff Research Institute for Mathematics\\Bonn\\July 9, 2024}

\begin{document}

\frame\titlepage




\begin{frame}
    \frametitle{Method of Fragments}
    \only<1-2>{
        \centering
        \only<1>{\def\fraglevel{0}}
        \only<2>{\def\fraglevel{1}}
        \includestandalone{fig/montague-fragments}

        \begin{minipage}[t][2cm]{\textwidth}\vspace{1em}
            How do we get from messy language to formal logic?\\[0.5em]
            \emph{Montague}~\cite{Montague:efl70}: Look at a ``nice'' subset
            and map into logic.
        \end{minipage}
    }

    \only<3>{
        \centering
        \def\fraglevel{1}
        \includestandalone{fig/montague-fragments}
        
        \begin{minipage}[t][2cm]{0.6\textwidth}\vspace{1em}
            \str{Ahmed paints and Berta is quiet.}\\[0.5em]
            \str{Ahmed doesn't paint.}
        \end{minipage}\hfill
        \begin{minipage}[t][2cm]{0.39\textwidth}\vspace{1em}
            $p(a) \wedge q(b)$\\[0.5em]
            $\neg p(a)$
        \end{minipage}
    }

    \only<4>{
        \centering
        \def\fraglevel{2}
        \includestandalone{fig/montague-fragments}
        
        \begin{minipage}[t][2cm]{0.6\textwidth}\vspace{1em}
            \str{Every student paints and is quiet.}\\[0.5em]
            \str{Nobody paints.}
        \end{minipage}\hfill
        \begin{minipage}[t][2cm]{0.39\textwidth}\vspace{1em}
            $\forall x.s(x) \Rightarrow (p(x) \wedge q(x))$\\[0.5em]
            $\neg \exists x.p(x)$
        \end{minipage}
    }

    \only<5>{
        \centering
        \def\fraglevel{3}
        \includestandalone{fig/montague-fragments}

        \begin{minipage}[t][2cm]{0.6\textwidth}\vspace{1em}
            \str{Ahmed isn't allowed to paint.}\\[0.5em]
            \str{Ahmed and Berta must paint.}
        \end{minipage}\hfill
        \begin{minipage}[t][2cm]{0.39\textwidth}\vspace{1em}
            $\neg\lozenge p(a)$\\[0.5em]
            $(\square p(a)) \wedge \square p(b)$
        \end{minipage}
    }
\end{frame}


\begin{frame}
    \frametitle{Method of Fragments}
    {\color{hlfont}Hand-waving} is problematic:

    \hspace{2em}\str{Ahmed paints. He is quiet.}
    {$\quad\stackrel{?}{\leadsto}\quad$ \color{logicfont} $p(a)\wedge q(a)$}

    \vspace{1.2em}
    {\color{hlfont}Montague}: Specify
    \begin{itemize}
        \item grammar,\com{fixes NL subset}
        \item target logic,
        \item semantics construction.\com{maps parse trees to logic}
    \end{itemize}

    {
        \centering
        \vspace{0.3em}
        {\itshape\footnotesize Example from~\cite{Montague:tptoqi73}}

        \vspace{0.2em}\fbox{\includegraphics[trim=0 0 0 80,clip,width=0.7\textwidth]{fig/montague-tptoqioe.png}}

        \vspace{1.2em}
        Claim: That doesn't scale well $\leadsto$ \textbf{We need {\color{hlfont}prototyping}!}
    }

    % \newcommand\VP{\text{\upshape\tiny VP}}
    % \hspace{2em}$\llbracket\text{\strplain{$P_{\VP}$ and $Q_{\VP}$}}\rrbracket_{\VP} = \lambda x. \llbracket\text{\strplain{$P_{\VP}$}}\rrbracket(x) \wedge \llbracket\text{\strplain{$Q_{\VP}$}}\rrbracket(x)$
\end{frame}


\begin{frame}
    \frametitle{NLU Prototyping}
    \begin{itemize}
        \item Traditionally done in Prolog/Haskell
        \begin{itemize}
            \item[$\raa$] requires a lot of work
        \end{itemize}
            \item A dedicated framework might be better
        \begin{itemize}
            \item[$\raa$] only partial solutions exist
        \end{itemize}
        \item Can we combine existing partial solutions?\com{Research Question}
        \begin{itemize}
            \item[$\leadsto$] GLIF
        \end{itemize}
    \end{itemize}
\end{frame}



{
    \enablepart{sometimesnarrow}
    \disablepart{showjupyter}
    \disablepart{showepistemicexample}
    \enablepart{mentionprovergen}
    \disablepart{showspa}
    \enablepart{sageexample}
    % \enablepart{forthelexample}
    \disablepart{intensionalexample}
    \disablepart{introduceelpi}
    \begin{frame}
    \frametitle{Components of GLIF: GF}
    \enablepart{hlgf}
    \includestandalone[width=\textwidth]{fig/glif-architecture}
\end{frame}

\begin{frame}[fragile]
    \frametitle{Components of GLIF: Grammatical Framework \cite{GF:on}}
    \begin{itemize}
        \item Specialized for developing natural language grammars
        \item Separates abstract and concrete syntax\par
            \quad\lstinline[language=GF]|sentence : NP -> VP -> S;               --abstract|\par
            \quad\lstinline[language=GF]|sentence np vp = np.s ++ vp.s!np.n;   --concrete|
        \item Abstract syntax based on type theory
        \item Comes with large library \com{$\ge 36$ languages}
    \end{itemize}

    \vspace{2em}
    \hspace{2.5em}\begin{tikzpicture}
        \node(eng) at (-2,0.7) {\str{Ahmed paints}};
        \node(ger) at (-2,-0.7) {\str{Ahmed zeichnet}};
        \node(ast) at (4,0) {
                    \color{logicfont!50!nlfont}
                    \resizebox{2.cm}{!}{\tikzset{edge from parent/.append style={very thick}}
                        \Tree [ .\texttt{sentence} \texttt{ahmed} \texttt{paint} ]
                    }
                };
        \draw[<->, thick] (eng) to node[above,rotate=-9]{Eng. concr. syn.} (ast);
        \draw[<->, thick] (ger) to node[below,rotate=9]{Ger. concr. syn.} (ast);
    \end{tikzpicture}
\end{frame}

\begin{frame}
    \frametitle{Components of GLIF: MMT}
    \enablepart{hlmmt}
    \includestandalone[width=\textwidth]{fig/glif-architecture}
\end{frame}

\begin{frame}[fragile]
    \frametitle{Components of GLIF: MMT}
    \lstset{frame=single}
    \begin{itemize}   
        \item Modular logic development and knowledge representation
        \item Not specialized in one logical framework \com{we use LF}
        \item We will use MMT to:
        \begin{itemize}
            \item { \only<2>{\bf\color{hlfont}} represent abstract syntax }
            \item { \only<3>{\bf\color{hlfont}} specify target logic and domain theory}
            \item { \only<4>{\bf\color{hlfont}} specify semantics construction}
        \end{itemize}
    \end{itemize}
    \lstset{basicstyle=\footnotesize\ttfamily}
    
    \vspace{1em}
    \begin{minipage}[t][4cm]{\textwidth}
        \centering
        \only<2>{% Used in slides/glif-components.tex
% Has to be in separate file...
\begin{minipage}[t]{0.4\textwidth}
    \parbox[t][1em][t]{\textwidth}{\centering\bf GF}\par
    \begin{lstlisting}[language=GF,linewidth=\textwidth]
cat
  NP; VP; S;
fun
  make_S :
    NP -> VP -> S;
    \end{lstlisting}
\end{minipage}
\begin{minipage}[t]{0.1\textwidth}\vskip3em \centering\LARGE$\mapsto$\end{minipage}
\begin{minipage}[t]{0.35\textwidth}
    \parbox[t][1em][t]{\textwidth}{\centering\bf MMT}\par
    \begin{lstlisting}[language=MMT,linewidth=\textwidth]
NP : type
VP : type
S  : type
make_S :
  NP \raa VP \raa S \end{lstlisting}
\end{minipage}
}
        \only<3>{% Used in slides/glif-components.tex
% Has to be in separate file...
\begin{minipage}[t]{0.45\textwidth}
    \parbox[t][1em][t]{\textwidth}{\centering\bf Logic}\par
    \begin{lstlisting}[language=MMT,linewidth=\textwidth]
o : type //propositions
\neg : o \raa o
\wedge : o \raa o \raa o
\vee : o \raa o \raa o

\iota : type //individuals
\forall : (\iota \raa o) \raa o
\exists : (\iota \raa o) \raa o \end{lstlisting}
\end{minipage}\hskip2em
\begin{minipage}[t]{0.3\textwidth}
    \parbox[t][1em][t]{\textwidth}{\centering\bf Domain Theory}\par
    \begin{lstlisting}[language=MMT,linewidth=\textwidth]
paint : \iota \raa o
quiet : \iota \raa o
ahmed : \iota
berta : \iota \end{lstlisting}

    \footnotesize
    \vspace{1.5em}
    \hskip-1em
    idea:
    $\forall f$
    or $\forall \lambda x.f(x)$\par
    \hskip-1em instead of $\forall x.f(x)$
\end{minipage}
}
        \only<4>{\textbf{Semantics Construction}

\textit{map symbols in abstract syntax to terms in logic/domain theory}

\vspace{0.5em}
\begin{minipage}[t]{0.4\textwidth}
\parbox[t][1em][t]{\textwidth}{\centering Simple setting}\par
\ifpart{switchtomathexample}{
    \lstinputlisting[language=MMT]{slides/misc/snippets/s000.txt}
}{
    \lstinputlisting[language=MMT]{slides/misc/snippets/s001.txt}
}
\end{minipage}
\hspace{1em}
\begin{minipage}[t]{0.5\textwidth}
\parbox[t][1em][t]{\textwidth}{\centering More advanced}\par
\ifpart{switchtomathexample}{
    \lstinputlisting[language=MMT]{slides/misc/snippets/s002.txt}
}{
    \lstinputlisting[language=MMT]{slides/misc/snippets/s003.txt}
}
\end{minipage}
}
    \end{minipage}
\end{frame}

\begin{frame}[fragile]
    \frametitle{Example: Parsing + Semantics Construction}
    {\centering\str{Ahmed and Berta paint}\par}\vspace{1em}
    \hspace{0.49\textwidth}$\downarrow_{\text{parsing}}$\par\vspace{1em}
    {\centering\color{logicfont!50!nlfont} sentence (andNP ahmed berta) paint\par}\vspace{1em}
    \hspace{0.49\textwidth}$\downarrow_{\text{semantics construction}}$\par\vspace{1em}
    {\color{logicfont}\footnotesize\lstinline[language=MMT]|(\lambdan.\lambdav.n v) ((\lambdaa.\lambdab.\lambdap.a p \wedge b p) (\lambdap.p ahmed) (\lambdap.p berta)) paint|}\par\vspace{1em}
    \hspace{0.49\textwidth}$\downarrow_{\text{$\beta$-reduction}}$\par\vspace{1em}
    \hspace{7.5em}{\color{logicfont}\small\lstinline[language=MMT]|paint ahmed \wedge paint berta|}
\end{frame}

\begin{frame}
    \frametitle{Components of GLIF: ELPI}
    \enablepart{hlelpi}
    \includestandalone[width=\textwidth]{fig/glif-architecture}
\end{frame}

\begin{frame}[fragile]
    \frametitle{Components of GLIF: ELPI}
    \begin{itemize}
        \item Implementation and extension of $\lambda$Prolog\com{$\approx$ Prolog + HOAS}
        \item MMT can generate logic signatures
        \item First experiments with prover generation
        \item Generic inference/reasoning step after semantics construction
    \end{itemize}
    \lstset{basicstyle=\footnotesize\ttfamily}

    \vspace{1em}
    \begin{minipage}[t]{\textwidth}
        \centering
        \begin{minipage}[t]{0.5\textwidth}
            \begin{lstlisting}[language=ELPI,frame=single]
kind o type.
type not o -> o.
type and o -> o -> o.

kind i type.
type forall (i -> o) -> o.
            \end{lstlisting}
        \end{minipage}
    \end{minipage}
\end{frame}

\begin{frame}
    \frametitle{Components of GLIF: Jupyter}
    \enablepart{hljupyter}
    \includestandalone[width=\textwidth]{fig/glif-architecture}
\end{frame}

\begin{frame}
    \frametitle{Components of GLIF: Jupyter}
    \begin{itemize}
        \item Unified, notebook-based interface
        \item Supports implementation and testing
        \item Useful for prototype, demos, teaching, \dots
    \end{itemize}

    \centering
    \vspace{1.5em}
    \fbox{\includegraphics[trim={0 0 20cm 5.7cm},clip,width=0.7\textwidth]{img/screenshot-glif-1.png}}
\end{frame}

}

\begin{frame}
    \frametitle{Levels of inference}
    \centering
    \begin{tikzpicture}[yscale=0.9]
        \draw[line width=1.5pt,rounded corners=.3cm,fill=black!10] (-1.9,-1) rectangle (1.5,3);

        \node[fill=nlbg] (nl) at (0,0) {$\mathcal{NL}$};
        \node[fill=logicbg] (L) at (0,2) {$\mathcal{FL}$};
        \node (M) at (0,4) {$\langle \mathcal{FL}, \mathcal{K}, \models\rangle$};
        \node (Inf) at (6,0) {$\models_{\mathcal{T}}$};
        \node (C) at (6,2) {$\vdash_\mathcal{C}$};
        \node (folg) at (6,4) {$\models$};
        \draw[->] (nl) -- node[left] {\begin{tabular}{c}Sem.\\Constr.\end{tabular}} (L);

        \draw[->] (L) -- (M);
        \draw[dotted,->] (nl) -- node[above] {induces} (Inf);
        \draw[dotted,->] (M) -- node[above] {induces} (folg);
        \draw[->] (L) -- node[above] {calculus} (C);
        \draw[<->] (folg) -- node[left]{$\models \; \equiv \; \vdash_\mathcal{C}$?} (C);
        \draw[<->] (C) -- node[left] {$\models_{\mathcal{T}} \; \equiv \; \vdash_\mathcal{C}$?} (Inf);
    \end{tikzpicture}

    \vspace{1em}
    \begin{enumerate}
        \item Test: Does \str{Ahmed and Berta paint.} $\models_\mathcal{T}$ \str{Berta paints.}?
        \item Model prediction: Yes, because {\color{logicfont}$p(a) \wedge p(b)$} $\vdash_\mathcal{C}$ {\color{logicfont}p(b)}.
        \item Correct result: Ask people.
    \end{enumerate}
%     \vspace{1.2em}
%     \begin{tabular}{r r c l}
%         & \str{Ahmed and Berta paint.} & $\stackrel?{\models_\mathcal{T}}$ & \str{Berta paints.}\\[6pt]
%         & {\color{logicfont}$p(a) \wedge p(b)$} & $\vdash_\mathcal{C}$ & {\color{logicfont}p(b)}
%     \end{tabular}
\end{frame}



\begin{frame}[fragile]
    \frametitle{Natural deduction in MMT: \textit{``Judgments as types''}}

    \begin{minipage}{0.6\textwidth}
        \lstinputlisting[language=MMT]{slides/misc/snippets/s017.txt}
    \end{minipage}
%     \begin{minipage}{0.3\textwidth}
%         \centering
%         \begin{prooftree}
%             \AxiomC{$A \wedge B$}
%             \RightLabel{\lstinline|andEr|}
%             \UnaryInfC{$B$}
%         \end{prooftree}
%     \end{minipage}
\end{frame}

\begin{frame}[fragile]
    \frametitle{$\llbracket\text{\str{Ahmed and Berta paint}}\rrbracket \vdash_{\mathcal{ND}} \llbracket\text{\str{Berta paints}}\rrbracket$}
    \centering
    \begin{tikzpicture}[xscale=1.5]
        \node[draw] (logic) at (0,0) {
            \begin{tabular}l
                Logic\\\hline
                \texttt{o, $\neg$, $\wedge$, \ldots}\\
                \texttt{$\vdash$, conjEr, \ldots}\\
                \texttt{a:$\upiota$, b:$\upiota$, p:$\upiota\to$o, \ldots}
            \end{tabular}
        };
        \node[draw] (s1) at (-3,-3) {
            \begin{tabular}l
                Premise\\\hline
                \texttt{x : $\vdash$ p(a)$\wedge$p(b)}
            \end{tabular}
        };
        \node[draw] (s2) at (3,-3) {
            \begin{tabular}l
                Conclusion\\\hline
                \texttt{y : $\vdash$ p(b)}
            \end{tabular}
        };
        \draw[meta] (logic) -- (s1);
        \draw[meta] (logic) -- (s2);
        \onslide<2->{
            \draw[view] (s2) --node[below=0.2cm,draw]{\texttt{y $\mapsto$ conjEr \textcolor{gray}{p(a) p(b)} x}} (s1);
        }
    \end{tikzpicture}
\end{frame}

\begin{frame}[fragile]
    \frametitle{\large$\llbracket\text{\str{Ahmed paints}}\rrbracket, \llbracket\text{\str{Berta knows Ahmed}}\rrbracket \not\vdash_{\mathcal{ND}} \llbracket\text{\str{Berta knows everyone who paints}}\rrbracket$}
    \centering
    \begin{tikzpicture}[xscale=1.5]
        \node[draw] (logic) at (0,-0.5) {
            \begin{tabular}l
                Logic\\\hline
                \ldots
            \end{tabular}
        };
        \node[draw] (s1) at (-3,-2) {
            \begin{tabular}l
                Premise\\\hline
                \texttt{x : $\vdash$ p(a)}\\
                \texttt{x' : $\vdash$ k(b,a)}
            \end{tabular}
        };
        \node[draw] (s2) at (3,-2) {
            \begin{tabular}l
                Negated Conclusion\\\hline
                \texttt{y : $\vdash$ $\neg$$\forall$e.p(e)$\Rightarrow$k(b,e)}
            \end{tabular}
        };
        \draw[meta] (logic) -- (s1);
        \draw[meta] (logic) -- (s2);
        \onslide<2->{
            \node[draw] (m) at (0, -5) {
                \begin{tabular}l
                    Counter-model\\\hline
                    \texttt{c : $\upiota$}\\
                    \texttt{m1 : $\vdash$ p(a)}\\
                    \texttt{m2 : $\vdash$ p(c)}\\
                    \texttt{m3 : $\vdash$ k(b,a)}\\
                    \texttt{m4 : $\vdash$ $\neg$k(b,c)}
                \end{tabular}
            };
            \draw[view] (s1) --node[left=0.2cm,yshift=-0.5cm,draw]{\begin{tabular}l
                \texttt{x $\mapsto$ m1}\\
                \texttt{x' $\mapsto$ m3}
            \end{tabular}} (m);
            \draw[view] (s2) --node[right=0.2cm,yshift=-0.2cm,draw]{\texttt{y $\mapsto$ \ldots}} (m);
            \draw[meta] (logic) -- (m);
        }
    \end{tikzpicture}
\end{frame}

\begin{frame}
    \frametitle{Mini summary}
    \begin{itemize}
        \item Parsing with GF
        \item Logic syntax in MMT\com{``Bring your own logic''}
        \item Semantics construction in MMT
        \item (Manual) inference in MMT
    \end{itemize}
\end{frame}

\begin{frame}
    \frametitle{Components of GLIF: ELPI}
    \enablepart{hlelpi}
    \autowidth{
    \includestandalone[width=\textwidth]{fig/glif-architecture}
    }
\end{frame}

\begin{frame}[fragile]
    \frametitle{Components of GLIF: ELPI}
    \autowidth{
    \begin{itemize}
        \item Implementation and extension of $\lambda$Prolog\com{$\approx$ Prolog + HOAS}
        \item MMT can generate logic signatures
        \ifpart{mentionprovergen}{
            \item First experiments with prover generation
        }{}
        \item Generic inference/reasoning step after semantics construction
        \ifpart{showspa}{\item Goal: Use it for semantic/pragmatic analysis}{}
    \end{itemize}
    \lstset{basicstyle=\footnotesize\ttfamily}

    \vspace{1em}
    \centering
    % Used in slides/glif-components.tex
% Has to be in separate file...
\begin{minipage}[t]{0.35\textwidth}
    \parbox[t][1em][t]{\textwidth}{\centering\bf MMT}\par
    \begin{lstlisting}[language=MMT,linewidth=\textwidth]
o : type //propositions
\neg : o \raa o
\wedge : o \raa o \raa o
\vee : o \raa o \raa o

\iota : type //individuals
\forall : (\iota \raa o) \raa o
\exists : (\iota \raa o) \raa o \end{lstlisting}
\end{minipage}\hskip2em
\begin{minipage}[t]{0.35\textwidth}
    \parbox[t][1em][t]{\textwidth}{\centering\bf ELPI}\par
    \begin{lstlisting}[language=ELPI,linewidth=\textwidth]
kind o type.
not : o -> o.
and : o -> o -> o.
or  : o -> o -> o.

kind i type.
type forall (i -> o) -> o.
type exists (i -> o) -> o.  \end{lstlisting}
\end{minipage}

    \par
%     \begin{minipage}[t]{\textwidth}
%         \centering
%         \begin{minipage}[t]{0.5\textwidth}
%             \begin{lstlisting}[language=ELPI,frame=single]
% kind o type.
% type not o -> o.
% type and o -> o -> o.
% 
% kind i type.
% type forall (i -> o) -> o.
%             \end{lstlisting}
%         \end{minipage}
%     \end{minipage}
    }
\end{frame}

\disablepart{aselpientro}
\provideenablepart{aselpientro}
\def\ifelpi#1#2{\ifpart{aselpientro}{#1}{#2}}

\begin{frame}
    \ifpart{aselpientro}{
        \frametitle{ELPI}
        \begin{itemize}
            \item Extension of $\lambda$Prolog\com{supports higher-order abstract syntax}
            \item Generic inference/reasoning step after semantics construction
            \item Goal: Use it for semantic/pragmatic analysis
        \end{itemize}
    }{\frametitle{Example: Discard wrong readings in controlled natural language}}

    \begin{minipage}[t][4cm][t]{\textwidth}
        \ifpart{aselpientro}{
            \pause
            \vspace{2em}
            Example: Discard wrong readings in controlled natural language

            \vspace{1em}
        }{}
        \only<\ifelpi2{1-4}>{
            \tikzset{every picture/.style={line width=0.7pt}}
            \begin{tikzpicture}[yscale=0.5]
                \node(str0) at (-4,0) {\str{the ball has a mass of 5kg}};
                \node(ast0) at (-0.5,0) {AST};
                \node(log0) at (4,0) {\ifcolorful\color{logicfont}\fi$\text{mass}(\text{theball}, \text{quant}(5, \text{kilo gram}))$};
                \draw[-{Straight Barb[length=6.3,width=5.0]},gray] (str0) -- (ast0);
                \draw[-{Straight Barb[length=6.3,width=5.0]},gray] (ast0) -- (log0);
            \end{tikzpicture}
            \vspace{3em}
        }
        \only<\ifelpi32>{\disablepart{crossout}}
        \only<\ifelpi4{3-4}>{\enablepart{crossout}}
        \enablepart{switchtonmexample}
        \onslide<\ifelpi{3-4}{2-4}>{
            \includestandalone[width=\textwidth]{fig/cnl-simple-discard} 
        }
    \end{minipage}

    \only<\ifelpi54>{
        \begin{tikzpicture}[overlay,remember picture]
            \fill[gray!80,opacity=0.8] (current page.north west) rectangle (current page.south east);
            \node at (current page.center) { \includegraphics[width=0.95\textwidth]{img/screenshot-glif-3.png} };
        \end{tikzpicture}
    }
\end{frame}


\begin{frame}
    \frametitle{Example: ForTheL}
    \autowidth{
        \includegraphics[width=\textwidth]{img/screenshot-glif-forthel.png}
    }
\end{frame}

\begin{frame}
    \frametitle{Example: ``pairwise disjoint''}
    \newcommand\ds{\text{disjoint}}
    \str{$A$, $B$ and $C$ are pairwise disjoint}\\
    {\color{logicfont} $\ds(A,B) \wedge \ds(A,C) \wedge \ds(B,C)$}
    \vspace{1em}

    \begin{itemize}
        \item \textbf{Approach 1}\\Semantics construction with lots of $\lambda$s:\com{difficult!}
            {\hspace{2em}\color{logicfont} $\ds(A,B) \wedge \ds(A,C) \wedge \top \wedge \ds(B,C) \wedge \top \wedge \top\wedge\top$}\\
                Simplify with ELPI:\\
            {\hspace{2em}\color{logicfont} $\ds(A,B) \wedge \ds(A,C) \wedge \ds(B,C)$}
        \pause
        \item \textbf{Approach 2}\\Semantics construction creates preliminary expression:\\
            {\hspace{2em}\color{logicfont}\ttfamily relNT disjoint (cons A (cons B (cons C nil))) }\\
                Convert with ELPI:\com{easier}
            {\hspace{2em}\color{logicfont} $\ds(A,B) \wedge \ds(A,C) \wedge \ds(B,C)$}
    \end{itemize}
\end{frame}

\providedisablepart{showscreenshot}
\begin{frame}[fragile]
    \def\mybox#1{\square_{#1}}
    \def\mydia#1{\lozenge_{#1}}
    \def\sfiven{{S5_n}}
    \frametitle{Example: Epistemic Q\&A}
    \centering
    \strplain{\makebox[9.5cm][l]{John knows that Mary or Eve knows that Ping has a dog.}\makebox[1.5em]{\upshape($S_1$)}\\
              \makebox[9.5cm][l]{Mary doesn't know if Ping has a dog.}\makebox[1.5em]{\upshape($S_2$)}\\
              \makebox[9.5cm][l]{Does Eve know if Ping has a dog?}\makebox[1.5em]{\upshape($Q$)}}

    {\color{logicfont}
        \begin{align*}
            S_1 &= \mybox{john}(\mybox{mary} hd(ping)\vee \mybox{eve}hd(ping))\\
            S_2 &= \neg(\mybox{mary}hd(ping) \vee \mybox{mary}\neg hd(ping))\\
            Q &= \mybox{eve}hd(ping) \vee \mybox{eve}\neg hd(ping)
        \end{align*}
    }

    \begin{table}
        \begin{tabular}{l l}
            $S_1, S_2 \vdash_\sfiven Q$\quad      &$\leadsto$\quad yes\\
            $S_1, S_2 \vdash_\sfiven \neg Q$\quad &$\leadsto$\quad no\\
            else &$\leadsto$\quad maybe
        \end{tabular}
    \end{table}
    \ifpart{showscreenshot}{\only<2>{
        \begin{tikzpicture}[overlay,remember picture]
            \fill[gray!80,opacity=0.8] (current page.north west) rectangle (current page.south east);
            \node at (current page.center) { \includegraphics[width=0.9\textwidth]{img/screenshot-glif-4.png} };
        \end{tikzpicture}
    }}{}
\end{frame}



\begin{frame}
    \frametitle{Conclusion}
    \begin{minipage}[t]{0.5\textwidth}
        \textbf{Summary:}
        \begin{itemize}
            \item GLIF = GF + MMT + ELPI
            \item Prototyping natural language understanding %\com{(symbolic)}
            \item We use it for teaching
        \end{itemize}
    \end{minipage}
    \begin{minipage}[t]{0.49\textwidth}
        \textbf{Examples:}
        \begin{enumerate}
            \item \str{What is the cardinality of G?}
            \item \str{a kinetic energy of 12mN}
            \item \str{$A$, $B$ and $C$ are pairwise disjoint}
            % \item \str{He loves her} (tableaux machine)
            \item \str{John knows that Eve has a dog}
            % \item \str{Ahmed's cousin}
        \end{enumerate}
    \end{minipage}
    
    \vspace{1em}
    \centering
    \includestandalone[width=0.8\textwidth]{fig/glif-architecture}
\end{frame}

\appendix


\begin{frame}
    \frametitle{Pipeline Specification}
    \includestandalone[width=\textwidth]{fig/glif-spec}
\end{frame}


\begin{frame}[fragile]
    \frametitle{Example: Tableaux Machine~\cite{KohKol:ramgpm03}}
    \begin{itemize}
        \item Can use tableaux for model generation
        \item Tableau machine: pick ``best'' branch as model and continue there with next sentence
            \com{like a human?}
    \end{itemize}

    \vspace{1.5em}
    \begin{minipage}[t][4.5cm]{\textwidth}
        \only<1>{\setshowlevel{1}}
        \only<2>{\setshowlevel{2}}
        \only<3>{\setshowlevel{3}}
        % \only<4>{\setshowlevel{4}}
        \includestandalone{fig/tab-machine-simple}
    \end{minipage}
\end{frame}

\begin{frame}
    \frametitle{Example: Tableaux Machine}
    \only<1>{\setshowlevel{1}}
    \only<2>{\setshowlevel{2}}
    \only<3>{\setshowlevel{3}}
    \only<4>{\setshowlevel{4}}
    \makebox[\linewidth]{\includestandalone[scale=0.9]{fig/tab-machine-complex}}
\end{frame}


\begin{frame}[fragile]
    \frametitle{Example: Translation}
    \begin{itemize}
        \item Two German words for \str{cousin}, depending on the gender
        \item Two entries in abstract syntax: \verb|cousin_female| and \verb|cousin_male|
        \item Use inference to discard ASTs
    \end{itemize}
    
    \vspace{2em}
    \small
    \begin{minipage}[t][5cm]{\textwidth}
        \begin{tikzpicture}
            \node(eng) at (-4,1) {\parbox{4.2cm}{\str{Kim is Ahmed's cousin and the father of Grace}}};
                \node(ger1) at (-4,-0.5) {\parbox{4.2cm}{\str{Kim ist Ahmeds {\upshape\bf Cousine} und Graces Vater}}};
                \node(ger2) at (-4,-2.0) {\parbox{4.2cm}{\str{Kim ist Ahmeds {\upshape\bf Cousin} und Graces Vater}}};
            \node(ast1) at (0,1) {AST$_1$};
            \node(ast2) at (0,-1) {AST$_2$};
            \only<2->{
                \node(log1) at (3,1) {\color{logicfont} \parbox{2.2cm}{$female(kim) \wedge$ $male(kim)$}};
                \node(log2) at (3,-1) {\color{logicfont} \parbox{2.2cm}{$male(kim) \wedge$ $male(kim)$}};
            }
            \draw[->,thick] (eng) -- (ast1);
            \draw[->,thick] (eng) -- (ast2);
                \draw[->,thick] (ast1) -- (ger1);
                \draw[->,thick] (ast2) -- (ger2);
            \only<2->{
                \draw[->,thick] (ast1) -- (log1);
                \draw[->,thick] (ast2) -- (log2);
            }
            \only<3>{
                \draw[ultra thick,red] (2,1.5) -- (4,0.5);
                \draw[ultra thick,red] (2,0.5) -- (4,1.5);
                
                \draw[ultra thick,red] (-0.5,1.5) -- (0.5,0.5);
                \draw[ultra thick,red] (-0.5,0.5) -- (0.5,1.5);

                \draw[ultra thick,red] (-6,0.0) -- (-2,-1.0);
                \draw[ultra thick,red] (-6,-1.0) -- (-2,0.0);
            }
        \end{tikzpicture}
    \end{minipage}
\end{frame}


\bgroup

% \providecolorgroup{inf}{blue!50!red}
\providecolorgroup{inf}{black}

% used for highlighting parts of the code.
% probably better solutions exist...
\lstset{literate={*1}{\color{hlfont}}{1} {*2}{\color{inffont}}{1} {*3}{\color{inffont!50}}{1}}

\def\proofvdots#1{
    \let\tmpvskip=\extraVskip
    \def\extraVskip{-2pt}
    \noLine
    \UnaryInfC{{$\raisebox{6pt}\vdots$}}
    \noLine
    #1
    \let\extraVskip=\tmpvskip
}
\newcommand\tabdivider{\;\bigl|\;}

\begin{frame}[fragile]
    \frametitle{Natural Deduction in MMT/LF}
    \begin{minipage}{0.9\textwidth}
        \centering
        \begin{minipage}{0.49\textwidth}
            \begin{prooftree}
                \AxiomC{$A \wedge B$}
                \RightLabel{$\wedge El$}
                \UnaryInfC{$A$}
            \end{prooftree}
        \end{minipage}
        \begin{minipage}{0.49\textwidth}
            \begin{prooftree}
                \def\defaultHypSeparation{\hskip 0pt}
                \AxiomC{$A \vee B$}
                \AxiomC{$\,\,[A]^1$}
                \proofvdots{\UnaryInfC{$C$}}
                \AxiomC{$\,\,[B]^1$}
                \proofvdots{\UnaryInfC{$C$}}
                \RightLabel{$\vee E^1$}
                \TrinaryInfC{$C$}
            \end{prooftree}
        \end{minipage}
    \end{minipage}

    \vspace{1.5em}
    \begin{lstlisting}[language=MMT]
    // \vdashX is type of proofs for X (judgments as types)
    \vdash : o \raa type

    \wedgeEl : \PiAo\PiBo \vdashA\wedgeB \raa \vdashA
    \veeE  : \PiAo\PiBo\PiCo \vdashA\veeB \raa (\vdashA \raa \vdashC) \raa (\vdashB \raa \vdashC) \raa \vdashC
    \end{lstlisting}
\end{frame}

\begin{frame}[fragile]
    \frametitle{Generating Provers in ELPI}
%     \begin{itemize}
%         \item ELPI is an extension of $\lambda$Prolog \com{$\approx$ Prolog + HOAS}\note{ELPI was developed by Claudio and others}
%         \item Optimized for fast execution of logical algorithms \com{type inference, unification, proof search, \dots}
%     \end{itemize}
    \makebox[2.5cm][l]{\textbf{LF rule}} \lstinline[language=MMT]|\wedgeEl : \PiAo\PiBo \vdashA\wedgeB \raa \vdashA|

    \vspace{1.0em}
    \textbf{ELPI equivalent}

    \vspace{0.5em}
    \makebox[2.5cm][r]{direct:$\;\;$} \lstinline[language=ELPI]|pi A \ pi B \ ded (and A B) => ded A.|

    \vspace{0.5em}
    \makebox[2.5cm][r]{syn. sugar:$\;\;$} \lstinline[language=ELPI]|ded A :- ded (and A B).|

% \end{frame}
% 
% \begin{frame}[fragile]
%     \frametitle{From LF to ELPI}
    % \textbf{Or-elimination}
    \vspace{1.5em}
    \pause

    \begin{block}{{\bfseries Example:} Or-Elimination}
    \makebox[1.2cm][l]{LF:}\begin{minipage}{0.85\textwidth}
        \lstinline[keepspaces=true,language=MMT]|\veeE : \PiAo\PiBo\PiCo \vdashA\veeB \raa (\vdashA \raa \vdashC) \raa (\vdashB \raa \vdashC) \raa \vdashC|
    \end{minipage}

    \vspace{0.5em}
    \makebox[1.2cm][l]{ELPI:}\lstinline[language=ELPI,keepspaces=true]|ded C :- ded (or A B), ded A => ded C, ded B => ded C.|
    \end{block}

    \vspace{0.5em}

    \begin{block}{{\bfseries Example:} Forall-Introduction}
    % \textbf{Forall-introduction}
    \makebox[1.2cm][l]{LF:}\begin{minipage}{0.85\textwidth}
        \lstinline[keepspaces=true,language=MMT]|\forallI : \PiPio (\Pixi \vdashP x) \raa \vdash\forallP|
    \end{minipage}

    \vspace{0.5em}
    \makebox[1.2cm][l]{ELPI:}\lstinline[language=ELPI,keepspaces=true]|ded (forall P) :- pi x \ ded (P x).|
    \vspace{0.5em}
    \end{block}
\end{frame}

\begin{frame}[fragile]
    \frametitle{Controlling the Proof Search}
    \begin{itemize}
        \item Problem: Search diverges \com{searching harder than checking}
        \item Solution: Control search with helper predicates:
            \com{inspired by ProofCert project by Miller et al.}\note{ProofCert assumes a focused logic, we don't}
            % \\{ \itshape\color{black!50}\makebox[10cm][r]{(inspired by ProofCert project by Miller et al.)}}
            \begin{itemize}
                \item Intuition: Decide whether to apply rule
                \item Do not affect correctness
                \item Extra argument tracks aspects of proof state
            \end{itemize}
    \end{itemize}

    \vspace{1.5em}
    \makebox[1.2cm][l]{Before:}{{
    \lstinline[language=ELPI,keepspaces=true]|ded*1 *2A :-*1                      *2ded *1   *2(and A B).|
    }}

    \vspace{0.5em}
    \makebox[1.2cm][l]{Now:}{{
    \lstinline[language=ELPI,keepspaces=true]|ded*1X*2A :- *1help/andEl X A B X1, *2ded *1X1 *2(and A B).|
    }}

%     \vspace{2.0em}
%     Example helper for depth-limit:
% 
%     \vspace{0.5em}
%     \lstinline[language=ELPI,keepspaces=true]|    help/andEl (idcert N) _ _ (idcert N1) :- N > 0, N1 is N - 1.|
\end{frame}

\begin{frame}[fragile]
    \frametitle{Helper Predicates}
        \renewcommand{\arraystretch}{1.5}
    \begin{tabular}{l p{4cm} p{4.5cm}}
        \textbf{Name} & \textbf{Predicate} & \textbf{Argument} \\
        Iter. deepening & checks depth & remaining depth \\
        Proof term & generates term & proof term \\
        Product & calls other predicates & arguments for other predicates \\
        Backchaining &
            \footnotesize Prolog's backchaining ($\approx$ forward reasoning from axioms via $\Rightarrow$/$\forall$ elimination rules) &
            \footnotesize pattern of formula to be proven (e.g. a conjunction) \\
        % Backchaining & \multicolumn2{p{7cm}}{\footnotesize Restricts new formulae in e.g. elimination rules to those that could be proven by forward reasoning} \\
    \end{tabular}

    \vspace{1.5em}
    \begin{block}{\textbf{Example helper:} Iterative deepening}
        \lstinline[language=ELPI,keepspaces=true]|help/andEl (idcert N) _ _ (idcert N1) :- N > 0, N1 is N - 1.|
    \end{block}

%     Example call:
%     \begin{lstlisting}[language=ELPI]
%     ?- ded (prodcert (idcert 2) (ptcert Proof)) (impl a (or a b)).
% 
%     Proof = implI a (or a b) (orIl a b (i a)).
%     \end{lstlisting}
\end{frame}

\begin{frame}[fragile]
    \frametitle{Tableau Provers}
    \note{We can scale in terms of logics supported or (orthogonally) in terms of prover strategies.
    We went for the latter.}
    \begin{minipage}[b][2cm][b]{0.4\textwidth}
        \begin{prooftree}
            \AxiomC{$\;\tabfalse{A \wedge B}$}
            \RightLabel{$\tabfalse\wedge$}
            \UnaryInfC{$\tabfalse{A} \tabdivider \tabfalse{B}$}
        \end{prooftree}
    \end{minipage}
    \begin{minipage}[b][2cm][b]{0.4\textwidth}
        \def\defaultHypSeparation{\hskip 0pt}
        \begin{prooftree}
            \AxiomC{$\tabfalse{A \wedge B}$}
            \AxiomC{$\;[\tabfalse{A}]$}
            \proofvdots{\UnaryInfC{$\bot$}}
            \AxiomC{$\;[\tabfalse{B}]$}
            \proofvdots{\UnaryInfC{$\bot$}}
            \RightLabel{$\tabfalse\wedge$}
            \TrinaryInfC{$\bot$}
        \end{prooftree}
    \end{minipage}

    \vspace{2em}
    \makebox[1.0cm][l]{LF:} \lstinline[language=MMT]|\wedge\tabfalse : \PiAo\PiBo A\wedgeB\tabfalse \raa (A\tabfalse \raa \bot) \raa (B\tabfalse \raa \bot) \raa \bot|

    \vspace{0.5em}
    \makebox[1.0cm][l]{ELPI:} \lstinline[language=ELPI]|closed *3X *2:- *3help/andF X A B X1 X2 X3, *2f *3X1 *2(and A B),|
    \lstinline[language=ELPI,keepspaces=true]|                         f*3/hyp *2A => closed *3X2*2, f*3/hyp*2 B => closed *3X3*2.|

    \vspace{2em}
    With iterative deepening we get a working prover!

    $\rightarrow$ Other helpers result in more efficient provers
\end{frame}

\egroup


\begin{frame}[allowframebreaks,t]
    \frametitle{References}
    \printbibliography
\end{frame}

\end{document}
