\bgroup

% \providecolorgroup{inf}{blue!50!red}
\providecolorgroup{inf}{black}

% used for highlighting parts of the code.
% probably better solutions exist...
\lstset{literate={*1}{\color{hlfont}}{1} {*2}{\color{inffont}}{1} {*3}{\color{inffont!50}}{1}}

\def\proofvdots#1{
    \let\tmpvskip=\extraVskip
    \def\extraVskip{-2pt}
    \noLine
    \UnaryInfC{{$\raisebox{6pt}\vdots$}}
    \noLine
    #1
    \let\extraVskip=\tmpvskip
}
\newcommand\tabdivider{\;\bigl|\;}

\begin{frame}[fragile]
    \frametitle{Natural Deduction in MMT/LF}
    \begin{minipage}{0.9\textwidth}
        \centering
        \begin{minipage}{0.49\textwidth}
            \begin{prooftree}
                \AxiomC{$A \wedge B$}
                \RightLabel{$\wedge El$}
                \UnaryInfC{$A$}
            \end{prooftree}
        \end{minipage}
        \begin{minipage}{0.49\textwidth}
            \begin{prooftree}
                \def\defaultHypSeparation{\hskip 0pt}
                \AxiomC{$A \vee B$}
                \AxiomC{$\,\,[A]^1$}
                \proofvdots{\UnaryInfC{$C$}}
                \AxiomC{$\,\,[B]^1$}
                \proofvdots{\UnaryInfC{$C$}}
                \RightLabel{$\vee E^1$}
                \TrinaryInfC{$C$}
            \end{prooftree}
        \end{minipage}
    \end{minipage}

    \vspace{1.5em}
    \begin{lstlisting}[language=MMT,keepspaces=true]
    // \vdashX is type of proofs for X (judgments as types)
    \vdash : o \raa type

    \wedgeEl : \PiAo \PiBo  \vdashA\wedgeB \raa \vdashA
    \veeE  : \PiAo \PiBo \PiCo  \vdashA\veeB \raa (\vdashA \raa \vdashC) \raa (\vdashB \raa \vdashC) \raa \vdashC
    \end{lstlisting}
\end{frame}

\begin{frame}[fragile]
    \frametitle{Generating Provers in ELPI}
%     \begin{itemize}
%         \item ELPI is an extension of $\lambda$Prolog \com{$\approx$ Prolog + HOAS}\note{ELPI was developed by Claudio and others}
%         \item Optimized for fast execution of logical algorithms \com{type inference, unification, proof search, \dots}
%     \end{itemize}
    \makebox[2.5cm][l]{\textbf{LF rule}} \lstinline[language=MMT]|\wedgeEl : \PiAo\PiBo \vdashA\wedgeB \raa \vdashA|

    \vspace{1.0em}
    \textbf{ELPI equivalent}

    \vspace{0.5em}
    \makebox[2.5cm][r]{direct:$\;\;$} \lstinline[language=ELPI]|pi A \ pi B \ ded (and A B) => ded A.|

    \vspace{0.5em}
    \makebox[2.5cm][r]{syn. sugar:$\;\;$} \lstinline[language=ELPI]|ded A :- ded (and A B).|

% \end{frame}
% 
% \begin{frame}[fragile]
%     \frametitle{From LF to ELPI}
    % \textbf{Or-elimination}
    \vspace{1.5em}
    \pause

    \begin{block}{{\bfseries Example:} Or-Elimination}
    \makebox[1.2cm][l]{LF:}\begin{minipage}{0.85\textwidth}
    \lstinline[keepspaces=true,language=MMT]|\veeE : \PiAo\PiBo\PiCo \vdashA\veeB \raa (\vdashA \raa \vdashC) \raa (\vdashB \raa \vdashC) \raa \vdashC|
\end{minipage}

    \vspace{0.5em}
    \makebox[1.2cm][l]{ELPI:}\lstinline[language=ELPI,keepspaces=true]|ded C :- ded (or A B), ded A => ded C, ded B => ded C.|
    \end{block}

    \vspace{0.5em}

    \begin{block}{{\bfseries Example:} Forall-Introduction}
    % \textbf{Forall-introduction}
    \makebox[1.2cm][l]{LF:}\begin{minipage}{0.85\textwidth}
        \lstinline[keepspaces=true,language=MMT]|\forallI : \PiPio (\Pixi \vdashP x) \raa \vdash\forallP|
    \end{minipage}

    \vspace{0.5em}
    \makebox[1.2cm][l]{ELPI:}\lstinline[language=ELPI,keepspaces=true]|ded (forall P) :- pi x \ ded (P x).|
    \vspace{0.5em}
    \end{block}
\end{frame}

\begin{frame}[fragile]
    \frametitle{Controlling the Proof Search}
    \begin{itemize}
        \item Problem: Search diverges \com{searching harder than checking}
        \item Solution: Control search with helper predicates:
            \com{inspired by ProofCert project by Miller et al.}\note{ProofCert assumes a focused logic, we don't}
            % \\{ \itshape\color{black!50}\makebox[10cm][r]{(inspired by ProofCert project by Miller et al.)}}
            \begin{itemize}
                \item Intuition: Decide whether to apply rule
                \item Do not affect correctness
                \item Extra argument tracks aspects of proof state
            \end{itemize}
    \end{itemize}

    \vspace{1.5em}
    \makebox[1.2cm][l]{Before:}{{
    \lstinline[language=ELPI,keepspaces=true]|ded*1 *2A :-*1                      *2ded *1   *2(and A B).|
    }}

    \vspace{0.5em}
    \makebox[1.2cm][l]{Now:}{{
    \lstinline[language=ELPI,keepspaces=true]|ded*1X*2A :- *1help/andEl X A B X1, *2ded *1X1 *2(and A B).|
    }}

%     \vspace{2.0em}
%     Example helper for depth-limit:
% 
%     \vspace{0.5em}
%     \lstinline[language=ELPI,keepspaces=true]|    help/andEl (idcert N) _ _ (idcert N1) :- N > 0, N1 is N - 1.|
\end{frame}

\begin{frame}[fragile]
    \frametitle{Helper Predicates}
        \renewcommand{\arraystretch}{1.5}
    \begin{tabular}{l p{4cm} p{4.5cm}}
        \textbf{Name} & \textbf{Predicate} & \textbf{Argument} \\
        Iter. deepening & checks depth & remaining depth \\
        Proof term & generates term & proof term \\
        Product & calls other predicates & arguments for other predicates \\
        Backchaining &
            \footnotesize Prolog's backchaining ($\approx$ forward reasoning from axioms via $\Rightarrow$/$\forall$ elimination rules) &
            \footnotesize pattern of formula to be proven (e.g. a conjunction) \\
        % Backchaining & \multicolumn2{p{7cm}}{\footnotesize Restricts new formulae in e.g. elimination rules to those that could be proven by forward reasoning} \\
    \end{tabular}

    \vspace{1.5em}
    \begin{block}{\textbf{Example helper:} Iterative deepening}
        \lstinline[language=ELPI,keepspaces=true]|help/andEl (idcert N) _ _ (idcert N1) :- N > 0, N1 is N - 1.|
    \end{block}

%     Example call:
%     \begin{lstlisting}[language=ELPI]
%     ?- ded (prodcert (idcert 2) (ptcert Proof)) (impl a (or a b)).
% 
%     Proof = implI a (or a b) (orIl a b (i a)).
%     \end{lstlisting}
\end{frame}

\begin{frame}[fragile]
    \frametitle{Tableau Provers}
    \note{We can scale in terms of logics supported or (orthogonally) in terms of prover strategies.
    We went for the latter.}
    \begin{minipage}[b][2cm][b]{0.4\textwidth}
        \begin{prooftree}
            \AxiomC{$\;\tabfalse{A \wedge B}$}
            \RightLabel{$\tabfalse\wedge$}
            \UnaryInfC{$\tabfalse{A} \tabdivider \tabfalse{B}$}
        \end{prooftree}
    \end{minipage}
    \begin{minipage}[b][2cm][b]{0.4\textwidth}
        \def\defaultHypSeparation{\hskip 0pt}
        \begin{prooftree}
            \AxiomC{$\tabfalse{A \wedge B}$}
            \AxiomC{$\;[\tabfalse{A}]$}
            \proofvdots{\UnaryInfC{$\bot$}}
            \AxiomC{$\;[\tabfalse{B}]$}
            \proofvdots{\UnaryInfC{$\bot$}}
            \RightLabel{$\tabfalse\wedge$}
            \TrinaryInfC{$\bot$}
        \end{prooftree}
    \end{minipage}

    \vspace{2em}
    \makebox[1.0cm][l]{LF:} \lstinline[language=MMT]|\wedge\tabfalse : \PiAo\PiBo A\wedgeB\tabfalse \raa (A\tabfalse \raa \bot) \raa (B\tabfalse \raa \bot) \raa \bot|

    \vspace{0.5em}
    \makebox[1.0cm][l]{ELPI:} \lstinline[language=ELPI]|closed *3X *2:- *3help/andF X A B X1 X2 X3, *2f *3X1 *2(and A B),|
    \lstinline[language=ELPI,keepspaces=true]|                         f*3/hyp *2A => closed *3X2*2, f*3/hyp*2 B => closed *3X3*2.|

    \vspace{2em}
    With iterative deepening we get a working prover!

    $\rightarrow$ Other helpers result in more efficient provers
\end{frame}

\egroup
