\documentclass[aspectratio=169]{beamer}
\usepackage{standalone}

\usepackage{stmaryrd}
\usepackage{listings}
\usepackage{bussproofs}

\usepackage[hyperref=auto,style=alphabetic,backend=bibtex]{biblatex}
\addbibresource{kwarcpubs.bib}
\addbibresource{extpubs.bib}
\addbibresource{extcrossrefs.bib}
\usepackage{appendixnumberbeamer}
\usepackage{tikz}
\usepackage{tikz-qtree}
\usetikzlibrary{arrows.meta}
\usetikzlibrary{mmt}
\usetikzlibrary{docicon}

\usetheme{Pittsburgh}
\setbeamertemplate{footline}[frame number]
\setbeamertemplate{navigation symbols}{}
\usecolortheme{beaver}
\setbeamertemplate{frametitle}[default][left]
% \setbeamersize{text margin left=3em}

\usepackage{utils/colors}
\usepackage[forbeamer]{utils/basic}
\usepackage{utils/operators}
\usepackage{utils/mylstmisc}
\usepackage{utils/lstmmt}

\lstset{basicstyle=\ttfamily}
\lstset{commentstyle=\itshape\color{commentfont}}

\title{\textit{What does it mean?}\\A framework for prototyping Montague-style semantics}

\author{Jan Frederik Schaefer}
\institute{FAU Erlangen-N\"urnberg/KWARC}
\date{\textbf{Seminar: Computing Meaning}\\Hildesheim\\July 21, 2022}

\begin{document}
\begin{frame}[fragile, noframenumbering]
    \textbf{What does it mean? A framework for prototyping Montague-style semantics}
    \small

    \begin{block}{}
        \parbox{\linewidth}{
            One approach to study the meaning of natural language is to translate
            utterances to logical expressions, which have well-defined semantics and allow
            for rigorous inference.  Richard Montague, a well-known pioneer of this
            approach, described the translation of a particular subset of English into
            logic in great detail.  Over the years, there have been many similar
            publications that propose a novel way to translate sentences into a logic
            (often first-order logic, but more obscure logics are also common).  Sometimes,
            the approach was extended with pragmatic analysis based on logical inference.

            We claim that research (and education) in this direction could benefit from
            more prototyping to test and demonstrate new ideas.

            In this talk, I will present GLIF, a declarative framework for prototyping the
            translation of natural language to logics.  GLIF combines existing, specialized
            frameworks that solve part of the problem: the Grammatical Framework
            (development of natural language grammars), MMT (logic development) and ELPI
            (inference).  These frameworks can be connected seamlessly because of their
            compatible underlying logical frameworks.  We successfully use GLIF in a
            lecture on symbolic natural language semantics.
        }
    \end{block}
\end{frame}

\frame\titlepage

\begin{frame}
    \frametitle{whoami}
    \begin{itemize}
        \item PhD student in the KWARC group (Erlangen)
            \com{Knowledge representation}
        \item Supervisor: Michael Kohlhase
        \item My interest: Mathematical language, precise semantics extraction
        \item Background for this talk:
            \begin{itemize}
                \item We teach a lecture in \emph{logic-based natural language semantics}
                \item Wanted more hands-on experience
                \item $\leadsto$ new framework: GLIF\com{Grammar, Logic, Inference}
            \end{itemize}
    \end{itemize}
\end{frame}

% \begin{frame}
    \frametitle{Natural Language Understanding (NLU)}
    \textbf{My definition:}\\
    NLU means translating natural language into a formal semantic representation.

    \vspace{3em}
    \makebox[9cm]{\str{How many people live in Slovakia?}} $\leadsto\;\;$ \str{5.458 million}

    \vspace{1.5em}
    \makebox[9cm]{\str{Do more people live in Slovakia than in Thailand?}} $\leadsto\;\;$ ???


%     \vspace{2em}
%     \centering
%     \begin{tikzpicture}
%         \node (nbnd) at (1.2,0.7) {useless};
%         \node (bnd) at (3.5,0.7) {\begin{tabular}{c}also in-\\teresting\end{tabular}};
%         \node (bd) at (3.5,2.0) {AI-complete?};
%         \node (nbd) at (1.2,2.0) {\begin{tabular}{c}this\\work\end{tabular}};
%         \draw[->,very thick] (-0.3,0) -- (5,0);
%         \draw[->,very thick] (0,-0.3) -- (0,2.7);
%         \node at (4.0, -0.3) {\itshape breadth};
%         \node[rotate=90] at (-0.3,1.8) {\itshape depth};
%     \end{tikzpicture}
\end{frame}



\begin{frame}
    \frametitle{Method of Fragments}
    \only<1-2>{
        \centering
        \only<1>{\def\fraglevel{0}}
        \only<2>{\def\fraglevel{1}}
        \includestandalone{fig/montague-fragments}

        \begin{minipage}[t][2cm]{\textwidth}\vspace{1em}
            How do we get from messy language to formal logic?\\[0.5em]
            \emph{Montague}~\cite{Montague:efl70}: Look at a ``nice'' subset
            and map into logic.
        \end{minipage}
    }

    \only<3>{
        \centering
        \def\fraglevel{1}
        \includestandalone{fig/montague-fragments}
        
        \begin{minipage}[t][2cm]{0.6\textwidth}\vspace{1em}
            \str{Ahmed paints and Berta is quiet.}\\[0.5em]
            \str{Ahmed doesn't paint.}
        \end{minipage}\hfill
        \begin{minipage}[t][2cm]{0.39\textwidth}\vspace{1em}
            $p(a) \wedge q(b)$\\[0.5em]
            $\neg p(a)$
        \end{minipage}
    }

    \only<4>{
        \centering
        \def\fraglevel{2}
        \includestandalone{fig/montague-fragments}
        
        \begin{minipage}[t][2cm]{0.6\textwidth}\vspace{1em}
            \str{Every student paints and is quiet.}\\[0.5em]
            \str{Nobody paints.}
        \end{minipage}\hfill
        \begin{minipage}[t][2cm]{0.39\textwidth}\vspace{1em}
            $\forall x.s(x) \Rightarrow (p(x) \wedge q(x))$\\[0.5em]
            $\neg \exists x.p(x)$
        \end{minipage}
    }

    \only<5>{
        \centering
        \def\fraglevel{3}
        \includestandalone{fig/montague-fragments}

        \begin{minipage}[t][2cm]{0.6\textwidth}\vspace{1em}
            \str{Ahmed isn't allowed to paint.}\\[0.5em]
            \str{Ahmed and Berta must paint.}
        \end{minipage}\hfill
        \begin{minipage}[t][2cm]{0.39\textwidth}\vspace{1em}
            $\neg\lozenge p(a)$\\[0.5em]
            $(\square p(a)) \wedge \square p(b)$
        \end{minipage}
    }
\end{frame}


\begin{frame}
    \frametitle{Method of Fragments}
    {\color{hlfont}Hand-waving} is problematic:

    \hspace{2em}\str{Ahmed paints. He is quiet.}
    {$\quad\stackrel{?}{\leadsto}\quad$ \color{logicfont} $p(a)\wedge q(a)$}

    \vspace{1.2em}
    {\color{hlfont}Montague}: Specify
    \begin{itemize}
        \item grammar,\com{fixes NL subset}
        \item target logic,
        \item semantics construction.\com{maps parse trees to logic}
    \end{itemize}

    {
        \centering
        \vspace{0.3em}
        {\itshape\footnotesize Example from~\cite{Montague:tptoqi73}}

        \vspace{0.2em}\fbox{\includegraphics[trim=0 0 0 80,clip,width=0.7\textwidth]{fig/montague-tptoqioe.png}}

        \vspace{1.2em}
        Claim: That doesn't scale well $\leadsto$ \textbf{We need {\color{hlfont}prototyping}!}
    }

    % \newcommand\VP{\text{\upshape\tiny VP}}
    % \hspace{2em}$\llbracket\text{\strplain{$P_{\VP}$ and $Q_{\VP}$}}\rrbracket_{\VP} = \lambda x. \llbracket\text{\strplain{$P_{\VP}$}}\rrbracket(x) \wedge \llbracket\text{\strplain{$Q_{\VP}$}}\rrbracket(x)$
\end{frame}


\begin{frame}
    \frametitle{NLU Prototyping}
    \begin{itemize}
        \item Traditionally done in Prolog/Haskell
        \begin{itemize}
            \item[$\raa$] requires a lot of work
        \end{itemize}
            \item A dedicated framework might be better
        \begin{itemize}
            \item[$\raa$] only partial solutions exist
        \end{itemize}
        \item Can we combine existing partial solutions?\com{Research Question}
        \begin{itemize}
            \item[$\leadsto$] GLIF
        \end{itemize}
    \end{itemize}
\end{frame}



\setlength\pdfpagewidth{10cm}%
\setlength\pdfpageheight{9cm}%
\setlength\textwidth{8cm}
% \newgeometry{left=0.1cm,bottom=0.1cm}

{
    \enablepart{showjupyter}
    \disablepart{showepistemicexample}
    \disablepart{mentionprovergen}
    \begin{frame}
    \frametitle{Components of GLIF: GF}
    \enablepart{hlgf}
    \includestandalone[width=\textwidth]{fig/glif-architecture}
\end{frame}

\begin{frame}[fragile]
    \frametitle{Components of GLIF: Grammatical Framework \cite{GF:on}}
    \begin{itemize}
        \item Specialized for developing natural language grammars
        \item Separates abstract and concrete syntax\par
            \quad\lstinline[language=GF]|sentence : NP -> VP -> S;               --abstract|\par
            \quad\lstinline[language=GF]|sentence np vp = np.s ++ vp.s!np.n;   --concrete|
        \item Abstract syntax based on type theory
        \item Comes with large library \com{$\ge 36$ languages}
    \end{itemize}

    \vspace{2em}
    \hspace{2.5em}\begin{tikzpicture}
        \node(eng) at (-2,0.7) {\str{Ahmed paints}};
        \node(ger) at (-2,-0.7) {\str{Ahmed zeichnet}};
        \node(ast) at (4,0) {
                    \color{logicfont!50!nlfont}
                    \resizebox{2.cm}{!}{\tikzset{edge from parent/.append style={very thick}}
                        \Tree [ .\texttt{sentence} \texttt{ahmed} \texttt{paint} ]
                    }
                };
        \draw[<->, thick] (eng) to node[above,rotate=-9]{Eng. concr. syn.} (ast);
        \draw[<->, thick] (ger) to node[below,rotate=9]{Ger. concr. syn.} (ast);
    \end{tikzpicture}
\end{frame}

\begin{frame}
    \frametitle{Components of GLIF: MMT}
    \enablepart{hlmmt}
    \includestandalone[width=\textwidth]{fig/glif-architecture}
\end{frame}

\begin{frame}[fragile]
    \frametitle{Components of GLIF: MMT}
    \lstset{frame=single}
    \begin{itemize}   
        \item Modular logic development and knowledge representation
        \item Not specialized in one logical framework \com{we use LF}
        \item We will use MMT to:
        \begin{itemize}
            \item { \only<2>{\bf\color{hlfont}} represent abstract syntax }
            \item { \only<3>{\bf\color{hlfont}} specify target logic and domain theory}
            \item { \only<4>{\bf\color{hlfont}} specify semantics construction}
        \end{itemize}
    \end{itemize}
    \lstset{basicstyle=\footnotesize\ttfamily}
    
    \vspace{1em}
    \begin{minipage}[t][4cm]{\textwidth}
        \centering
        \only<2>{% Used in slides/glif-components.tex
% Has to be in separate file...
\begin{minipage}[t]{0.4\textwidth}
    \parbox[t][1em][t]{\textwidth}{\centering\bf GF}\par
    \begin{lstlisting}[language=GF,linewidth=\textwidth]
cat
  NP; VP; S;
fun
  make_S :
    NP -> VP -> S;
    \end{lstlisting}
\end{minipage}
\begin{minipage}[t]{0.1\textwidth}\vskip3em \centering\LARGE$\mapsto$\end{minipage}
\begin{minipage}[t]{0.35\textwidth}
    \parbox[t][1em][t]{\textwidth}{\centering\bf MMT}\par
    \begin{lstlisting}[language=MMT,linewidth=\textwidth]
NP : type
VP : type
S  : type
make_S :
  NP \raa VP \raa S \end{lstlisting}
\end{minipage}
}
        \only<3>{% Used in slides/glif-components.tex
% Has to be in separate file...
\begin{minipage}[t]{0.45\textwidth}
    \parbox[t][1em][t]{\textwidth}{\centering\bf Logic}\par
    \begin{lstlisting}[language=MMT,linewidth=\textwidth]
o : type //propositions
\neg : o \raa o
\wedge : o \raa o \raa o
\vee : o \raa o \raa o

\iota : type //individuals
\forall : (\iota \raa o) \raa o
\exists : (\iota \raa o) \raa o \end{lstlisting}
\end{minipage}\hskip2em
\begin{minipage}[t]{0.3\textwidth}
    \parbox[t][1em][t]{\textwidth}{\centering\bf Domain Theory}\par
    \begin{lstlisting}[language=MMT,linewidth=\textwidth]
paint : \iota \raa o
quiet : \iota \raa o
ahmed : \iota
berta : \iota \end{lstlisting}

    \footnotesize
    \vspace{1.5em}
    \hskip-1em
    idea:
    $\forall f$
    or $\forall \lambda x.f(x)$\par
    \hskip-1em instead of $\forall x.f(x)$
\end{minipage}
}
        \only<4>{\textbf{Semantics Construction}

\textit{map symbols in abstract syntax to terms in logic/domain theory}

\vspace{0.5em}
\begin{minipage}[t]{0.4\textwidth}
\parbox[t][1em][t]{\textwidth}{\centering Simple setting}\par
\ifpart{switchtomathexample}{
    \lstinputlisting[language=MMT]{slides/misc/snippets/s000.txt}
}{
    \lstinputlisting[language=MMT]{slides/misc/snippets/s001.txt}
}
\end{minipage}
\hspace{1em}
\begin{minipage}[t]{0.5\textwidth}
\parbox[t][1em][t]{\textwidth}{\centering More advanced}\par
\ifpart{switchtomathexample}{
    \lstinputlisting[language=MMT]{slides/misc/snippets/s002.txt}
}{
    \lstinputlisting[language=MMT]{slides/misc/snippets/s003.txt}
}
\end{minipage}
}
    \end{minipage}
\end{frame}

\begin{frame}[fragile]
    \frametitle{Example: Parsing + Semantics Construction}
    {\centering\str{Ahmed and Berta paint}\par}\vspace{1em}
    \hspace{0.49\textwidth}$\downarrow_{\text{parsing}}$\par\vspace{1em}
    {\centering\color{logicfont!50!nlfont} sentence (andNP ahmed berta) paint\par}\vspace{1em}
    \hspace{0.49\textwidth}$\downarrow_{\text{semantics construction}}$\par\vspace{1em}
    {\color{logicfont}\footnotesize\lstinline[language=MMT]|(\lambdan.\lambdav.n v) ((\lambdaa.\lambdab.\lambdap.a p \wedge b p) (\lambdap.p ahmed) (\lambdap.p berta)) paint|}\par\vspace{1em}
    \hspace{0.49\textwidth}$\downarrow_{\text{$\beta$-reduction}}$\par\vspace{1em}
    \hspace{7.5em}{\color{logicfont}\small\lstinline[language=MMT]|paint ahmed \wedge paint berta|}
\end{frame}

\begin{frame}
    \frametitle{Components of GLIF: ELPI}
    \enablepart{hlelpi}
    \includestandalone[width=\textwidth]{fig/glif-architecture}
\end{frame}

\begin{frame}[fragile]
    \frametitle{Components of GLIF: ELPI}
    \begin{itemize}
        \item Implementation and extension of $\lambda$Prolog\com{$\approx$ Prolog + HOAS}
        \item MMT can generate logic signatures
        \item First experiments with prover generation
        \item Generic inference/reasoning step after semantics construction
    \end{itemize}
    \lstset{basicstyle=\footnotesize\ttfamily}

    \vspace{1em}
    \begin{minipage}[t]{\textwidth}
        \centering
        \begin{minipage}[t]{0.5\textwidth}
            \begin{lstlisting}[language=ELPI,frame=single]
kind o type.
type not o -> o.
type and o -> o -> o.

kind i type.
type forall (i -> o) -> o.
            \end{lstlisting}
        \end{minipage}
    \end{minipage}
\end{frame}

\begin{frame}
    \frametitle{Components of GLIF: Jupyter}
    \enablepart{hljupyter}
    \includestandalone[width=\textwidth]{fig/glif-architecture}
\end{frame}

\begin{frame}
    \frametitle{Components of GLIF: Jupyter}
    \begin{itemize}
        \item Unified, notebook-based interface
        \item Supports implementation and testing
        \item Useful for prototype, demos, teaching, \dots
    \end{itemize}

    \centering
    \vspace{1.5em}
    \fbox{\includegraphics[trim={0 0 20cm 5.7cm},clip,width=0.7\textwidth]{img/screenshot-glif-1.png}}
\end{frame}

}

% \begin{frame}[fragile]
    \frametitle{Example: Tableaux Machine~\cite{KohKol:ramgpm03}}
    \begin{itemize}
        \item Can use tableaux for model generation
        \item Tableau machine: pick ``best'' branch as model and continue there with next sentence
            \com{like a human?}
    \end{itemize}

    \vspace{1.5em}
    \begin{minipage}[t][4.5cm]{\textwidth}
        \only<1>{\setshowlevel{1}}
        \only<2>{\setshowlevel{2}}
        \only<3>{\setshowlevel{3}}
        % \only<4>{\setshowlevel{4}}
        \includestandalone{fig/tab-machine-simple}
    \end{minipage}
\end{frame}

\begin{frame}
    \frametitle{Example: Tableaux Machine}
    \only<1>{\setshowlevel{1}}
    \only<2>{\setshowlevel{2}}
    \only<3>{\setshowlevel{3}}
    \only<4>{\setshowlevel{4}}
    \makebox[\linewidth]{\includestandalone[scale=0.9]{fig/tab-machine-complex}}
\end{frame}

% 
% \begin{frame}[fragile]
    \frametitle{Example: Translation}
    \begin{itemize}
        \item Two German words for \str{cousin}, depending on the gender
        \item Two entries in abstract syntax: \verb|cousin_female| and \verb|cousin_male|
        \item Use inference to discard ASTs
    \end{itemize}
    
    \vspace{2em}
    \small
    \begin{minipage}[t][5cm]{\textwidth}
        \begin{tikzpicture}
            \node(eng) at (-4,1) {\parbox{4.2cm}{\str{Kim is Ahmed's cousin and the father of Grace}}};
                \node(ger1) at (-4,-0.5) {\parbox{4.2cm}{\str{Kim ist Ahmeds {\upshape\bf Cousine} und Graces Vater}}};
                \node(ger2) at (-4,-2.0) {\parbox{4.2cm}{\str{Kim ist Ahmeds {\upshape\bf Cousin} und Graces Vater}}};
            \node(ast1) at (0,1) {AST$_1$};
            \node(ast2) at (0,-1) {AST$_2$};
            \only<2->{
                \node(log1) at (3,1) {\color{logicfont} \parbox{2.2cm}{$female(kim) \wedge$ $male(kim)$}};
                \node(log2) at (3,-1) {\color{logicfont} \parbox{2.2cm}{$male(kim) \wedge$ $male(kim)$}};
            }
            \draw[->,thick] (eng) -- (ast1);
            \draw[->,thick] (eng) -- (ast2);
                \draw[->,thick] (ast1) -- (ger1);
                \draw[->,thick] (ast2) -- (ger2);
            \only<2->{
                \draw[->,thick] (ast1) -- (log1);
                \draw[->,thick] (ast2) -- (log2);
            }
            \only<3>{
                \draw[ultra thick,red] (2,1.5) -- (4,0.5);
                \draw[ultra thick,red] (2,0.5) -- (4,1.5);
                
                \draw[ultra thick,red] (-0.5,1.5) -- (0.5,0.5);
                \draw[ultra thick,red] (-0.5,0.5) -- (0.5,1.5);

                \draw[ultra thick,red] (-6,0.0) -- (-2,-1.0);
                \draw[ultra thick,red] (-6,-1.0) -- (-2,0.0);
            }
        \end{tikzpicture}
    \end{minipage}
\end{frame}

% 
% 
% \begin{frame}
%     \frametitle{GLIF Summary}
%     \includestandalone[width=\textwidth]{fig/glif-architecture}
% \end{frame}
% 
% \begin{frame}
%     \frametitle{Supporting the Semantic/Pragmatic Analysis}
%     Observation:
%     \begin{itemize}
%         \item Parsing and semantics construction are based on specialized frameworks
%         \item ELPI is a more general programming language
%     \end{itemize}
%     
%     \vspace{1em}
%     Can we do something more specialized?
%     \begin{itemize}
%         \item Not really -- there is no ``standard recipe'' for
%             semantic/pragmatic analysis
%         \item But: It usually involves inferential reasoning
%     \end{itemize}
% 
%     \vspace{2em}
%     \centering
%     \bfseries Let's generate provers!\par
% \end{frame}
% 
% \bgroup

% \providecolorgroup{inf}{blue!50!red}
\providecolorgroup{inf}{black}

% used for highlighting parts of the code.
% probably better solutions exist...
\lstset{literate={*1}{\color{hlfont}}{1} {*2}{\color{inffont}}{1} {*3}{\color{inffont!50}}{1}}

\def\proofvdots#1{
    \let\tmpvskip=\extraVskip
    \def\extraVskip{-2pt}
    \noLine
    \UnaryInfC{{$\raisebox{6pt}\vdots$}}
    \noLine
    #1
    \let\extraVskip=\tmpvskip
}
\newcommand\tabdivider{\;\bigl|\;}

\begin{frame}[fragile]
    \frametitle{Natural Deduction in MMT/LF}
    \begin{minipage}{0.9\textwidth}
        \centering
        \begin{minipage}{0.49\textwidth}
            \begin{prooftree}
                \AxiomC{$A \wedge B$}
                \RightLabel{$\wedge El$}
                \UnaryInfC{$A$}
            \end{prooftree}
        \end{minipage}
        \begin{minipage}{0.49\textwidth}
            \begin{prooftree}
                \def\defaultHypSeparation{\hskip 0pt}
                \AxiomC{$A \vee B$}
                \AxiomC{$\,\,[A]^1$}
                \proofvdots{\UnaryInfC{$C$}}
                \AxiomC{$\,\,[B]^1$}
                \proofvdots{\UnaryInfC{$C$}}
                \RightLabel{$\vee E^1$}
                \TrinaryInfC{$C$}
            \end{prooftree}
        \end{minipage}
    \end{minipage}

    \vspace{1.5em}
    \begin{lstlisting}[language=MMT]
    // \vdashX is type of proofs for X (judgments as types)
    \vdash : o \raa type

    \wedgeEl : \PiAo\PiBo \vdashA\wedgeB \raa \vdashA
    \veeE  : \PiAo\PiBo\PiCo \vdashA\veeB \raa (\vdashA \raa \vdashC) \raa (\vdashB \raa \vdashC) \raa \vdashC
    \end{lstlisting}
\end{frame}

\begin{frame}[fragile]
    \frametitle{Generating Provers in ELPI}
%     \begin{itemize}
%         \item ELPI is an extension of $\lambda$Prolog \com{$\approx$ Prolog + HOAS}\note{ELPI was developed by Claudio and others}
%         \item Optimized for fast execution of logical algorithms \com{type inference, unification, proof search, \dots}
%     \end{itemize}
    \makebox[2.5cm][l]{\textbf{LF rule}} \lstinline[language=MMT]|\wedgeEl : \PiAo\PiBo \vdashA\wedgeB \raa \vdashA|

    \vspace{1.0em}
    \textbf{ELPI equivalent}

    \vspace{0.5em}
    \makebox[2.5cm][r]{direct:$\;\;$} \lstinline[language=ELPI]|pi A \ pi B \ ded (and A B) => ded A.|

    \vspace{0.5em}
    \makebox[2.5cm][r]{syn. sugar:$\;\;$} \lstinline[language=ELPI]|ded A :- ded (and A B).|

% \end{frame}
% 
% \begin{frame}[fragile]
%     \frametitle{From LF to ELPI}
    % \textbf{Or-elimination}
    \vspace{1.5em}
    \pause

    \begin{block}{{\bfseries Example:} Or-Elimination}
    \makebox[1.2cm][l]{LF:}\begin{minipage}{0.85\textwidth}
        \lstinline[keepspaces=true,language=MMT]|\veeE : \PiAo\PiBo\PiCo \vdashA\veeB \raa (\vdashA \raa \vdashC) \raa (\vdashB \raa \vdashC) \raa \vdashC|
    \end{minipage}

    \vspace{0.5em}
    \makebox[1.2cm][l]{ELPI:}\lstinline[language=ELPI,keepspaces=true]|ded C :- ded (or A B), ded A => ded C, ded B => ded C.|
    \end{block}

    \vspace{0.5em}

    \begin{block}{{\bfseries Example:} Forall-Introduction}
    % \textbf{Forall-introduction}
    \makebox[1.2cm][l]{LF:}\begin{minipage}{0.85\textwidth}
        \lstinline[keepspaces=true,language=MMT]|\forallI : \PiPio (\Pixi \vdashP x) \raa \vdash\forallP|
    \end{minipage}

    \vspace{0.5em}
    \makebox[1.2cm][l]{ELPI:}\lstinline[language=ELPI,keepspaces=true]|ded (forall P) :- pi x \ ded (P x).|
    \vspace{0.5em}
    \end{block}
\end{frame}

\begin{frame}[fragile]
    \frametitle{Controlling the Proof Search}
    \begin{itemize}
        \item Problem: Search diverges \com{searching harder than checking}
        \item Solution: Control search with helper predicates:
            \com{inspired by ProofCert project by Miller et al.}\note{ProofCert assumes a focused logic, we don't}
            % \\{ \itshape\color{black!50}\makebox[10cm][r]{(inspired by ProofCert project by Miller et al.)}}
            \begin{itemize}
                \item Intuition: Decide whether to apply rule
                \item Do not affect correctness
                \item Extra argument tracks aspects of proof state
            \end{itemize}
    \end{itemize}

    \vspace{1.5em}
    \makebox[1.2cm][l]{Before:}{{
    \lstinline[language=ELPI,keepspaces=true]|ded*1 *2A :-*1                      *2ded *1   *2(and A B).|
    }}

    \vspace{0.5em}
    \makebox[1.2cm][l]{Now:}{{
    \lstinline[language=ELPI,keepspaces=true]|ded*1X*2A :- *1help/andEl X A B X1, *2ded *1X1 *2(and A B).|
    }}

%     \vspace{2.0em}
%     Example helper for depth-limit:
% 
%     \vspace{0.5em}
%     \lstinline[language=ELPI,keepspaces=true]|    help/andEl (idcert N) _ _ (idcert N1) :- N > 0, N1 is N - 1.|
\end{frame}

\begin{frame}[fragile]
    \frametitle{Helper Predicates}
        \renewcommand{\arraystretch}{1.5}
    \begin{tabular}{l p{4cm} p{4.5cm}}
        \textbf{Name} & \textbf{Predicate} & \textbf{Argument} \\
        Iter. deepening & checks depth & remaining depth \\
        Proof term & generates term & proof term \\
        Product & calls other predicates & arguments for other predicates \\
        Backchaining &
            \footnotesize Prolog's backchaining ($\approx$ forward reasoning from axioms via $\Rightarrow$/$\forall$ elimination rules) &
            \footnotesize pattern of formula to be proven (e.g. a conjunction) \\
        % Backchaining & \multicolumn2{p{7cm}}{\footnotesize Restricts new formulae in e.g. elimination rules to those that could be proven by forward reasoning} \\
    \end{tabular}

    \vspace{1.5em}
    \begin{block}{\textbf{Example helper:} Iterative deepening}
        \lstinline[language=ELPI,keepspaces=true]|help/andEl (idcert N) _ _ (idcert N1) :- N > 0, N1 is N - 1.|
    \end{block}

%     Example call:
%     \begin{lstlisting}[language=ELPI]
%     ?- ded (prodcert (idcert 2) (ptcert Proof)) (impl a (or a b)).
% 
%     Proof = implI a (or a b) (orIl a b (i a)).
%     \end{lstlisting}
\end{frame}

\begin{frame}[fragile]
    \frametitle{Tableau Provers}
    \note{We can scale in terms of logics supported or (orthogonally) in terms of prover strategies.
    We went for the latter.}
    \begin{minipage}[b][2cm][b]{0.4\textwidth}
        \begin{prooftree}
            \AxiomC{$\;\tabfalse{A \wedge B}$}
            \RightLabel{$\tabfalse\wedge$}
            \UnaryInfC{$\tabfalse{A} \tabdivider \tabfalse{B}$}
        \end{prooftree}
    \end{minipage}
    \begin{minipage}[b][2cm][b]{0.4\textwidth}
        \def\defaultHypSeparation{\hskip 0pt}
        \begin{prooftree}
            \AxiomC{$\tabfalse{A \wedge B}$}
            \AxiomC{$\;[\tabfalse{A}]$}
            \proofvdots{\UnaryInfC{$\bot$}}
            \AxiomC{$\;[\tabfalse{B}]$}
            \proofvdots{\UnaryInfC{$\bot$}}
            \RightLabel{$\tabfalse\wedge$}
            \TrinaryInfC{$\bot$}
        \end{prooftree}
    \end{minipage}

    \vspace{2em}
    \makebox[1.0cm][l]{LF:} \lstinline[language=MMT]|\wedge\tabfalse : \PiAo\PiBo A\wedgeB\tabfalse \raa (A\tabfalse \raa \bot) \raa (B\tabfalse \raa \bot) \raa \bot|

    \vspace{0.5em}
    \makebox[1.0cm][l]{ELPI:} \lstinline[language=ELPI]|closed *3X *2:- *3help/andF X A B X1 X2 X3, *2f *3X1 *2(and A B),|
    \lstinline[language=ELPI,keepspaces=true]|                         f*3/hyp *2A => closed *3X2*2, f*3/hyp*2 B => closed *3X3*2.|

    \vspace{2em}
    With iterative deepening we get a working prover!

    $\rightarrow$ Other helpers result in more efficient provers
\end{frame}

\egroup

% 
% \begin{frame}
%     \frametitle{Conclusion: Prototyping NLU Pipelines}
%     \includestandalone[width=\textwidth]{fig/glif-spec}
% \end{frame}

\appendix

\begin{frame}[allowframebreaks,t]
    \frametitle{References}
    \printbibliography
\end{frame}

\end{document}
