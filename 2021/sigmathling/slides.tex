\documentclass[aspectratio=169]{beamer}
\usepackage{standalone}

\usepackage{stmaryrd}
\usepackage{listings}
\usepackage{bussproofs}

\usepackage[hyperref=auto,style=alphabetic,backend=bibtex]{biblatex}
\addbibresource{kwarcpubs.bib}
\addbibresource{extpubs.bib}
\addbibresource{extcrossrefs.bib}
\usepackage{appendixnumberbeamer}
\usepackage{tikz}
\usepackage{tikz-qtree}
\usetikzlibrary{arrows.meta}
\usetikzlibrary{mmt}
\usetikzlibrary{docicon}

\usetheme{Pittsburgh}
\setbeamertemplate{footline}[frame number]
\setbeamertemplate{navigation symbols}{}
\usecolortheme{beaver}
\setbeamertemplate{frametitle}[default][left]
% \setbeamersize{text margin left=3em}

\usepackage{utils/colors}
\usepackage[forbeamer]{utils/basic}
\usepackage{utils/operators}
\usepackage{utils/lstmisc}
\usepackage{utils/lstmmt}

\lstset{basicstyle=\ttfamily}
\lstset{commentstyle=\itshape\color{commentfont}}

\title{A Symbolic Framework for Mathematical Language Understanding}

\author{Jan Frederik Schaefer}
\institute{FAU Erlangen-N\"urnberg}
\date{\textbf{SIGMathLing Seminar} \\ \textit{virtual event} \\ January 18, 2021 }

\begin{document}
\frame\titlepage

\begin{frame}
    \frametitle{A Case for Symbolic Approaches}
    {
        \centering
        \str{Every integer is even.} $\quad\leadsto\quad$ {\color{logicfg}$\forall x.int(x) \Rightarrow even(x)$}\par
    }

    \vspace{1em}
    \begin{itemize}
        \item[\textbf{+}] No need for training data
        \item[\textbf{+}] No need for resource-heavy training
        \item[\textbf{+}] Verifiable, predictable, accurate
        \item[\textbf{--\hspace{0.15em}}] Coverage very limited
    \end{itemize}

    \vspace{1em}
    Sometimes the pros outweigh the cons:
    \begin{itemize}
        \item Need for high reliability\com{CNLs}
            \begin{itemize}
                \item Proof verification
                \item Fabstracts
                \item ...
            \end{itemize}
        \item Prototyping
    \end{itemize}

    \pause
    \centering
    \vspace{1em}
    \bfseries GLIF: A framework for prototyping symbolic NLU
\end{frame}

\begin{frame}[fragile]
    \frametitle{Teaser: Input Language for SageMath}
    \lstset{basicstyle=\small\ttfamily,columns=fullflexible,commentstyle={\sffamily\color{nlfg}},morecomment=[l]{Let},morecomment=[l]{What}}
    \begin{lstlisting}
Enter command: Let G be the dihedral group of order 8.
gVar = DihedralGroup(int(8)//2)


Enter command: Let A_N be a notation for the alternating group on N symbols.
def aVar(nVar): return AlternatingGroup(nVar)


Enter command: What are the cardinalities of G and A_5?
print(gVar.cardinality())
print(aVar(int(5)).cardinality())
sage: 8
sage: 60
    \end{lstlisting}
\end{frame}

\begin{frame}
    \frametitle{GLIF: Prototyping Symbolic NLU}
    \begin{itemize}
        \item Claim: Prototyping NLU is important but requires much work
        \item GLIF as a dedicated, declarative framework for NLU prototyping
        \item Montague's approach:
            \begin{enumerate}
                \item Parsing
                \item Compositional semantics construction\com{lots of $\lambda$s}
            \end{enumerate}
        \item We also need
            \begin{enumerate}
                \setcounter{enumi}{2}
                \item Semantic/pragmatic analysis\com{disambiguation, \textellipsis}
            \end{enumerate}
    \end{itemize}
\end{frame}

\enablepart{switchtomathexample}
\providedisablepart{switchtomathexample}
\providedisablepart{delayspa}


\begin{frame}
    \frametitle{GLIF: Grammatical Logical Inference Framework}
    \centering

    \vspace{1.5em}
    \ifpart{delayspa}{\only<1-1>{\disablepart{sempragarrow}}}{}
    \only<1-\ifpart{delayspa}21>{\disablepart{includejupyter}}
    \only<1-\ifpart{delayspa}21>{\disablepart{gfbox}\disablepart{mmtbox}\disablepart{elpibox}}
    \includestandalone[width=\textwidth]{fig/glif-architecture}

    \vspace{1em}
    \begin{minipage}[t][2cm]{\textwidth}
        \only<1-\ifpart{delayspa}21>{
            \begin{tikzpicture}
                \ifpart{switchtomathexample}{
                    \node(str) at (-4,0) {\str{Every integer is even}};
                    % REQUIRES \usepackage{tikz-qtree}
                    \node(ast) at (0,0) {\color{nlfont!50!logicfont}
                        \resizebox{1.5cm}{!}{\tikzset{edge from parent/.append style={very thick}}
                            \Tree [ .\textbf{sentence} [ .\textbf{everyNP} \textbf{int} ] [ .\textbf{beVP} \textbf{even} ] ]
                        }};
                    \node(log) at (4.0,0) {\color{logicfont}$\forall x.int(x) \Rightarrow even(x)$};
                }{
                    \node(str) at (-4,0) {\str{Ahmed and Berta paint.}};
                    % REQUIRES \usepackage{tikz-qtree}
                    \node(ast) at (0,0) {\color{nlfont!50!logicfont}
                        \resizebox{1.5cm}{!}{\tikzset{edge from parent/.append style={very thick}}
                        \Tree [ .\textbf{mkS} [ .\textbf{andNP} [ \textbf{Ahmed} \textbf{Berta} ] ] \textbf{paint} ]
                        }};
                    \node(log) at (2.8,0) {\color{logicfont}$p(a) \wedge p(b)$};
                }
                \draw[-{Straight Barb[length=6.3,width=5.0]},gray] (str) -- (ast);
                \draw[-{Straight Barb[length=6.3,width=5.0]},gray] (ast) -- (log);
            \end{tikzpicture}
        }
        \only<\ifpart{delayspa}32>{
            \setlength{\arrayrulewidth}{1.0pt}
            \begin{tabular}{r@{\hskip3pt} l l}
                \vspace{0.1em} &\textbf{GF}   &{(= \textbf{grammar} framework)}\\
                \vspace{0.1em}+&\textbf{MMT}  &{(= \textbf{logic} framework)}\\
                \vspace{0.1em}+&\textbf{ELPI} &{(= \textbf{inference} framework)}\\
                \hline
                \\[-1em]
                =&\textbf{GLIF} &{(= \textbf{natural language understanding} framework)}\\
                \end{tabular}
        }
    \end{minipage}
\end{frame}


\disablepart{showspa}
\disablepart{showjupyter}
\begin{frame}
    \frametitle{Components of GLIF: GF}
    \enablepart{hlgf}
    \includestandalone[width=\textwidth]{fig/glif-architecture}
\end{frame}

\begin{frame}[fragile]
    \frametitle{Components of GLIF: Grammatical Framework \cite{GF:on}}
    \begin{itemize}
        \item Specialized for developing natural language grammars
        \item Separates abstract and concrete syntax\par
            \quad\lstinline[language=GF]|sentence : NP -> VP -> S;               --abstract|\par
            \quad\lstinline[language=GF]|sentence np vp = np.s ++ vp.s!np.n;   --concrete|
        \item Abstract syntax based on type theory
        \item Comes with large library \com{$\ge 36$ languages}
    \end{itemize}

    \vspace{2em}
    \hspace{2.5em}\begin{tikzpicture}
        \node(eng) at (-2,0.7) {\str{Ahmed paints}};
        \node(ger) at (-2,-0.7) {\str{Ahmed zeichnet}};
        \node(ast) at (4,0) {
                    \color{logicfont!50!nlfont}
                    \resizebox{2.cm}{!}{\tikzset{edge from parent/.append style={very thick}}
                        \Tree [ .\texttt{sentence} \texttt{ahmed} \texttt{paint} ]
                    }
                };
        \draw[<->, thick] (eng) to node[above,rotate=-9]{Eng. concr. syn.} (ast);
        \draw[<->, thick] (ger) to node[below,rotate=9]{Ger. concr. syn.} (ast);
    \end{tikzpicture}
\end{frame}

\begin{frame}
    \frametitle{Components of GLIF: MMT}
    \enablepart{hlmmt}
    \includestandalone[width=\textwidth]{fig/glif-architecture}
\end{frame}

\begin{frame}[fragile]
    \frametitle{Components of GLIF: MMT}
    \lstset{frame=single}
    \begin{itemize}   
        \item Modular logic development and knowledge representation
        \item Not specialized in one logical framework \com{we use LF}
        \item We will use MMT to:
        \begin{itemize}
            \item { \only<2>{\bf\color{hlfont}} represent abstract syntax }
            \item { \only<3>{\bf\color{hlfont}} specify target logic and domain theory}
            \item { \only<4>{\bf\color{hlfont}} specify semantics construction}
        \end{itemize}
    \end{itemize}
    \lstset{basicstyle=\footnotesize\ttfamily}
    
    \vspace{1em}
    \begin{minipage}[t][4cm]{\textwidth}
        \centering
        \only<2>{% Used in slides/glif-components.tex
% Has to be in separate file...
\begin{minipage}[t]{0.4\textwidth}
    \parbox[t][1em][t]{\textwidth}{\centering\bf GF}\par
    \begin{lstlisting}[language=GF,linewidth=\textwidth]
cat
  NP; VP; S;
fun
  make_S :
    NP -> VP -> S;
    \end{lstlisting}
\end{minipage}
\begin{minipage}[t]{0.1\textwidth}\vskip3em \centering\LARGE$\mapsto$\end{minipage}
\begin{minipage}[t]{0.35\textwidth}
    \parbox[t][1em][t]{\textwidth}{\centering\bf MMT}\par
    \begin{lstlisting}[language=MMT,linewidth=\textwidth]
NP : type
VP : type
S  : type
make_S :
  NP \raa VP \raa S \end{lstlisting}
\end{minipage}
}
        \only<3>{% Used in slides/glif-components.tex
% Has to be in separate file...
\begin{minipage}[t]{0.45\textwidth}
    \parbox[t][1em][t]{\textwidth}{\centering\bf Logic}\par
    \begin{lstlisting}[language=MMT,linewidth=\textwidth]
o : type //propositions
\neg : o \raa o
\wedge : o \raa o \raa o
\vee : o \raa o \raa o

\iota : type //individuals
\forall : (\iota \raa o) \raa o
\exists : (\iota \raa o) \raa o \end{lstlisting}
\end{minipage}\hskip2em
\begin{minipage}[t]{0.3\textwidth}
    \parbox[t][1em][t]{\textwidth}{\centering\bf Domain Theory}\par
    \begin{lstlisting}[language=MMT,linewidth=\textwidth]
paint : \iota \raa o
quiet : \iota \raa o
ahmed : \iota
berta : \iota \end{lstlisting}

    \footnotesize
    \vspace{1.5em}
    \hskip-1em
    idea:
    $\forall f$
    or $\forall \lambda x.f(x)$\par
    \hskip-1em instead of $\forall x.f(x)$
\end{minipage}
}
        \only<4>{\textbf{Semantics Construction}

\textit{map symbols in abstract syntax to terms in logic/domain theory}

\vspace{0.5em}
\begin{minipage}[t]{0.4\textwidth}
\parbox[t][1em][t]{\textwidth}{\centering Simple setting}\par
\ifpart{switchtomathexample}{
    \lstinputlisting[language=MMT]{slides/misc/snippets/s000.txt}
}{
    \lstinputlisting[language=MMT]{slides/misc/snippets/s001.txt}
}
\end{minipage}
\hspace{1em}
\begin{minipage}[t]{0.5\textwidth}
\parbox[t][1em][t]{\textwidth}{\centering More advanced}\par
\ifpart{switchtomathexample}{
    \lstinputlisting[language=MMT]{slides/misc/snippets/s002.txt}
}{
    \lstinputlisting[language=MMT]{slides/misc/snippets/s003.txt}
}
\end{minipage}
}
    \end{minipage}
\end{frame}

\begin{frame}[fragile]
    \frametitle{Example: Parsing + Semantics Construction}
    {\centering\str{Ahmed and Berta paint}\par}\vspace{1em}
    \hspace{0.49\textwidth}$\downarrow_{\text{parsing}}$\par\vspace{1em}
    {\centering\color{logicfont!50!nlfont} sentence (andNP ahmed berta) paint\par}\vspace{1em}
    \hspace{0.49\textwidth}$\downarrow_{\text{semantics construction}}$\par\vspace{1em}
    {\color{logicfont}\footnotesize\lstinline[language=MMT]|(\lambdan.\lambdav.n v) ((\lambdaa.\lambdab.\lambdap.a p \wedge b p) (\lambdap.p ahmed) (\lambdap.p berta)) paint|}\par\vspace{1em}
    \hspace{0.49\textwidth}$\downarrow_{\text{$\beta$-reduction}}$\par\vspace{1em}
    \hspace{7.5em}{\color{logicfont}\small\lstinline[language=MMT]|paint ahmed \wedge paint berta|}
\end{frame}

\begin{frame}
    \frametitle{Components of GLIF: ELPI}
    \enablepart{hlelpi}
    \includestandalone[width=\textwidth]{fig/glif-architecture}
\end{frame}

\begin{frame}[fragile]
    \frametitle{Components of GLIF: ELPI}
    \begin{itemize}
        \item Implementation and extension of $\lambda$Prolog\com{$\approx$ Prolog + HOAS}
        \item MMT can generate logic signatures
        \item First experiments with prover generation
        \item Generic inference/reasoning step after semantics construction
    \end{itemize}
    \lstset{basicstyle=\footnotesize\ttfamily}

    \vspace{1em}
    \begin{minipage}[t]{\textwidth}
        \centering
        \begin{minipage}[t]{0.5\textwidth}
            \begin{lstlisting}[language=ELPI,frame=single]
kind o type.
type not o -> o.
type and o -> o -> o.

kind i type.
type forall (i -> o) -> o.
            \end{lstlisting}
        \end{minipage}
    \end{minipage}
\end{frame}

\begin{frame}
    \frametitle{Components of GLIF: Jupyter}
    \enablepart{hljupyter}
    \includestandalone[width=\textwidth]{fig/glif-architecture}
\end{frame}

\begin{frame}
    \frametitle{Components of GLIF: Jupyter}
    \begin{itemize}
        \item Unified, notebook-based interface
        \item Supports implementation and testing
        \item Useful for prototype, demos, teaching, \dots
    \end{itemize}

    \centering
    \vspace{1.5em}
    \fbox{\includegraphics[trim={0 0 20cm 5.7cm},clip,width=0.7\textwidth]{img/screenshot-glif-1.png}}
\end{frame}


\begin{frame}
    \frametitle{Example: Controlled Natural Languages}
    \begin{itemize}
        \item Formal languages
        \item that are a subset of natural language
        \item and have fixed semantics\com{formal verification, \dots}
    \end{itemize}

    \vspace{1em}
    \str{$S$ is a subset of every set iff $S$ is empty}

    $\leadsto$ {\color{logicfont}$(\forall V_{new}. set(V_{new}) \Rightarrow subset(V_S, V_{new})) \Leftrightarrow empty(V_S)$}

    \pause
    \vspace{1.5em}
    Use inference for disambiguation:

    \vspace{1em}
    \only<2>{\disablepart{crossout}}
    \only<3>{\enablepart{crossout}}
    \includestandalone[width=\textwidth]{fig/cnl-simple-discard} 
\end{frame}


\begin{frame}[fragile]
    \frametitle{Example: Input Language for SageMath}
    {\centering\begin{adjustbox}{}
    \begin{lstlisting}
sage: g = AlternatingGroup(5)
sage: g.cardinality()
60
    \end{lstlisting}
    \end{adjustbox}\par
    \vspace{1em}
    \str{Let G be the alternating group on 5 symbols. What is the cardinality of G?}\par
}

    \vspace{2em}
    \begin{itemize}
        \item Can we make a natural input language for SageMath?\com{WolframAlpha-like}
        \item GLIF Prototype:
            \begin{itemize}
                \item Parsing
                \item Semantics construction translates to SageMath command (not logic)
            \end{itemize}
    \end{itemize}
\end{frame}

\begin{frame}
    \frametitle{Example: Input Language for SageMath -- Grammar}
    \centering
    \only<1>{\includegraphics[scale=0.35]{img/glif-sage-ast-1.png}}
    \only<2>{\includegraphics[scale=0.35]{img/glif-sage-ast-2.png}}
    \par
\end{frame}

\begin{frame}[fragile]
    \frametitle{Example: Input Language for SageMath -- Semantics Construction}
    \begin{itemize}
        \item Target logic: Python/SageMath commands
        \item Can experiment with ideas (e.g. notations)
    \end{itemize}

    \centering
    \onslide<1->{
        \vspace{1em}
        \str{let G be the alternating group on 5 symbols}\par
        $\downarrow$\par
        {\color{logicfg!50}assign gVar (alternating{\_}group (int{\_}term 5))}\par
        {\color{logicfg}\ttfamily g = AlternatingGroup(int(5))}\par
    }
    \onslide<2>{
        \vspace{2em}
        \str{let $|$ G $|$ be a notation for the cardinality of G}\par
        $\downarrow$\par
        {\color{logicfg}\ttfamily def bar(gVar): return gVar.cardinality()}\par
        \vspace{2em}
        \str{let D{\_}N be a notation for the dihedral group of order 2 * N}\par
    }
\end{frame}

\begin{frame}[fragile]
    \frametitle{Example: Input Language for SageMath}
    \lstset{basicstyle=\small\ttfamily,commentstyle={\sffamily\color{nlfg}},morecomment=[l]{Let},morecomment=[l]{What}}
    \begin{lstlisting}[columns=flexible]
Enter command: What are the Cayley tables of the alternating groups on 2 and 3 symbols?
    \end{lstlisting}
    \vskip-1em
    \begin{lstlisting}
print(AlternatingGroup(int(2)).cayley_table())
print(AlternatingGroup(int(3)).cayley_table())
sage:
*  a
 +--
a| a

sage:
*  a b c
 +------
a| a b c
b| b c a
c| c a b
    \end{lstlisting}
\end{frame}

\begin{frame}[fragile]
    \frametitle{Example: Input Language for SageMath}
    \begin{itemize}
        \item Took just a few hours to create prototype
        \item Maybe useful for teaching?
        \item GF made it easy to support another language (German)
        \item $\leadsto$ can also translate automatically:
    \end{itemize}

    \centering
    \vspace{1.5em}
    \str{What are the Cayley tables of the alternating groups on 2 and 3 symbols?}\par
    $\downarrow$\par
    \str{Was sind die Verkn\"upfungstafeln der alternierenden Gruppen \"uber 2 und 3 Elemente?}\par
\end{frame}



\begin{frame}
    \frametitle{Conclusion}
    \begin{itemize}
        \item GLIF: Declarative framework for prototyping NLU
        \item Used in a 1-semester course on logic-based NL semantics
        \item First experiments with mathematical language
    \end{itemize}

    \centering
    \vspace{1.5em}
    \includestandalone[width=0.7\textwidth]{fig/glif-architecture}\par
\end{frame}

\appendix

\begin{frame}
    \frametitle{Specification of a GLIF Prototype}
    \includestandalone[width=\textwidth]{fig/glif-spec}
\end{frame}

\begin{frame}[allowframebreaks,t]
    \frametitle{References}
    \printbibliography
\end{frame}

\end{document}
