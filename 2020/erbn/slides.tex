\documentclass[aspectratio=149]{beamer}
\usepackage{standalone}

\usepackage{stmaryrd}
\usepackage{listings}
\usepackage{courier}
\usepackage{adjustbox}

\usepackage[hyperref=auto,style=alphabetic]{biblatex}
\addbibresource{kwarc.bib}
\usepackage{appendixnumberbeamer}
\usepackage{tikz}
\usepackage{tikz-qtree}
\usetikzlibrary{arrows.meta}

\usetheme{Pittsburgh}
\setbeamertemplate{footline}[frame number]
\setbeamertemplate{navigation symbols}{}
\usecolortheme{beaver}
\setbeamertemplate{frametitle}[default][left]
\setbeamersize{text margin left=3em}

\usepackage{utils/colors}
\usepackage[forbeamer]{utils/basic}
\usepackage{utils/operators}
\usepackage{utils/lstmisc}
\usepackage{utils/lstmmt}

\usepackage{fontspec}
\setmonofont[Scale=MatchLowercase]{Consolas}

\lstset{basicstyle=\small\ttfamily}
\lstset{commentstyle=\itshape\color{commentfont}}

\title{ForTheL and GF}

\author{Jan Frederik Schaefer}
\institute{FAU Erlangen-N\"urnberg}
\date{ERBN \\ December 1, 2020 }

\begin{document}
\frame\titlepage

\begin{frame}[fragile]
    \frametitle{ForTheL -- Formal Theory Language}
    Controlled natural language for \emph{System of Automated Deduction}
    \begin{itemize}
        \item[$\rightarrow$] Formal language
        \item[$\rightarrow$] Subset of natural language
    \end{itemize}

    \vspace{0.5em}
    \lstincreaseemptylineheight{-0.6}
    \color{nlfg}
    \begin{lstlisting}[language=ftl]
Signature SetSort.  A set is a notion.
Let S,T denote sets.

Signature ElmSort.  An element of S is a notion.
Let x belongs to X stand for x is an element of X.

Definition DefEmpty.    S is empty iff S has no elements.

Axiom ExEmpty.  There exists an empty set.
    \end{lstlisting}
    \color{black}

    \pause
    \vspace{0.5em}
    Semantics in first-order logic: \color{logicfg}$\exists x.set(x) \wedge empty(x)$.
\end{frame}


\begin{frame}
    \frametitle{GF -- Grammatical Framework}
    ``\textit{A programming language for multilingual grammar applications}''

    \begin{itemize}
        \item Separate abstract and concrete syntax
        \item Often used for high-precision machine translation
    \end{itemize}

    \centering
    \disablepart{showlogic}
    \only<1>{
        \vspace{2em}
        \includestandalone[scale=1.0]{fig/gf-overview-math-1}
        \vspace{2em}
    }
    \only<2>{
        \vspace{1em}
        \includestandalone[scale=0.8]{fig/gf-overview-math-2}
        \vspace{1em}
    }
\end{frame}


\begin{frame}
    \frametitle{My Master's Project}
    Use GLF = (GF + MMT) to parse ForTheL and create logical expressions

    \vspace{2em}

    \includestandalone[scale=0.8]{fig/gf-overview-math-2}
\end{frame}


\begin{frame}
    \frametitle{Conclusion (everything else is optional)}
    \begin{itemize}
        \item ForTheL is a CNL for mathematics.
        \item It is a great case study for GLF (= GF + MMT).
        \item With GF multiple languages and translation can be easily added.
    \end{itemize}
\end{frame}

\begin{frame}[fragile]
    \frametitle{GF in More Detail: Abstract Syntax}
    \centering
    \disablepart{showlogic}
    \includestandalone[width=0.5\textwidth]{fig/gf-overview-math-2}

    \begin{lstlisting}[language=GF]
abstract Grammar = {
  cat            -- "types of nodes"
    Stmt; Term; Notion; Prop;
  fun            -- production rules
    state      : Term -> Prop -> Stmt;
    every      : Notion -> Term;
    derivative : Term -> Term;
    integer    : Notion;
    even       : Prop;
}
    \end{lstlisting}
\end{frame}


\begin{frame}[fragile]
    \frametitle{GF in More Detail: Concrete Syntax}
    \centering
    \disablepart{showlogic}
    \includestandalone[width=0.5\textwidth]{fig/gf-overview-math-2}

    \begin{lstlisting}[language=GF]
concrete GrammarEng of Grammar = {
  lincat
    Stmt = Str; Term = Str; Notion = Str; Prop = Str;
  lin
    state term prop = term ++ "is" ++ prop;
    every notion    = ("every"|"any") ++ notion;
    derivative term = "the derivative of" ++ term;
    integer         = "integer";
    even            = "even";
}
    \end{lstlisting}
\end{frame}


\begin{frame}[fragile]
    \frametitle{GF for ForTheL}
    {
        \centering
        \str{Let a, b be sets that aren't empty}\par
    }
    \vspace{1em}

    \begin{lstlisting}[language=GF]
letAssume : Names -> ClassNoun -> Assume;

-- ClassNoun = {pref : Plurality=>Str; suf : Plurality=>Str};
--   pref: set/sets
--   suf: that isn't empty/that aren't empty

letAssume names cn =
    "let" ++ names.s ++ ("be"|"denote"|"stand for") ++
    indefArt!names.p ++ cn.pref!names.p ++ cn.suf!names.p;
    \end{lstlisting}
\end{frame}

\begin{frame}[fragile]
    \def\logexpr#1{{\color{logicfg}#1}}
    \frametitle{A Closer Look at Notions}
    Example notions: \str{set}, \str{subgroup of $D_8$}, \str{set that isn't empty}

    \vspace{0.5em}
    On the logic side: \logexpr{$set(\cdot)$}, \logexpr{$subgroup(\cdot, D_8)$}, \logexpr{$set(\cdot)\wedge\neg empty(\cdot)$}

    \vspace{0.5em}
    Should \str{subgroup $H$ of $D_8$} be a notion?
    \begin{itemize}
        \item Yes\com{ForTheL does this}
            \begin{itemize}
                \item[\boldmath$+$] Notions are continuous strings
                \item[\boldmath$-$] Semantically tricky
                \item[\boldmath$-$] \str{$G$ is a subgroup $H$ of $D_8$}\com{feels wrong...}
            \end{itemize}
        \item No\com{I did this}
            \begin{itemize}
                \item[\boldmath$+$] Semantically easy
                \item[\boldmath$+$] Definitely isn't only plural/singular yet
                \item[\boldmath$-$] Can't use Resource Grammar Library easily
            \end{itemize}
    \end{itemize}
\end{frame}



\begin{frame}
    \frametitle{Another Conclusion (everything else is optional)}
    Good things:
    \begin{itemize}
        \item GF is great for parsing
        \item Can avoid much over-generation
        \item Relatively extensible
        \item Allows for quick prototyping
    \end{itemize}

    Bad things:
    \begin{itemize}
        \item Problems with using Resource Grammar Library
        \item No dynamic lexicon extension
    \end{itemize}
\end{frame}

\begin{frame}
    \frametitle{Result of my Master's Thesis: GLIF}
    \disablepart{includejupyter}
    \only<2>{\enablepart{hlmmt}}
    \includestandalone[width=\textwidth]{fig/glif-architecture}
\end{frame}

\begin{frame}
    \frametitle{MMT}
    One way to look at it:
    \begin{itemize}
        \item Use logics to represent knowledge
        \item Use logical frameworks to represent logics
        \item Use MMT to implement logical frameworks\com{Meta Meta Tool}
    \end{itemize}

    \vspace{0.7em}
    In GLIF:
    \begin{itemize}
        \item Develop a logic\com{syntax, semantics, calculus}
        \item Describe semantics construction\com{map ASTs to logical expressions}
    \end{itemize}
\end{frame}


\begin{frame}
    \frametitle{Logic in MMT}
    \begin{minipage}[t]{0.49\textwidth}
        \begin{align*}
            o    &:type\\
            \iota&:type\\
            \neg&:o\raa o\\
            \wedge&:o\raa o\raa o\\
            \vee&:o\raa o \raa o\\
            \forall&:(\iota\raa o) \raa o\\
            \exists&:(\iota\raa o) \raa o\\
        \end{align*}
    \end{minipage}
    \begin{minipage}[t]{0.49\textwidth}
        \begin{align*}
            set&:\iota\raa o\\
            int&:\iota\raa o\\
            even&:\iota\raa o\\
            empty&:\iota\raa o\\
            subgroup&:\iota\raa\iota\raa o\\
            derivative&:\iota\raa\iota\\
        \end{align*}
    \end{minipage}
\end{frame}


\begin{frame}[fragile]
    \frametitle{Semantics Construction in MMT}

    \centering
    \vspace{1em}
    {
        \tikzset{frontier/.style={distance from root=90pt}}
        \def\leaf#1{{\itshape\color{nlfg}#1}}
        \color{logicfg!50!nlfg}
        \Tree [ .{hasProp}
            [ .{andId} [ .{a} \leaf{a} ] \leaf{and} [ .{b} \leaf{b} ] ]
            \leaf{are}
            [ .{orProp} [ .{even} \leaf{even} ] \leaf{or} [ .{prime} \leaf{prime} ] ]
        ]

        \centering
        \vspace{1.5em}
        \hfill
        \adjustbox{}{
            \lstinline|hasProp (andId a b) (orProp even prime)|
        }
        \hfill
    }
\end{frame}

\begin{frame}[fragile]
    \frametitle{Semantics Construction in MMT}
    Substitute every node in AST with lambda functions\com{$\lambda x.M$ as $x \mapsto M$}

    \centering
    \vspace{1.5em}
    \color{logicfg!50!nlfg}
    \adjustbox{}{
        \lstinline|hasProp (andId a b) (orProp even prime)|
    }


    \vspace{1.5em}
    \color{logicfg}
    \lstset{commentstyle=\itshape\color{logicfg!50!nlfg!60}}
    \only<1>{
        \begin{lstlisting}[language=MMT]
(\lambdax,p.x(p))                        // haspProp
(
    (\lambdax,y.\lambdap.x(p)\wedgey(p))            // andId
    (\lambdap.p(A))                      // a
    (\lambdap.p(B))                      // b
)
(
    (\lambdap,q.\lambdax.p(x)\veeq(x))            // orProp
    even                           // even
    prime                          // prime
)
\end{lstlisting}



    }
    \only<2>{
        \begin{lstlisting}[language=MMT]
(\lambdax,p.x(p))                 // haspProp
(\lambdap.p(A)\wedgep(B))              // andId a b
(\lambdax.even(x)\veeprime(x))       // orProp even prime
\end{lstlisting}

$$\leadsto_\beta$$

\begin{lstlisting}[language=MMT]
(even(A)\veeprime(A))\wedge(even(B)\veeprime(B))
\end{lstlisting}

    }
\end{frame}

\begin{frame}[fragile]
    \frametitle{ForTheL can be illustrated with Textual Transformations}

    \hspace{1.2em}\str{a and b are even or prime}

    \vspace{0.5em}
    $\leadsto$\str{a is even or prime and b is even or prime}

    \vspace{0.5em}
    $\leadsto$\str{a is even or a is prime and b is even or b is prime}

    \vspace{0.5em}
    $\leadsto$\color{logicfg}$(even(a) \vee prime(a)) \wedge (even(b) \vee prime(b))$
\end{frame}


\begin{frame}
    \frametitle{Another Conclusion (everything else is optional)}
    Good things:
    \begin{itemize}
        \item GF is great for parsing
        \item Can avoid much over-generation
        \item Relatively extensible
        \item Allows for quick prototyping
        \item MMT is a dedicated logic development framework
    \end{itemize}

    Bad things:
    \begin{itemize}
        \item Problems with using GF's Resource Grammar Library
        \item No dynamic lexicon extension
        \item Semantics construction can be tricky
    \end{itemize}
\end{frame}


\begin{frame}
    \frametitle{Method of Fragments}
    \only<1-2>{
        \centering
        \only<1>{\def\fraglevel{0}}
        \only<2>{\def\fraglevel{1}}
        \includestandalone{fig/montague-fragments}

        \begin{minipage}[t][2cm]{\textwidth}\vspace{1em}
            How do we get from messy language to formal logic?\\[0.5em]
            \emph{Montague}~\cite{Montague:efl70}: Look at a ``nice'' subset
            and map into logic.
        \end{minipage}
    }

    \only<3>{
        \centering
        \def\fraglevel{1}
        \includestandalone{fig/montague-fragments}
        
        \begin{minipage}[t][2cm]{0.6\textwidth}\vspace{1em}
            \str{Ahmed paints and Berta is quiet.}\\[0.5em]
            \str{Ahmed doesn't paint.}
        \end{minipage}\hfill
        \begin{minipage}[t][2cm]{0.39\textwidth}\vspace{1em}
            $p(a) \wedge q(b)$\\[0.5em]
            $\neg p(a)$
        \end{minipage}
    }

    \only<4>{
        \centering
        \def\fraglevel{2}
        \includestandalone{fig/montague-fragments}
        
        \begin{minipage}[t][2cm]{0.6\textwidth}\vspace{1em}
            \str{Every student paints and is quiet.}\\[0.5em]
            \str{Nobody paints.}
        \end{minipage}\hfill
        \begin{minipage}[t][2cm]{0.39\textwidth}\vspace{1em}
            $\forall x.s(x) \Rightarrow (p(x) \wedge q(x))$\\[0.5em]
            $\neg \exists x.p(x)$
        \end{minipage}
    }

    \only<5>{
        \centering
        \def\fraglevel{3}
        \includestandalone{fig/montague-fragments}

        \begin{minipage}[t][2cm]{0.6\textwidth}\vspace{1em}
            \str{Ahmed isn't allowed to paint.}\\[0.5em]
            \str{Ahmed and Berta must paint.}
        \end{minipage}\hfill
        \begin{minipage}[t][2cm]{0.39\textwidth}\vspace{1em}
            $\neg\lozenge p(a)$\\[0.5em]
            $(\square p(a)) \wedge \square p(b)$
        \end{minipage}
    }
\end{frame}


\begin{frame}
    \frametitle{Method of Fragments}
    If we only hand-wave, we gloss over problems:

    \hspace{2em}\str{Ahmed paints. He is quiet.}
    $\quad\stackrel{?}{\leadsto}\quad p(a)\wedge q(a)$

    \vspace{1.5em}
    Specify:
    \begin{itemize}
        \item Grammar\com{fixes NL subset}
        \item Target logic
        \item Semantics construction\com{maps parse trees to logic}
    \end{itemize}

    \vspace{1.5em}
    On paper~\cite{Montague:tptoqi73}:\com{difficult to scale}

    \vspace{0.3em}\hspace{2em}\includegraphics[trim=0 0 0 80,clip,width=0.8\textwidth]{fig/montague-tptoqioe.png}
    % \newcommand\VP{\text{\upshape\tiny VP}}
    % \hspace{2em}$\llbracket\text{\strplain{$P_{\VP}$ and $Q_{\VP}$}}\rrbracket_{\VP} = \lambda x. \llbracket\text{\strplain{$P_{\VP}$}}\rrbracket(x) \wedge \llbracket\text{\strplain{$Q_{\VP}$}}\rrbracket(x)$
\end{frame}

\begin{frame}
    \frametitle{GLIF: Grammatical Logical Inference Framework}
    \centering
    \only<1-1>{\disablepart{sempragarrow}}
    \only<1-2>{\disablepart{gfbox}}
    \only<1-2>{\disablepart{mmtbox}}
    \only<1-2>{\disablepart{elpibox}}
    \only<3>{}
    \includestandalone[width=\textwidth]{fig/glif-architecture}
\end{frame}


% \begin{frame}
    \frametitle{Components of GLIF: GF}
    \enablepart{hlgf}
    \includestandalone[width=\textwidth]{fig/glif-architecture}
\end{frame}

\begin{frame}[fragile]
    \frametitle{Components of GLIF: Grammatical Framework \cite{GF:on}}
    \begin{itemize}
        \item Specialized for developing natural language grammars
        \item Separates abstract and concrete syntax\par
            \quad\lstinline[language=GF]|sentence : NP -> VP -> S;               --abstract|\par
            \quad\lstinline[language=GF]|sentence np vp = np.s ++ vp.s!np.n;   --concrete|
        \item Abstract syntax based on type theory
        \item Comes with large library \com{$\ge 36$ languages}
    \end{itemize}

    \vspace{2em}
    \hspace{2.5em}\begin{tikzpicture}
        \node(eng) at (-2,0.7) {\str{Ahmed paints}};
        \node(ger) at (-2,-0.7) {\str{Ahmed zeichnet}};
        \node(ast) at (4,0) {
                    \color{logicfont!50!nlfont}
                    \resizebox{2.cm}{!}{\tikzset{edge from parent/.append style={very thick}}
                        \Tree [ .\texttt{sentence} \texttt{ahmed} \texttt{paint} ]
                    }
                };
        \draw[<->, thick] (eng) to node[above,rotate=-9]{Eng. concr. syn.} (ast);
        \draw[<->, thick] (ger) to node[below,rotate=9]{Ger. concr. syn.} (ast);
    \end{tikzpicture}
\end{frame}

\begin{frame}
    \frametitle{Components of GLIF: MMT}
    \enablepart{hlmmt}
    \includestandalone[width=\textwidth]{fig/glif-architecture}
\end{frame}

\begin{frame}[fragile]
    \frametitle{Components of GLIF: MMT}
    \lstset{frame=single}
    \begin{itemize}   
        \item Modular logic development and knowledge representation
        \item Not specialized in one logical framework \com{we use LF}
        \item We will use MMT to:
        \begin{itemize}
            \item { \only<2>{\bf\color{hlfont}} represent abstract syntax }
            \item { \only<3>{\bf\color{hlfont}} specify target logic and domain theory}
            \item { \only<4>{\bf\color{hlfont}} specify semantics construction}
        \end{itemize}
    \end{itemize}
    \lstset{basicstyle=\footnotesize\ttfamily}
    
    \vspace{1em}
    \begin{minipage}[t][4cm]{\textwidth}
        \centering
        \only<2>{% Used in slides/glif-components.tex
% Has to be in separate file...
\begin{minipage}[t]{0.4\textwidth}
    \parbox[t][1em][t]{\textwidth}{\centering\bf GF}\par
    \begin{lstlisting}[language=GF,linewidth=\textwidth]
cat
  NP; VP; S;
fun
  make_S :
    NP -> VP -> S;
    \end{lstlisting}
\end{minipage}
\begin{minipage}[t]{0.1\textwidth}\vskip3em \centering\LARGE$\mapsto$\end{minipage}
\begin{minipage}[t]{0.35\textwidth}
    \parbox[t][1em][t]{\textwidth}{\centering\bf MMT}\par
    \begin{lstlisting}[language=MMT,linewidth=\textwidth]
NP : type
VP : type
S  : type
make_S :
  NP \raa VP \raa S \end{lstlisting}
\end{minipage}
}
        \only<3>{% Used in slides/glif-components.tex
% Has to be in separate file...
\begin{minipage}[t]{0.45\textwidth}
    \parbox[t][1em][t]{\textwidth}{\centering\bf Logic}\par
    \begin{lstlisting}[language=MMT,linewidth=\textwidth]
o : type //propositions
\neg : o \raa o
\wedge : o \raa o \raa o
\vee : o \raa o \raa o

\iota : type //individuals
\forall : (\iota \raa o) \raa o
\exists : (\iota \raa o) \raa o \end{lstlisting}
\end{minipage}\hskip2em
\begin{minipage}[t]{0.3\textwidth}
    \parbox[t][1em][t]{\textwidth}{\centering\bf Domain Theory}\par
    \begin{lstlisting}[language=MMT,linewidth=\textwidth]
paint : \iota \raa o
quiet : \iota \raa o
ahmed : \iota
berta : \iota \end{lstlisting}

    \footnotesize
    \vspace{1.5em}
    \hskip-1em
    idea:
    $\forall f$
    or $\forall \lambda x.f(x)$\par
    \hskip-1em instead of $\forall x.f(x)$
\end{minipage}
}
        \only<4>{\textbf{Semantics Construction}

\textit{map symbols in abstract syntax to terms in logic/domain theory}

\vspace{0.5em}
\begin{minipage}[t]{0.4\textwidth}
\parbox[t][1em][t]{\textwidth}{\centering Simple setting}\par
\ifpart{switchtomathexample}{
    \lstinputlisting[language=MMT]{slides/misc/snippets/s000.txt}
}{
    \lstinputlisting[language=MMT]{slides/misc/snippets/s001.txt}
}
\end{minipage}
\hspace{1em}
\begin{minipage}[t]{0.5\textwidth}
\parbox[t][1em][t]{\textwidth}{\centering More advanced}\par
\ifpart{switchtomathexample}{
    \lstinputlisting[language=MMT]{slides/misc/snippets/s002.txt}
}{
    \lstinputlisting[language=MMT]{slides/misc/snippets/s003.txt}
}
\end{minipage}
}
    \end{minipage}
\end{frame}

\begin{frame}[fragile]
    \frametitle{Example: Parsing + Semantics Construction}
    {\centering\str{Ahmed and Berta paint}\par}\vspace{1em}
    \hspace{0.49\textwidth}$\downarrow_{\text{parsing}}$\par\vspace{1em}
    {\centering\color{logicfont!50!nlfont} sentence (andNP ahmed berta) paint\par}\vspace{1em}
    \hspace{0.49\textwidth}$\downarrow_{\text{semantics construction}}$\par\vspace{1em}
    {\color{logicfont}\footnotesize\lstinline[language=MMT]|(\lambdan.\lambdav.n v) ((\lambdaa.\lambdab.\lambdap.a p \wedge b p) (\lambdap.p ahmed) (\lambdap.p berta)) paint|}\par\vspace{1em}
    \hspace{0.49\textwidth}$\downarrow_{\text{$\beta$-reduction}}$\par\vspace{1em}
    \hspace{7.5em}{\color{logicfont}\small\lstinline[language=MMT]|paint ahmed \wedge paint berta|}
\end{frame}

\begin{frame}
    \frametitle{Components of GLIF: ELPI}
    \enablepart{hlelpi}
    \includestandalone[width=\textwidth]{fig/glif-architecture}
\end{frame}

\begin{frame}[fragile]
    \frametitle{Components of GLIF: ELPI}
    \begin{itemize}
        \item Implementation and extension of $\lambda$Prolog\com{$\approx$ Prolog + HOAS}
        \item MMT can generate logic signatures
        \item First experiments with prover generation
        \item Generic inference/reasoning step after semantics construction
    \end{itemize}
    \lstset{basicstyle=\footnotesize\ttfamily}

    \vspace{1em}
    \begin{minipage}[t]{\textwidth}
        \centering
        \begin{minipage}[t]{0.5\textwidth}
            \begin{lstlisting}[language=ELPI,frame=single]
kind o type.
type not o -> o.
type and o -> o -> o.

kind i type.
type forall (i -> o) -> o.
            \end{lstlisting}
        \end{minipage}
    \end{minipage}
\end{frame}

\begin{frame}
    \frametitle{Components of GLIF: Jupyter}
    \enablepart{hljupyter}
    \includestandalone[width=\textwidth]{fig/glif-architecture}
\end{frame}

\begin{frame}
    \frametitle{Components of GLIF: Jupyter}
    \begin{itemize}
        \item Unified, notebook-based interface
        \item Supports implementation and testing
        \item Useful for prototype, demos, teaching, \dots
    \end{itemize}

    \centering
    \vspace{1.5em}
    \fbox{\includegraphics[trim={0 0 20cm 5.7cm},clip,width=0.7\textwidth]{img/screenshot-glif-1.png}}
\end{frame}


\begin{frame}[fragile]
    \frametitle{Example: Tableaux Machine~\cite{KohKol:ramgpm03}}
    \begin{itemize}
        \item Can use tableaux for model generation
        \item Tableau machine: pick ``best'' branch as model and continue there with next sentence
            \com{like a human?}
    \end{itemize}

    \vspace{1.5em}
    \begin{minipage}[t][4.5cm]{\textwidth}
        \only<1>{\setshowlevel{1}}
        \only<2>{\setshowlevel{2}}
        \only<3>{\setshowlevel{3}}
        % \only<4>{\setshowlevel{4}}
        \includestandalone{fig/tab-machine-simple}
    \end{minipage}
\end{frame}

\begin{frame}
    \frametitle{Example: Tableaux Machine}
    \only<1>{\setshowlevel{1}}
    \only<2>{\setshowlevel{2}}
    \only<3>{\setshowlevel{3}}
    \only<4>{\setshowlevel{4}}
    \makebox[\linewidth]{\includestandalone[scale=0.9]{fig/tab-machine-complex}}
\end{frame}

\providedisablepart{showscreenshot}
\begin{frame}[fragile]
    \def\mybox#1{\square_{#1}}
    \def\mydia#1{\lozenge_{#1}}
    \def\sfiven{{S5_n}}
    \frametitle{Example: Epistemic Q\&A}
    \centering
    \strplain{\makebox[9.5cm][l]{John knows that Mary or Eve knows that Ping has a dog.}\makebox[1.5em]{\upshape($S_1$)}\\
              \makebox[9.5cm][l]{Mary doesn't know if Ping has a dog.}\makebox[1.5em]{\upshape($S_2$)}\\
              \makebox[9.5cm][l]{Does Eve know if Ping has a dog?}\makebox[1.5em]{\upshape($Q$)}}

    {\color{logicfont}
        \begin{align*}
            S_1 &= \mybox{john}(\mybox{mary} hd(ping)\vee \mybox{eve}hd(ping))\\
            S_2 &= \neg(\mybox{mary}hd(ping) \vee \mybox{mary}\neg hd(ping))\\
            Q &= \mybox{eve}hd(ping) \vee \mybox{eve}\neg hd(ping)
        \end{align*}
    }

    \begin{table}
        \begin{tabular}{l l}
            $S_1, S_2 \vdash_\sfiven Q$\quad      &$\leadsto$\quad yes\\
            $S_1, S_2 \vdash_\sfiven \neg Q$\quad &$\leadsto$\quad no\\
            else &$\leadsto$\quad maybe
        \end{tabular}
    \end{table}
    \ifpart{showscreenshot}{\only<2>{
        \begin{tikzpicture}[overlay,remember picture]
            \fill[gray!80,opacity=0.8] (current page.north west) rectangle (current page.south east);
            \node at (current page.center) { \includegraphics[width=0.9\textwidth]{img/screenshot-glif-4.png} };
        \end{tikzpicture}
    }}{}
\end{frame}




% \appendix
% 
% \begin{frame}[allowframebreaks,t]
%     \frametitle{References}
%     \printbibliography
% \end{frame}




\end{document}
