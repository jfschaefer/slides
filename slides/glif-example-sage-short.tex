\begin{frame}[fragile]
    \frametitle{Example: Input Language for SageMath}
    \begin{itemize}
        \item Can we make a natural input language for SageMath?\com{WolframAlpha-like}
        % \item Idea: Semantics construction translates to SageMath (not logic)
    \end{itemize}

    \vspace{2.0em}
    {\centering\begin{adjustbox}{}
    \begin{lstlisting}
sage: g = AlternatingGroup(5)
sage: g.cardinality()
60
    \end{lstlisting}
    \end{adjustbox}\par
    \vspace{1.5em}
    \str{Let G be the alternating group on 5 symbols. What is the cardinality of G?}\par
    }
\end{frame}

\begin{frame}
    \frametitle{Example: Input Language for SageMath}
    \centering
    \begin{tikzpicture}[xscale=1.05,yscale=0.9]
        \tikzset{ll/.style={line width=0.7pt}}
            \draw[ll,rounded corners=.2cm,fill=black!20] (-3.4,3.2) rectangle (-0.1,0.0);
            \node at (-1.75,0.5) {\bfseries GF};
            \draw[ll,rounded corners=.2cm,fill=black!20] (0.1,3.2) rectangle (3.4,0.0);
            \node at (1.75,0.5) {\bfseries MMT};
            \draw[ll,rounded corners=.2cm,fill=black!20] (3.6,3.2) rectangle (6.9,0.0);
            \node at (5.25,0.5) {\bfseries GF};
        % rectangle and triangles have same area
        \node[ll,fill=\ifcolorful nlbg\else white\fi,draw=\ifcolorful nlfg\else black\fi,minimum width=1.8cm,minimum height=1cm] (utt) at (-3.5,1.8) {$\text{String}_{NL}$};
        \draw[ll,fill=\ifcolorful nlbg!50!logicbg\else white\fi,draw=\ifcolorful nlfg!50!logicfg\else black\fi] (-1,1) -- (0,3) -- (1,1) -- cycle;
        \node[] (st) at (0,1.5) {$\text{AST}_{NL}$};
        % \usetikzlibrary{arrows.meta}
        \draw[ll,-{Straight Barb[length=6.3,width=5.0]}] (-2.5, 1.8) to[bend left=15] node[above] {\small parsing} (-0.8,1.8);
        \draw[thick,fill=\ifcolorful logicbg\else white\fi,draw=\ifcolorful logicfg\else black\fi] (2.5,1) -- (3.5,3) -- (4.5,1) -- cycle;
        \node (qlf) at (3.5,1.5) {$\text{AST}_{\text{Sage}}$};
        \node[ll,draw=blue!50!red!50!black,fill=blue!50!red!25,minimum width=1.8cm,minimum height=1cm] (out) at (7.0,1.8) {$\text{String}_{\text{Sage}}$};
        \draw[ll,-{Straight Barb[length=6.3,width=5.0]}] (4.3, 1.8) to[bend left=15] node[above] {\small linearization} (6.0,1.8);
        \draw[thick,-{Straight Barb[length=6.3,width=5.0]}] (0.8,1.8) to[bend left=15] node[above] {\small semantics} node[below]{\small construction} (2.7,1.8);
    \end{tikzpicture}\par
\end{frame}

\begin{frame}[fragile]
    \frametitle{Example: Input Language for SageMath}
    \lstset{basicstyle=\small\ttfamily,commentstyle={\sffamily\color{nlfg}},morecomment=[l]{Let},morecomment=[l]{What},morecomment=[l]{\#}}
    \begin{lstlisting}[columns=flexible]
> Let G be the alternating group on 5 symbols.
    \end{lstlisting}
    \vskip-1em
    \begin{lstlisting}[commentstyle=\color{blue!50!red!50!black}]
# G = AlternatingGroup(5)
    \end{lstlisting}
    \begin{lstlisting}[columns=flexible]
> Let |H| be a notation for the cardinality of H.
    \end{lstlisting}
    \vskip-1em
    \begin{lstlisting}[commentstyle=\color{blue!50!red!50!black}]
# def bars(H): return H.cardinality()
    \end{lstlisting}
    \begin{lstlisting}[columns=flexible]
> What is |G|?
    \end{lstlisting}
    \vskip-1em
    \begin{lstlisting}[commentstyle=\color{blue!50!red!50!black}]
# print(bars(G))
60
    \end{lstlisting}
    \begin{lstlisting}[columns=flexible]
> Let A_N be a notation for the alternating group on N symbols.
    \end{lstlisting}
    \vskip-1em
    \begin{lstlisting}[commentstyle=\color{blue!50!red!50!black}]
# def A(N): return AlternatingGroup(N)
    \end{lstlisting}
    \begin{lstlisting}[columns=flexible]
> What are the cardinalities of A_4 and A_5?
    \end{lstlisting}
    \vskip-1em
    \begin{lstlisting}[commentstyle=\color{blue!50!red!50!black}]
# print(A(4).cardinality()); print(A(5).cardinality())
12
60
    \end{lstlisting}
\end{frame}

