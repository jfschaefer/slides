\documentclass[aspectratio=169]{beamer}
\usepackage{standalone}

\usepackage{stmaryrd}
\usepackage{listings}
\usepackage{bussproofs}

\usepackage[hyperref=auto,style=alphabetic,backend=bibtex]{biblatex}
\addbibresource{kwarcpubs.bib}
\addbibresource{extpubs.bib}
\addbibresource{extcrossrefs.bib}
\addbibresource{bib.bib}
\usepackage{appendixnumberbeamer}
\usepackage{tikz}
\usepackage{tikz-qtree}
\usetikzlibrary{arrows.meta}
\usetikzlibrary{mmt}
\usetikzlibrary{docicon}

\usetheme{Pittsburgh}
% \setbeamertemplate{footline}[frame number]
\setbeamertemplate{footline}{\hfill\insertframenumber\,/\,\inserttotalframenumber\quad\strut}
\setbeamertemplate{navigation symbols}{}
\usecolortheme{beaver}
\setbeamertemplate{frametitle}[default][left]
% \setbeamersize{text margin left=3em}

\usepackage{utils/colors}
\usepackage[forbeamer]{utils/basic}
\usepackage{utils/operators}
\usepackage{utils/mylstmisc}
\usepackage{utils/lstmmt}

\lstset{basicstyle=\ttfamily}
\lstset{commentstyle=\itshape\color{commentfont}}

\title{Language Research in the KWARC group}

\author{Jan Frederik Schaefer}
\institute{FAU Erlangen-N\"urnberg/KWARC}
\date{\textbf{Workshop: Approaches to the Logic and Syntax of Mathematical Texts}\\Erlangen\\Dec. 6, 2022}

\begin{document}
\frame\titlepage


\begin{frame}
    \frametitle{Corpus work with arxiv at KWARC}
    arxiv.org:
    \begin{itemize}
        \item Open-access pre-print server
        \item $> 2,000,000$ scientific articles
        \item Fields: physics, mathematics, computer science, ...
        \item \LaTeX\ sources
        \item $\leadsto$ a great corpus
    \end{itemize}
\end{frame}

\begin{frame}[fragile]
    \frametitle{Corpus work with arxiv at KWARC}
    \textbf{Problem:}
    \begin{tabular}[t]{r l}
    {This:}& \str{The average is $\frac{A+B}2$.}\\[0.3cm]

    {Could be written like this:}&
    \begin{lstlisting}[language=TeX]
The average is $\frac{A+B}{2}$.
    \end{lstlisting}\\[0.3cm]

    {Or like this:}&
    \begin{lstlisting}[language=TeX]
\def\avg#1#2{\ensuremath{\frac{#1+#2}2}}
    \end{lstlisting}\\
                          &
    \begin{lstlisting}[language=TeX]
% ...
The average is \avg AB.
    \end{lstlisting}\\
    \end{tabular}

    \onslide<3>{
    \vspace{2em}
    \textbf{Solution:}
    Convert to more managable format: HTML with MathML.
    }
    \only<2>{
        \begin{tikzpicture}[overlay,remember picture]
            \fill[gray!80,opacity=0.8] (current page.north west) rectangle (current page.south east);
            \node at (current page.center) { \begin{tabular}{c} \\\includegraphics[width=1.1\textwidth]{img/screenshot_math.png} \\
                        \\
            Example from \cite{https://doi.org/10.48550/arxiv.2210.03463}\end{tabular}};
        \end{tikzpicture}
    }
\end{frame}

\begin{frame}[fragile]
    \frametitle{HTML, MathML}
    \str{The average is $\frac{A+B}{2}$.}

    \begin{lstlisting}[language=HTML]
<p>The average is <math>...</math>.</p>
    \end{lstlisting}

    \vspace{2em}
    \begin{tabular}{p{6cm} | p{6cm}}
        \textbf{Presentation MathML} & \textbf{Content MathML} \\
        \small
        \begin{lstlisting}[language=HTML]
<mfrac>
    <mrow>
        <mi>A</mi>
        <mo>+</mo>
        <mi>B</mi>
    </mrow>
    <mn>2</mn>
</mfrac>
        \end{lstlisting} & 
        \small
        \begin{lstlisting}[language=HTML]
<apply>
    <divide/>
    <apply>
        <plus/>
        <ci>A</ci>
        <ci>B</ci>
    </apply>
    <cn type="integer">2</cn>
</apply>
        \end{lstlisting}
    \end{tabular}
\end{frame}

% \SI{1.5}{\kilo\gram}
% <mrow><mn>1.5</mn><mtext>&nbsp;</mtext><mi>kg</mi></mrow>
% <apply><csymbol cd="latexml">times</csymbol><cn type="float">1.5</cn><csymbol cd="latexml">kilogram</csymbol></apply>

\begin{frame}[fragile]
    \frametitle{ar5iv corpus}
    \begin{itemize}
        \item Use LaTeXML to convert arxiv to HTML+MathML\com{Done by Deyan Ginev}
        \item $\leadsto$ ar5iv corpus\com{$\approx 2\cdot 10^6$ documents}
        \item Goal: Extract semantic information and provide services\com{search, interactive documents, \ldots}
    \end{itemize}

    \vspace{1em}
    \centering
    \only<1>{
        \includegraphics[scale=0.3]{img/unitsscreenshot}
        \footnotesize \\ Screenshot from~\cite{odkd49}\par
    }
    \only<2>{
        \vspace{2em}
        \includegraphics[scale=0.4]{img/declscreenshot}
        \footnotesize \\ Screenshot from~\cite{Schaefer:bsc16}\par
    }
\end{frame}


\begin{frame}
    \centering
    \textit{---And Now for Something Completely Different---}
    
    \vspace{4em}

    \begin{beamercolorbox}[sep=4pt,center]{part title}
      \usebeamerfont{section title}
        GLIF: A tool for prototyping natural language semantics
      \par
    \end{beamercolorbox}
    \vspace{4em}
\end{frame}

\begin{frame}
    \frametitle{Natural Language Semantics (Symbolic)}
    \textbf{For me:}\\
    \hspace{1.5em}Translating natural language into a formal semantic representation (logic).

    \vspace{0.5em}
    \textbf{Example:}\\
    \hspace{1.5em}\str{Every student paints and is quiet.} $\;\;\leadsto\;\;$ {\color{logicfont}$\forall x.s(x)\Rightarrow(p(x)\wedge q(x))$}
    
    \vspace{0.5em}
    \textbf{Rule-based (no ML):}\\
    \hspace{1.5em}Parsing $\;\;\leadsto\;\;$ semantics construction $\;\;\leadsto\;\;$ inference.

    \vspace{1.0em}
    \centering
    \begin{tikzpicture}
        \node (nbnd) at (1.2,0.7) {useless};
        \node (bnd) at (3.5,0.7) {\begin{tabular}{c}also\\interesting\end{tabular}};
        \node (bd) at (3.5,2.0) {AI-complete?};
        \node (nbd) at (1.2,2.0) {\begin{tabular}{c}this\\work\end{tabular}};
        \draw[->,very thick] (-0.3,0) -- (5,0);
        \draw[->,very thick] (0,-0.3) -- (0,2.7);
        \node at (4.0, -0.3) {\itshape breadth};
        \node[rotate=90] at (-0.3,1.8) {\itshape depth};
    \end{tikzpicture}
\end{frame}

\begin{frame}
    \frametitle{Method of Fragments}
    \only<1-2>{
        \centering
        \only<1>{\def\fraglevel{0}}
        \only<2>{\def\fraglevel{1}}
        \includestandalone{fig/montague-fragments}

        \begin{minipage}[t][2cm]{\textwidth}\vspace{1em}
            How do we get from messy language to formal logic?\\[0.5em]
            \emph{Montague}~\cite{Montague:efl70}: Look at a ``nice'' subset
            and map into logic.
        \end{minipage}
    }

    \only<3>{
        \centering
        \def\fraglevel{1}
        \includestandalone{fig/montague-fragments}
        
        \begin{minipage}[t][2cm]{0.6\textwidth}\vspace{1em}
            \str{Ahmed paints and Berta is quiet.}\\[0.5em]
            \str{Ahmed doesn't paint.}
        \end{minipage}\hfill
        \begin{minipage}[t][2cm]{0.39\textwidth}\vspace{1em}
            $p(a) \wedge q(b)$\\[0.5em]
            $\neg p(a)$
        \end{minipage}
    }

    \only<4>{
        \centering
        \def\fraglevel{2}
        \includestandalone{fig/montague-fragments}
        
        \begin{minipage}[t][2cm]{0.6\textwidth}\vspace{1em}
            \str{Every student paints and is quiet.}\\[0.5em]
            \str{Nobody paints.}
        \end{minipage}\hfill
        \begin{minipage}[t][2cm]{0.39\textwidth}\vspace{1em}
            $\forall x.s(x) \Rightarrow (p(x) \wedge q(x))$\\[0.5em]
            $\neg \exists x.p(x)$
        \end{minipage}
    }

    \only<5>{
        \centering
        \def\fraglevel{3}
        \includestandalone{fig/montague-fragments}

        \begin{minipage}[t][2cm]{0.6\textwidth}\vspace{1em}
            \str{Ahmed isn't allowed to paint.}\\[0.5em]
            \str{Ahmed and Berta must paint.}
        \end{minipage}\hfill
        \begin{minipage}[t][2cm]{0.39\textwidth}\vspace{1em}
            $\neg\lozenge p(a)$\\[0.5em]
            $(\square p(a)) \wedge \square p(b)$
        \end{minipage}
    }
\end{frame}


\begin{frame}
    \frametitle{Method of Fragments}
    {\color{hlfont}Hand-waving} is problematic:

    \hspace{2em}\str{Ahmed paints. He is quiet.}
    {$\quad\stackrel{?}{\leadsto}\quad$ \color{logicfont} $p(a)\wedge q(a)$}

    \vspace{1.2em}
    {\color{hlfont}Montague}: Specify
    \begin{itemize}
        \item grammar,\com{fixes NL subset}
        \item target logic,
        \item semantics construction.\com{maps parse trees to logic}
    \end{itemize}

    {
        \centering
        \vspace{0.3em}
        {\itshape\footnotesize Example from~\cite{Montague:tptoqi73}}

        \vspace{0.2em}\fbox{\includegraphics[trim=0 0 0 80,clip,width=0.7\textwidth]{fig/montague-tptoqioe.png}}

        \vspace{1.2em}
        Claim: That doesn't scale well $\leadsto$ \textbf{We need {\color{hlfont}prototyping}!}
    }

    % \newcommand\VP{\text{\upshape\tiny VP}}
    % \hspace{2em}$\llbracket\text{\strplain{$P_{\VP}$ and $Q_{\VP}$}}\rrbracket_{\VP} = \lambda x. \llbracket\text{\strplain{$P_{\VP}$}}\rrbracket(x) \wedge \llbracket\text{\strplain{$Q_{\VP}$}}\rrbracket(x)$
\end{frame}


\begin{frame}
    \frametitle{NLU Prototyping}
    \begin{itemize}
        \item Traditionally done in Prolog/Haskell
        \begin{itemize}
            \item[$\raa$] requires a lot of work
        \end{itemize}
            \item A dedicated framework might be better
        \begin{itemize}
            \item[$\raa$] only partial solutions exist
        \end{itemize}
        \item Can we combine existing partial solutions?\com{Research Question}
        \begin{itemize}
            \item[$\leadsto$] GLIF
        \end{itemize}
    \end{itemize}
\end{frame}



{
    \enablepart{sometimesnarrow}
    \disablepart{showjupyter}
    \disablepart{showepistemicexample}
    \enablepart{mentionprovergen}
    \disablepart{showspa}
    % \disablepart{intensionalexample}
    \begin{frame}
    \frametitle{Components of GLIF: GF}
    \enablepart{hlgf}
    \includestandalone[width=\textwidth]{fig/glif-architecture}
\end{frame}

\begin{frame}[fragile]
    \frametitle{Components of GLIF: Grammatical Framework \cite{GF:on}}
    \begin{itemize}
        \item Specialized for developing natural language grammars
        \item Separates abstract and concrete syntax\par
            \quad\lstinline[language=GF]|sentence : NP -> VP -> S;               --abstract|\par
            \quad\lstinline[language=GF]|sentence np vp = np.s ++ vp.s!np.n;   --concrete|
        \item Abstract syntax based on type theory
        \item Comes with large library \com{$\ge 36$ languages}
    \end{itemize}

    \vspace{2em}
    \hspace{2.5em}\begin{tikzpicture}
        \node(eng) at (-2,0.7) {\str{Ahmed paints}};
        \node(ger) at (-2,-0.7) {\str{Ahmed zeichnet}};
        \node(ast) at (4,0) {
                    \color{logicfont!50!nlfont}
                    \resizebox{2.cm}{!}{\tikzset{edge from parent/.append style={very thick}}
                        \Tree [ .\texttt{sentence} \texttt{ahmed} \texttt{paint} ]
                    }
                };
        \draw[<->, thick] (eng) to node[above,rotate=-9]{Eng. concr. syn.} (ast);
        \draw[<->, thick] (ger) to node[below,rotate=9]{Ger. concr. syn.} (ast);
    \end{tikzpicture}
\end{frame}

\begin{frame}
    \frametitle{Components of GLIF: MMT}
    \enablepart{hlmmt}
    \includestandalone[width=\textwidth]{fig/glif-architecture}
\end{frame}

\begin{frame}[fragile]
    \frametitle{Components of GLIF: MMT}
    \lstset{frame=single}
    \begin{itemize}   
        \item Modular logic development and knowledge representation
        \item Not specialized in one logical framework \com{we use LF}
        \item We will use MMT to:
        \begin{itemize}
            \item { \only<2>{\bf\color{hlfont}} represent abstract syntax }
            \item { \only<3>{\bf\color{hlfont}} specify target logic and domain theory}
            \item { \only<4>{\bf\color{hlfont}} specify semantics construction}
        \end{itemize}
    \end{itemize}
    \lstset{basicstyle=\footnotesize\ttfamily}
    
    \vspace{1em}
    \begin{minipage}[t][4cm]{\textwidth}
        \centering
        \only<2>{% Used in slides/glif-components.tex
% Has to be in separate file...
\begin{minipage}[t]{0.4\textwidth}
    \parbox[t][1em][t]{\textwidth}{\centering\bf GF}\par
    \begin{lstlisting}[language=GF,linewidth=\textwidth]
cat
  NP; VP; S;
fun
  make_S :
    NP -> VP -> S;
    \end{lstlisting}
\end{minipage}
\begin{minipage}[t]{0.1\textwidth}\vskip3em \centering\LARGE$\mapsto$\end{minipage}
\begin{minipage}[t]{0.35\textwidth}
    \parbox[t][1em][t]{\textwidth}{\centering\bf MMT}\par
    \begin{lstlisting}[language=MMT,linewidth=\textwidth]
NP : type
VP : type
S  : type
make_S :
  NP \raa VP \raa S \end{lstlisting}
\end{minipage}
}
        \only<3>{% Used in slides/glif-components.tex
% Has to be in separate file...
\begin{minipage}[t]{0.45\textwidth}
    \parbox[t][1em][t]{\textwidth}{\centering\bf Logic}\par
    \begin{lstlisting}[language=MMT,linewidth=\textwidth]
o : type //propositions
\neg : o \raa o
\wedge : o \raa o \raa o
\vee : o \raa o \raa o

\iota : type //individuals
\forall : (\iota \raa o) \raa o
\exists : (\iota \raa o) \raa o \end{lstlisting}
\end{minipage}\hskip2em
\begin{minipage}[t]{0.3\textwidth}
    \parbox[t][1em][t]{\textwidth}{\centering\bf Domain Theory}\par
    \begin{lstlisting}[language=MMT,linewidth=\textwidth]
paint : \iota \raa o
quiet : \iota \raa o
ahmed : \iota
berta : \iota \end{lstlisting}

    \footnotesize
    \vspace{1.5em}
    \hskip-1em
    idea:
    $\forall f$
    or $\forall \lambda x.f(x)$\par
    \hskip-1em instead of $\forall x.f(x)$
\end{minipage}
}
        \only<4>{\textbf{Semantics Construction}

\textit{map symbols in abstract syntax to terms in logic/domain theory}

\vspace{0.5em}
\begin{minipage}[t]{0.4\textwidth}
\parbox[t][1em][t]{\textwidth}{\centering Simple setting}\par
\ifpart{switchtomathexample}{
    \lstinputlisting[language=MMT]{slides/misc/snippets/s000.txt}
}{
    \lstinputlisting[language=MMT]{slides/misc/snippets/s001.txt}
}
\end{minipage}
\hspace{1em}
\begin{minipage}[t]{0.5\textwidth}
\parbox[t][1em][t]{\textwidth}{\centering More advanced}\par
\ifpart{switchtomathexample}{
    \lstinputlisting[language=MMT]{slides/misc/snippets/s002.txt}
}{
    \lstinputlisting[language=MMT]{slides/misc/snippets/s003.txt}
}
\end{minipage}
}
    \end{minipage}
\end{frame}

\begin{frame}[fragile]
    \frametitle{Example: Parsing + Semantics Construction}
    {\centering\str{Ahmed and Berta paint}\par}\vspace{1em}
    \hspace{0.49\textwidth}$\downarrow_{\text{parsing}}$\par\vspace{1em}
    {\centering\color{logicfont!50!nlfont} sentence (andNP ahmed berta) paint\par}\vspace{1em}
    \hspace{0.49\textwidth}$\downarrow_{\text{semantics construction}}$\par\vspace{1em}
    {\color{logicfont}\footnotesize\lstinline[language=MMT]|(\lambdan.\lambdav.n v) ((\lambdaa.\lambdab.\lambdap.a p \wedge b p) (\lambdap.p ahmed) (\lambdap.p berta)) paint|}\par\vspace{1em}
    \hspace{0.49\textwidth}$\downarrow_{\text{$\beta$-reduction}}$\par\vspace{1em}
    \hspace{7.5em}{\color{logicfont}\small\lstinline[language=MMT]|paint ahmed \wedge paint berta|}
\end{frame}

\begin{frame}
    \frametitle{Components of GLIF: ELPI}
    \enablepart{hlelpi}
    \includestandalone[width=\textwidth]{fig/glif-architecture}
\end{frame}

\begin{frame}[fragile]
    \frametitle{Components of GLIF: ELPI}
    \begin{itemize}
        \item Implementation and extension of $\lambda$Prolog\com{$\approx$ Prolog + HOAS}
        \item MMT can generate logic signatures
        \item First experiments with prover generation
        \item Generic inference/reasoning step after semantics construction
    \end{itemize}
    \lstset{basicstyle=\footnotesize\ttfamily}

    \vspace{1em}
    \begin{minipage}[t]{\textwidth}
        \centering
        \begin{minipage}[t]{0.5\textwidth}
            \begin{lstlisting}[language=ELPI,frame=single]
kind o type.
type not o -> o.
type and o -> o -> o.

kind i type.
type forall (i -> o) -> o.
            \end{lstlisting}
        \end{minipage}
    \end{minipage}
\end{frame}

\begin{frame}
    \frametitle{Components of GLIF: Jupyter}
    \enablepart{hljupyter}
    \includestandalone[width=\textwidth]{fig/glif-architecture}
\end{frame}

\begin{frame}
    \frametitle{Components of GLIF: Jupyter}
    \begin{itemize}
        \item Unified, notebook-based interface
        \item Supports implementation and testing
        \item Useful for prototype, demos, teaching, \dots
    \end{itemize}

    \centering
    \vspace{1.5em}
    \fbox{\includegraphics[trim={0 0 20cm 5.7cm},clip,width=0.7\textwidth]{img/screenshot-glif-1.png}}
\end{frame}

}

\disablepart{aselpientro}
\provideenablepart{aselpientro}
\def\ifelpi#1#2{\ifpart{aselpientro}{#1}{#2}}

\begin{frame}
    \ifpart{aselpientro}{
        \frametitle{ELPI}
        \begin{itemize}
            \item Extension of $\lambda$Prolog\com{supports higher-order abstract syntax}
            \item Generic inference/reasoning step after semantics construction
            \item Goal: Use it for semantic/pragmatic analysis
        \end{itemize}
    }{\frametitle{Example: Discard wrong readings in controlled natural language}}

    \begin{minipage}[t][4cm][t]{\textwidth}
        \ifpart{aselpientro}{
            \pause
            \vspace{2em}
            Example: Discard wrong readings in controlled natural language

            \vspace{1em}
        }{}
        \only<\ifelpi2{1-4}>{
            \tikzset{every picture/.style={line width=0.7pt}}
            \begin{tikzpicture}[yscale=0.5]
                \node(str0) at (-4,0) {\str{the ball has a mass of 5kg}};
                \node(ast0) at (-0.5,0) {AST};
                \node(log0) at (4,0) {\ifcolorful\color{logicfont}\fi$\text{mass}(\text{theball}, \text{quant}(5, \text{kilo gram}))$};
                \draw[-{Straight Barb[length=6.3,width=5.0]},gray] (str0) -- (ast0);
                \draw[-{Straight Barb[length=6.3,width=5.0]},gray] (ast0) -- (log0);
            \end{tikzpicture}
            \vspace{3em}
        }
        \only<\ifelpi32>{\disablepart{crossout}}
        \only<\ifelpi4{3-4}>{\enablepart{crossout}}
        \enablepart{switchtonmexample}
        \onslide<\ifelpi{3-4}{2-4}>{
            \includestandalone[width=\textwidth]{fig/cnl-simple-discard} 
        }
    \end{minipage}

    \only<\ifelpi54>{
        \begin{tikzpicture}[overlay,remember picture]
            \fill[gray!80,opacity=0.8] (current page.north west) rectangle (current page.south east);
            \node at (current page.center) { \includegraphics[width=0.95\textwidth]{img/screenshot-glif-3.png} };
        \end{tikzpicture}
    }
\end{frame}


\providedisablepart{showscreenshot}
\begin{frame}[fragile]
    \def\mybox#1{\square_{#1}}
    \def\mydia#1{\lozenge_{#1}}
    \def\sfiven{{S5_n}}
    \frametitle{Example: Epistemic Q\&A}
    \centering
    \strplain{\makebox[9.5cm][l]{John knows that Mary or Eve knows that Ping has a dog.}\makebox[1.5em]{\upshape($S_1$)}\\
              \makebox[9.5cm][l]{Mary doesn't know if Ping has a dog.}\makebox[1.5em]{\upshape($S_2$)}\\
              \makebox[9.5cm][l]{Does Eve know if Ping has a dog?}\makebox[1.5em]{\upshape($Q$)}}

    {\color{logicfont}
        \begin{align*}
            S_1 &= \mybox{john}(\mybox{mary} hd(ping)\vee \mybox{eve}hd(ping))\\
            S_2 &= \neg(\mybox{mary}hd(ping) \vee \mybox{mary}\neg hd(ping))\\
            Q &= \mybox{eve}hd(ping) \vee \mybox{eve}\neg hd(ping)
        \end{align*}
    }

    \begin{table}
        \begin{tabular}{l l}
            $S_1, S_2 \vdash_\sfiven Q$\quad      &$\leadsto$\quad yes\\
            $S_1, S_2 \vdash_\sfiven \neg Q$\quad &$\leadsto$\quad no\\
            else &$\leadsto$\quad maybe
        \end{tabular}
    \end{table}
    \ifpart{showscreenshot}{\only<2>{
        \begin{tikzpicture}[overlay,remember picture]
            \fill[gray!80,opacity=0.8] (current page.north west) rectangle (current page.south east);
            \node at (current page.center) { \includegraphics[width=0.9\textwidth]{img/screenshot-glif-4.png} };
        \end{tikzpicture}
    }}{}
\end{frame}


% \begin{frame}
    \frametitle{Levels of inference}
    \centering
    \begin{tikzpicture}[yscale=0.9]
        \draw[line width=1.5pt,rounded corners=.3cm,fill=black!10] (-1.9,-1) rectangle (1.5,3);

        \node[fill=nlbg] (nl) at (0,0) {$\mathcal{NL}$};
        \node[fill=logicbg] (L) at (0,2) {$\mathcal{FL}$};
        \node (M) at (0,4) {$\langle \mathcal{FL}, \mathcal{K}, \models\rangle$};
        \node (Inf) at (6,0) {$\models_{\mathcal{T}}$};
        \node (C) at (6,2) {$\vdash_\mathcal{C}$};
        \node (folg) at (6,4) {$\models$};
        \draw[->] (nl) -- node[left] {\begin{tabular}{c}Sem.\\Constr.\end{tabular}} (L);

        \draw[->] (L) -- (M);
        \draw[dotted,->] (nl) -- node[above] {induces} (Inf);
        \draw[dotted,->] (M) -- node[above] {induces} (folg);
        \draw[->] (L) -- node[above] {calculus} (C);
        \draw[<->] (folg) -- node[left]{$\models \; \equiv \; \vdash_\mathcal{C}$?} (C);
        \draw[<->] (C) -- node[left] {$\models_{\mathcal{T}} \; \equiv \; \vdash_\mathcal{C}$?} (Inf);
    \end{tikzpicture}

    \vspace{1em}
    \begin{enumerate}
        \item Test: Does \str{Ahmed and Berta paint.} $\models_\mathcal{T}$ \str{Berta paints.}?
        \item Model prediction: Yes, because {\color{logicfont}$p(a) \wedge p(b)$} $\vdash_\mathcal{C}$ {\color{logicfont}p(b)}.
        \item Correct result: Ask people.
    \end{enumerate}
%     \vspace{1.2em}
%     \begin{tabular}{r r c l}
%         & \str{Ahmed and Berta paint.} & $\stackrel?{\models_\mathcal{T}}$ & \str{Berta paints.}\\[6pt]
%         & {\color{logicfont}$p(a) \wedge p(b)$} & $\vdash_\mathcal{C}$ & {\color{logicfont}p(b)}
%     \end{tabular}
\end{frame}


\begin{frame}[fragile]
    \frametitle{Example: Tableaux Machine~\cite{KohKol:ramgpm03}}
    \begin{itemize}
        \item Can use tableaux for model generation
        \item Tableau machine: pick ``best'' branch as model and continue there with next sentence
            \com{like a human?}
    \end{itemize}

    \vspace{1.5em}
    \begin{minipage}[t][4.5cm]{\textwidth}
        \only<1>{\setshowlevel{1}}
        \only<2>{\setshowlevel{2}}
        \only<3>{\setshowlevel{3}}
        % \only<4>{\setshowlevel{4}}
        \includestandalone{fig/tab-machine-simple}
    \end{minipage}
\end{frame}

\begin{frame}
    \frametitle{Example: Tableaux Machine}
    \only<1>{\setshowlevel{1}}
    \only<2>{\setshowlevel{2}}
    \only<3>{\setshowlevel{3}}
    \only<4>{\setshowlevel{4}}
    \makebox[\linewidth]{\includestandalone[scale=0.9]{fig/tab-machine-complex}}
\end{frame}


% \begin{frame}[fragile]
    \frametitle{Example: Translation}
    \begin{itemize}
        \item Two German words for \str{cousin}, depending on the gender
        \item Two entries in abstract syntax: \verb|cousin_female| and \verb|cousin_male|
        \item Use inference to discard ASTs
    \end{itemize}
    
    \vspace{2em}
    \small
    \begin{minipage}[t][5cm]{\textwidth}
        \begin{tikzpicture}
            \node(eng) at (-4,1) {\parbox{4.2cm}{\str{Kim is Ahmed's cousin and the father of Grace}}};
                \node(ger1) at (-4,-0.5) {\parbox{4.2cm}{\str{Kim ist Ahmeds {\upshape\bf Cousine} und Graces Vater}}};
                \node(ger2) at (-4,-2.0) {\parbox{4.2cm}{\str{Kim ist Ahmeds {\upshape\bf Cousin} und Graces Vater}}};
            \node(ast1) at (0,1) {AST$_1$};
            \node(ast2) at (0,-1) {AST$_2$};
            \only<2->{
                \node(log1) at (3,1) {\color{logicfont} \parbox{2.2cm}{$female(kim) \wedge$ $male(kim)$}};
                \node(log2) at (3,-1) {\color{logicfont} \parbox{2.2cm}{$male(kim) \wedge$ $male(kim)$}};
            }
            \draw[->,thick] (eng) -- (ast1);
            \draw[->,thick] (eng) -- (ast2);
                \draw[->,thick] (ast1) -- (ger1);
                \draw[->,thick] (ast2) -- (ger2);
            \only<2->{
                \draw[->,thick] (ast1) -- (log1);
                \draw[->,thick] (ast2) -- (log2);
            }
            \only<3>{
                \draw[ultra thick,red] (2,1.5) -- (4,0.5);
                \draw[ultra thick,red] (2,0.5) -- (4,1.5);
                
                \draw[ultra thick,red] (-0.5,1.5) -- (0.5,0.5);
                \draw[ultra thick,red] (-0.5,0.5) -- (0.5,1.5);

                \draw[ultra thick,red] (-6,0.0) -- (-2,-1.0);
                \draw[ultra thick,red] (-6,-1.0) -- (-2,0.0);
            }
        \end{tikzpicture}
    \end{minipage}
\end{frame}


\begin{frame}[fragile]
    \frametitle{Example: Input Language for SageMath}
    \begin{itemize}
        \item Can we make a natural input language for SageMath?\com{WolframAlpha-like}
        \item Idea: Semantics construction translates to SageMath (not logic)
    \end{itemize}

    \vspace{1.5em}
    {\centering\begin{adjustbox}{}
    \begin{lstlisting}
sage: g = AlternatingGroup(5)
sage: g.cardinality()
60
    \end{lstlisting}
    \end{adjustbox}\par
    \vspace{1.5em}
    \str{Let G be the alternating group on 5 symbols. What is the cardinality of G?}\par
    }
\end{frame}

\begin{frame}
    \frametitle{Example: Input Language for SageMath}
    \centering
    \begin{tikzpicture}[xscale=1.05,yscale=0.9]
        \tikzset{ll/.style={line width=0.7pt}}
            \draw[ll,rounded corners=.2cm,fill=black!20] (-3.4,3.2) rectangle (-0.1,0.0);
            \node at (-1.75,0.5) {\bfseries GF};
            \draw[ll,rounded corners=.2cm,fill=black!20] (0.1,3.2) rectangle (3.4,0.0);
            \node at (1.75,0.5) {\bfseries MMT};
            \draw[ll,rounded corners=.2cm,fill=black!20] (3.6,3.2) rectangle (6.9,0.0);
            \node at (5.25,0.5) {\bfseries GF};
        % rectangle and triangles have same area
        \node[ll,fill=\ifcolorful nlbg\else white\fi,draw=\ifcolorful nlfg\else black\fi,minimum width=1.8cm,minimum height=1cm] (utt) at (-3.5,1.8) {$\text{String}_{NL}$};
        \draw[ll,fill=\ifcolorful nlbg!50!logicbg\else white\fi,draw=\ifcolorful nlfg!50!logicfg\else black\fi] (-1,1) -- (0,3) -- (1,1) -- cycle;
        \node[] (st) at (0,1.5) {$\text{AST}_{NL}$};
        % \usetikzlibrary{arrows.meta}
        \draw[ll,-{Straight Barb[length=6.3,width=5.0]}] (-2.5, 1.8) to[bend left=15] node[above] {\small parsing} (-0.8,1.8);
        \draw[thick,fill=\ifcolorful logicbg\else white\fi,draw=\ifcolorful logicfg\else black\fi] (2.5,1) -- (3.5,3) -- (4.5,1) -- cycle;
        \node (qlf) at (3.5,1.5) {$\text{AST}_{\text{Sage}}$};
        \node[ll,draw=blue!50!red!50!black,fill=blue!50!red!25,minimum width=1.8cm,minimum height=1cm] (out) at (7.0,1.8) {$\text{String}_{\text{Sage}}$};
        \draw[ll,-{Straight Barb[length=6.3,width=5.0]}] (4.3, 1.8) to[bend left=15] node[above] {\small linearization} (6.0,1.8);
        \draw[thick,-{Straight Barb[length=6.3,width=5.0]}] (0.8,1.8) to[bend left=15] node[above] {\small semantics} node[below]{\small construction} (2.7,1.8);
    \end{tikzpicture}\par
\end{frame}

\begin{frame}[fragile]
    \frametitle{Example: Input Language for SageMath}
    \lstset{basicstyle=\small\ttfamily,commentstyle={\sffamily\color{nlfg}},morecomment=[l]{Let},morecomment=[l]{What},morecomment=[l]{\#}}
    \begin{lstlisting}[columns=flexible]
> Let G be the alternating group on 5 symbols.
    \end{lstlisting}
    \vskip-1em
    \begin{lstlisting}[commentstyle=\color{blue!50!red!50!black}]
# G = AlternatingGroup(5)
    \end{lstlisting}
    \begin{lstlisting}[columns=flexible]
> Let |H| be a notation for the cardinality of H.
    \end{lstlisting}
    \vskip-1em
    \begin{lstlisting}[commentstyle=\color{blue!50!red!50!black}]
# def bars(H): return H.cardinality()
    \end{lstlisting}
    \begin{lstlisting}[columns=flexible]
> What is |G|?
    \end{lstlisting}
    \vskip-1em
    \begin{lstlisting}[commentstyle=\color{blue!50!red!50!black}]
# print(bars(G))
60
    \end{lstlisting}
    \begin{lstlisting}[columns=flexible]
> Let A_N be a notation for the alternating group on N symbols.
    \end{lstlisting}
    \vskip-1em
    \begin{lstlisting}[commentstyle=\color{blue!50!red!50!black}]
# def A(N): return AlternatingGroup(N)
    \end{lstlisting}
    \begin{lstlisting}[columns=flexible]
> What are the cardinalities of A_4 and A_5?
    \end{lstlisting}
    \vskip-1em
    \begin{lstlisting}[commentstyle=\color{blue!50!red!50!black}]
# print(A(4).cardinality()); print(A(5).cardinality())
12
60
    \end{lstlisting}
\end{frame}




\begin{frame}
    \frametitle{Conclusion}
    \begin{minipage}[t]{0.5\textwidth}
        \textbf{Summary:}
        \begin{itemize}
            \item GLIF = GF + MMT + ELPI
            \item Prototyping natural language semantics%\com{(symbolic)}
            \item We use it for teaching
        \end{itemize}
    \end{minipage}
    \begin{minipage}[t]{0.49\textwidth}
        \textbf{Examples:}
        \begin{enumerate}
            \item \str{a kinetic energy of 12mN}
            \item \str{He loves her} (tableaux machine)
            \item \str{John knows that Eve has a dog}
            \item \str{What is the cardinality of G?}
        \end{enumerate}
    \end{minipage}
    
    \vspace{1em}
    \centering
    \includestandalone[width=0.8\textwidth]{fig/glif-architecture}
\end{frame}

\appendix

\begin{frame}
    \frametitle{Pipeline Specification}
    \includestandalone[width=\textwidth]{fig/glif-spec}
\end{frame}



\begin{frame}[allowframebreaks,t]
    \frametitle{References}
    \printbibliography
\end{frame}

\end{document}

