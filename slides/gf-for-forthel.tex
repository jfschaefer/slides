\begin{frame}[fragile]
    \frametitle{GF for ForTheL}
    {
        \centering
        \str{Let a, b be sets that aren't empty}\par
    }
    \vspace{1em}

    \begin{lstlisting}[language=GF]
letAssume : Names -> ClassNoun -> Assume;

-- ClassNoun = {pref : Plurality=>Str; suf : Plurality=>Str};
--   pref: set/sets
--   suf: that isn't empty/that aren't empty

letAssume names cn =
    "let" ++ names.s ++ ("be"|"denote"|"stand for") ++
    indefArt!names.p ++ cn.pref!names.p ++ cn.suf!names.p;
    \end{lstlisting}
\end{frame}

\begin{frame}[fragile]
    \def\logexpr#1{{\color{logicfg}#1}}
    \frametitle{A Closer Look at Notions}
    Example notions: \str{set}, \str{subgroup of $D_8$}, \str{set that isn't empty}

    \vspace{0.5em}
    On the logic side: \logexpr{$set(\cdot)$}, \logexpr{$subgroup(\cdot, D_8)$}, \logexpr{$set(\cdot)\wedge\neg empty(\cdot)$}

    \vspace{0.5em}
    Should \str{subgroup $H$ of $D_8$} be a notion?
    \begin{itemize}
        \item Yes\com{ForTheL does this}
            \begin{itemize}
                \item[\boldmath$+$] Notions are continuous strings
                \item[\boldmath$-$] Semantically tricky
                \item[\boldmath$-$] \str{$G$ is a subgroup $H$ of $D_8$}\com{feels wrong...}
            \end{itemize}
        \item No\com{I did this}
            \begin{itemize}
                \item[\boldmath$+$] Semantically easy
                \item[\boldmath$+$] Definitely isn't only plural/singular yet
                \item[\boldmath$-$] Can't use Resource Grammar Library easily
            \end{itemize}
    \end{itemize}
\end{frame}

