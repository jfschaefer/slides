\documentclass{beamer}

% \usepackage[orientation=landscape,size=custom,width=16,height=9,scale=0.6,debug]{beamerposter}

\usepackage{appendixnumberbeamer}
\usepackage{tikz}
\usetikzlibrary{arrows.meta}
\usetikzlibrary{positioning}
\usetikzlibrary{docicon}

\usetheme{Pittsburgh}
\setbeamertemplate{footline}[frame number]
\setbeamertemplate{navigation symbols}{}
\usecolortheme{beaver}
\setbeamertemplate{frametitle}[default][left]
\setbeamersize{text margin left=3em}
\usepackage{listings}
\usepackage{courier}

\newcommand{\com}[1]{\strut\hfil\strut\null\nobreak\hfill\hbox{{\itshape \color{black!50}#1}}\par}
\newcommand\str[1]{{\color{black!50!green}``\textit{#1}''}}

\title{High-Precision Semantics Extraction in STEM}

\author{Jan Frederik Schaefer \\ Supervisor: Michael Kohlhase}
\institute{FAU Erlangen-N\"urnberg}
\date{\textbf{CICM 2020 --- Doctoral Program} \\ remotely from Erlangen, Germany}


\begin{document}

\begin{frame}
    \maketitle
\end{frame}

\begin{frame}
    \frametitle{My Situation}
    \begin{itemize}
        \item Finishing up my master studies
        \item Did already some research
            \begin{itemize}
                \item Symbolic natural language semantics
                \item Controlled natural languages for mathematics
            \end{itemize}
        \item Going to start my PhD soon
            \begin{itemize}
                \item Supervisor: Michael Kohlhase \com{kwarc group}
                \item Tentative topic: \emph{high-precision semantics extraction in STEM}
                \item Topic still very flexible
                \item[\boldmath$\rightarrow$] \textbf{Any feedback appreciated!}
            \end{itemize}
    \end{itemize}
\end{frame}

\begin{frame}
    \frametitle{Motivation}
    \begin{itemize}
        \item We have large corpora of STEM knowledge \com{e.g. arxiv}
        \item Computers can make it more accessible:
            \begin{itemize}
                \item Unit conversion
                \item Applicable theorem search
                \item Screen readers
                \item \dots
            \end{itemize}
        \item Such services require/benefit from semantic information
        \item Authors often don't provide much \com{semantic \TeX{} macros}
        \item[\boldmath$\rightarrow$] Semantics extraction
    \end{itemize}
\end{frame}

\begin{frame}
    \frametitle{Approaches to Semantics Extraction}
    \begin{itemize}
        \item Machine learning--based
            \begin{itemize}
                \item Training data?
                \item Low precision
            \end{itemize}
        \item Symbolic
            \begin{itemize}
                \item Low coverage
            \end{itemize}
    \end{itemize}

    \vspace{1.5em}
    \centering
    \tikzset{every picture/.style={line width=0.7pt}}
    \begin{tikzpicture}
        \node at (-0.9,2.5) {\itshape precision};
        \node at (4.5, -0.4) {\itshape coverage};
        \node at (0.8, 2.5) {symb.};
        \node at (4.4, 0.6) {ML};
        \node at (4.4, 2.5) {\color{black!50} ???};
        \node at (0.8, 0.6) {\color{black!50} useless};
        \draw[-{Straight Barb[length=6.3,width=5.0]}] (-0.5,0) -- (5.0,0);
        \draw[-{Straight Barb[length=6.3,width=5.0]}] (0,-0.5) -- (0,3.0);
        \only<2->{
            \draw[->,dashed] (1.5,2.5) -- (2.5,2.5);
            \draw[->,dashed] (4.4,1.0) -- (4.4,1.8);
        }
        \onslide<3->{
            \draw[ultra thick, color=blue] (1.43,2.5) ellipse (1.3 and 0.7);
        }
    \end{tikzpicture}
\end{frame}

\begin{frame}
    \frametitle{Plan}
    \begin{itemize}
        \item Symbolic approaches offer high precision
        \item Often, high precision is more important than coverage\\
            \str{each has the mass $\frac12m$} vs \str{each has the mass $1.64ft$}
        \item We already have a tool: GLIF
        \item Use statistical methods to increase coverage
    \end{itemize}
\end{frame}

\begin{frame}
    \frametitle{GLIF (Grammatical Logical Inference Framework)}
    {
        \centering
        \tikzset{every picture/.style={line width=0.7pt}}
        \begin{tikzpicture}
            \node[draw] (str) {\begin{tabular}{c}String\\\color{black!60} $\in \mathcal{FL} \subseteq \mathcal{NL}$\end{tabular}};
            \node[draw,right=of str,minimum width=3cm,minimum height=2cm,rounded corners=0.5cm,fill=black!20] (glif) {\bfseries GLIF};
            \node[draw,right=of glif] (logic) {\begin{tabular}{c}Logical\\Expression\end{tabular}};
            \node[draw,shape=document,above=of glif,color=black!70] (spec) {Specification};
            \draw[-{Straight Barb[length=6.3,width=5.0]}] (str) -- (glif);
            \draw[-{Straight Barb[length=6.3,width=5.0]}] (glif) -- (logic);
            \draw[-{Straight Barb[length=6.3,width=5.0]},color=black!70] (spec) -- (glif);
        \end{tikzpicture}
    }
    
    \vspace{2em}
    Use cases:
    \begin{itemize}
        \item Designing controlled natural languages
        \item Prototyping approaches to natural-language semantics
    \end{itemize}
\end{frame}

\begin{frame}
    \frametitle{GLIF (Grammatical Logical Inference Framework)}
    \resizebox{\textwidth}{!}{
        \begin{tikzpicture}[xscale=1.05,yscale=0.9]
    \tikzset{ll/.style={line width=0.7pt}}

    \draw[ll,rounded corners=.2cm,fill=black!20] (-3.4,3.2) rectangle (-0.1,0.0);
    \draw[ll,rounded corners=.2cm,fill=black!20] (0.1,3.2) rectangle (3.4,0.0);
    \draw[ll,rounded corners=.2cm,fill=black!20] (3.6,3.2) rectangle (6.9,0.0);
    \draw[ll,rounded corners=.2cm] (-4.5,3.3) rectangle (7.0,-1.0);
    \node at (-1.75,0.5) {\bfseries GF};
    \node at (1.75,0.5) {\bfseries MMT};
    \node at (5.25,0.5) {\bfseries ELPI};
    \node at (1.25,-0.5) {\bfseries Jupyter/GLIF};

    % rectangle and triangles have same area
    \node[ll,fill=white,draw,minimum width=1.8cm,minimum height=1cm] (utt) at (-3.5,1.8) {String};
    \draw[ll,fill=white] (-1,1) -- (0,3) -- (1,1) -- cycle;
    \node[] (st) at (0,1.5) {AST};
    \draw[ll,-{Straight Barb[length=6.3,width=5.0]}] (-2.5, 1.8) to[bend left=15] node[above] {\small parsing} (-0.8,1.8);
    \draw[thick,fill=white] (2.5,1) -- (3.5,3) -- (4.5,1) -- cycle;
    \node (qlf) at (3.5,1.5) {\begin{tabular}{c}Logic\\Expression\end{tabular}};
    \draw[thick,-{Straight Barb[length=6.3,width=5.0]}] (0.8,1.8) to[bend left=15] node[above] {\small semantics} node[below]{\small construction} (2.7,1.8);
    \draw[thick,-{Straight Barb[length=6.3,width=5.0]}] (4.4,1.6) .. controls (4.9,1.8) and (4.6,2.4) .. (4.1,2.1);
    \node (inf) at (5.5, 1.9) {\small \begin{tabular}{c}semantic/\\pragmatic\\analysis\end{tabular}};
\end{tikzpicture}

    }

    \vspace{1em}
    \only<1>{
        Idea: Combine existing frameworks
        \begin{itemize}
            \item GF (Grammatical Framework): $\mathcal{NL}$ grammars
            \item MMT: Logic, knowledge representation
            \item ELPI ($\supseteq$ $\lambda$Prolog): Inference
            \item Jupyter: Intuitive UI
        \end{itemize}
    }
    \only<2->{
        \vspace{3em}
        \resizebox{\textwidth}{!}{
            \tikzset{every picture/.style={line width=0.7pt}}
            \begin{tikzpicture}[yscale=0.5]
                \node(str) at (-4,0) {\str{\dots has a mass of 2m}};
                \node(ast1) at (0,1) {AST$_1$};
                \node(ast2) at (0,-1) {AST$_2$};
                \node(log1) at (4,1) {$\lambda x.\text{mass}(x, \text{quant}(2, \text{\bfseries meters}))$};
                \node(log2) at (4,-1) {$\lambda x.\text{mass}(x, \text{mul}(2, \text{\bfseries mVar}))$};
                \draw[-{Straight Barb[length=6.3,width=5.0]},gray] (str) -- (ast1);
                \draw[-{Straight Barb[length=6.3,width=5.0]},gray] (str) -- (ast2);
                \draw[-{Straight Barb[length=6.3,width=5.0]},gray] (ast1) -- (log1);
                \draw[-{Straight Barb[length=6.3,width=5.0]},gray] (ast2) -- (log2);
                \only<3>{
                    \draw[ultra thick,red] (2,0.5) -- (6,1.5);
                    \draw[ultra thick,red] (2,1.5) -- (6,0.5);
                }
            \end{tikzpicture}
        }
    }
\end{frame}

\begin{frame}
    \frametitle{Lexicon Extension}
    {
        \centering\str{Therefore, $A$ is clopen.}\par
    }

    \vspace{2em}
    Different options:
    \begin{enumerate}
        \item Use a dynamic parser \com{DynGenPar}
        \item Generate lexicon automatically:\\
            \small\texttt{clopen\_Adj : Adjective = "clopen"} \\
            \small\texttt{clopen : $\iota \rightarrow o$}
        \item Replace lexicon entries with tokens:\\
            \str{Therefore, $A$ is {\upshape\ttfamily ADJ-1}.}
    \end{enumerate}
\end{frame}

\begin{frame}
    \frametitle{Blanking out Unparsable Parts}

    This may be impossible:

    \str{Let $X$ denote a data set where all entries $x \in X$ are normalized as described above.}

    \vspace{1em}
    This would still be useful:

    \str{Let $X$ denote a data set where {\upshape\ttfamily SUB-CLAUSE}.}
\end{frame}

\begin{frame}
    \frametitle{Semantic Representation?}

    \begin{itemize}
        \item Open question
        \item VIP, Naproche use DRT
        \item DRT not really supported by GLIF/MMT yet
    \end{itemize}
    
    \vspace{2em}
    \resizebox{\textwidth}{!}{
        \begin{tikzpicture}[xscale=1.05,yscale=0.9]
    \tikzset{ll/.style={line width=0.7pt}}

    \draw[ll,rounded corners=.2cm,fill=black!20] (-3.4,3.2) rectangle (-0.1,0.0);
    \draw[ll,rounded corners=.2cm,fill=black!20] (0.1,3.2) rectangle (3.4,0.0);
    \draw[ll,rounded corners=.2cm,fill=black!20] (3.6,3.2) rectangle (6.9,0.0);
    \draw[ll,rounded corners=.2cm] (-4.5,3.3) rectangle (7.0,-1.0);
    \node at (-1.75,0.5) {\bfseries GF};
    \node at (1.75,0.5) {\bfseries MMT};
    \node at (5.25,0.5) {\bfseries ELPI};
    \node at (1.25,-0.5) {\bfseries Jupyter/GLIF};

    % rectangle and triangles have same area
    \node[ll,fill=white,draw,minimum width=1.8cm,minimum height=1cm] (utt) at (-3.5,1.8) {String};
    \draw[ll,fill=white] (-1,1) -- (0,3) -- (1,1) -- cycle;
    \node[] (st) at (0,1.5) {AST};
    \draw[ll,-{Straight Barb[length=6.3,width=5.0]}] (-2.5, 1.8) to[bend left=15] node[above] {\small parsing} (-0.8,1.8);
    \draw[thick,fill=white] (2.5,1) -- (3.5,3) -- (4.5,1) -- cycle;
    \node (qlf) at (3.5,1.5) {\begin{tabular}{c}Logic\\Expression\end{tabular}};
    \draw[thick,-{Straight Barb[length=6.3,width=5.0]}] (0.8,1.8) to[bend left=15] node[above] {\small semantics} node[below]{\small construction} (2.7,1.8);
    \draw[thick,-{Straight Barb[length=6.3,width=5.0]}] (4.4,1.6) .. controls (4.9,1.8) and (4.6,2.4) .. (4.1,2.1);
    \node (inf) at (5.5, 1.9) {\small \begin{tabular}{c}semantic/\\pragmatic\\analysis\end{tabular}};
\end{tikzpicture}

    }
\end{frame}

\begin{frame}
    \frametitle{Late Disambiguation}

    \begin{enumerate}
        \item Syntactic disambiguation:\\
            \vspace{0.3em}
            \begin{minipage}{0.5\textwidth}
                \str{2\textbf{m}}
            \end{minipage}
            $\,\;\rightarrow$ \textit{unit?}
        \item Semantic disambiguation:\\
            \vspace{0.3em}
            \begin{minipage}{0.5\textwidth}
                \str{a mass of 2\textbf{m}}
            \end{minipage}
            $\,\;\rightarrow$ \textit{not a unit}
        \item Later semantic disambiguation:\\
            \vspace{0.3em}
            \begin{minipage}{0.5\textwidth}
                \str{it has a length of 2\textbf{m}, where m is the length of a module}
            \end{minipage}
            $\,\;\rightarrow$ \textit{not a unit}
    \end{enumerate}
\end{frame}


\begin{frame}
    \frametitle{Data Set}
    
    {
        \centering
        \tikzset{every picture/.style={line width=0.7pt}}
        \begin{tikzpicture}
            \node[draw] (arxiv) at (0,0) {\begin{tabular}{c}\textbf{arXiv}\\ \small \LaTeX\end{tabular}};
            \node[draw] (arxmliv) at (6,0) {\begin{tabular}{c}\textbf{arXMLiv}\\ \small HTML + MathML\end{tabular}};
            \draw[-{Straight Barb[length=6.3,width=5.0]}] (arxiv) to node[above] {\textbf{LaTeXML}} (arxmliv);
        \end{tikzpicture}\par
    }

    \vspace{2em}
    \begin{itemize}
        \item $> 10^6$ documents
        \item aims to preserve any semantic information from {\LaTeX} sources
        \item have some experience with processing \textbf{arXMLiv} documents
    \end{itemize}

    \vspace{2em}\centering\url{https://sigmathling.kwarc.info/}\par
\end{frame}

\begin{frame}
    \frametitle{Work Plan}

    \begin{enumerate}
        \item Prototype GLIF pipeline
            \begin{itemize}
                \item Target: variable declarations and uses
                \item Use generated lexicon
            \end{itemize}
        \item Prototype pipeline for corpus work
            \begin{itemize}
                \item Load document
                \item Enter pre-processed sentences into GLIF pipeline
                \item Export results
            \end{itemize}
        \item Introduce blanking out
        \item Scaling
            \begin{itemize}
                \item Larger grammar
                \item More semantic phenomena
            \end{itemize}
        \item Build example semantic services
        \item Can we replace more with ML?\\Can the results be used as training data?
    \end{enumerate}
\end{frame}


\begin{frame}
    \frametitle{Discussion}
    \begin{itemize}
        \item Is it desirable? \com{I think so}
        \item Could this work?
        \item Other ideas?
        \item Anything else?
    \end{itemize}

    \vspace{2em}
    \centering
    \textbf{I haven't started yet \boldmath $\rightarrow$ any feedback is welcome!}
\end{frame}

\end{document}
